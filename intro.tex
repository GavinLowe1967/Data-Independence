
This note seeks an alternative characterisation of data independence.

We consider a \CSPm\ script that is data independent in a type~$t$, including
no equality tests on values of type~$t$.  We make these requirements more
precise below.

Let $P[t]$ be a family of processes parameterised by the type~$t$.  We
consider instantiating the type parameter by two concrete types~$T_1$ and
$T_2$; we write the resulting processes as $P[T_1]$ and $P[T_2]$.  Let $f: T_1
\fun T_2$ be a surjection.  Lift $f$ to events by pointwise application; and
then lift $f$ to traces by pointwise application.  We will prove the following
results.\footnote{The notation $f\inverse \banana{X}$ represents the
  relational inverse image of~$X$ under~$f$, i.e.~$\set{y \| f(y) \in X}$.}
%
\begin{eqnarray}
\traces(P[T_2]) & = & \set{f(tr) \| tr \in \traces(P[T_1])},
  \label{eqn:traces} \\
\failures(P[T_2]) & = &
  \set{(f(tr), X) \| (tr, f\inverse \banana{X}) \in \failures(P[T_1])},
  \label{eqn:failures} \\
\divergences(P[T_2]) & = & \set{f(tr) \| tr \in \divergences(P[T_1])}.
\end{eqnarray}

%%%%%

We now discuss the restrictions we need to place on the usage of~$t$, and of
elements of~$t$, in order to obtain these results.  First, what can we do with
the type $t$ itself?  Figure~\ref{fig:t-allowed} gives some examples.  (We use
these definitions in examples below.)
\framebox{\ldots} 

\begin{definition}
We say that a set or type is \emph{built from~$t$} if
%
\begin{itemize}
\item It is $t$ itself;

\item It is declared as a \CSPM{datatype}, and one of the branches uses a set
  built from $t$;

\item It is defined using a set comprehension, and one of the generators is
  over a set built from~$t$;

\item It is defined as all events over channels that are built from~$t$.
\end{itemize}
%
A channel is \emph{built from~$t$} if its type uses a type built from~$t$.
\end{definition}
 
%%%%%

\begin{figure}[ht]
\begin{cspm}
datatype Option = Some.t | None
channel in : t
channel c1 : t . Z -- Z some other set, independent of t
channel c2 : t . t
channel c2o : t . Option
Y = {| in |}
T1 = { x.z | x <- t, z <- Z } -- Z some other set
\end{cspm}
\caption{Allowable uses of the data independent type $t$. \label{fig:t-allowed}}
\end{figure}

%%%%%

What can we do with a value~$v$ built from~$t$?  A process can: receive a
value over a channel and store it; nondeterministically choose a value and
store it; build a new value (possibly using stored values) and store it; or
send a stored value.  However, no operation can be performed on the value~$v$
that restricts the type of~$t$: in effect, $t$~is a free polymorphic type.

We do not allow the script to contain any constant from~$t$, other than within
the definition of the instantiation of~$t$.  \framebox{**} I
think it would be ok to allow constants as long as all instantiations contain
those constants, and we restrict to functions that are the identity on each
constant. 

Also not functions like \CSPM{seq} \framebox{\ldots}

We also do not allow any equality or inequality tests over values of type~$t$.
Consider the process
\[
c?x \then c?y \then \If x = y \Then a \then STOP \Else b \then STOP
\]
%
This does not satisfy equation~\ref{eqn:traces}.  Let $v$ and~$v'$ be distinct
elements of~$T_1$; then $tr = \trace{c.v, c.v', b} \in traces(P[T_1])$.  But
consider $f: T_1 \fun T_2$ such that $f(v) = f(v') = w$; then the
corresponding trace $f(tr) = \trace{c.w, c.w, b}$ is not a trace of
$P[T_2]$.  If remapping by~$f$ unifies two values, equality tests can affect
the path followed by a process. 

We allow external choices and nondeterministic choices to be indexed over a
set built from~$t$, for example~$\Intchoice x \in t \spot P(x)$.  However, we
do not allow other operators, such as parallel operators, to be indexed over a
set built from~$t$.  We allow arbitrary indexing over sets that are
independent of~$t$. 

When two processes synchronise on a channel built from~$t$, we require that,
for each field from~$t$ in the corresponding events, at least one of the
processes is willing to accept an arbitrary value whenever it is willing to
accept any value.  For example, we allow
\[
c!x \then STOP \parallel[\eset{c}] c?y \then STOP.
\]
However, we do not allow 
\[
c!x \then STOP \parallel[\eset{c}] c!y \then STOP,
\]
for that corresponds to an equality test between~$x$ and~$y$. 

%% Consider the following, in addition to the
%% declarations in Figure~\ref{fig:t-allowed}.
%% %
%% \begin{cspm}
%% datatype Pair = Two.t.t
%% channel two : { Two.x.x | x <- t }
%% P = c?x -> c?y -> two.Two.x.y -> STOP
%% \end{cspm}
%% %
%% The processes $P$ do not satisfy equation~\ref{eqn:failures}, in general.  For
%% suppose $f(A) = f(B) = C$.  Then $P[T_2]$ does not have the failure
%% \[
%% (\trace{c.C, c.C}, \set{two.Two.x.x \| x \in T_2}),
%% \]
%% even though $P[T_1]$ has a corresponding failure
%% \[
%% (\trace{c.A, c.B}, \set{two.Two.x.x \| x \in T_1}).
%% \]

%% \begin{cspm}
%% datatype Triple = Three.t.t.t
%% S3 = union({Three.x.x.y | x <- t, y <- t}, {Three.x.y.x | x <- t, y <- t})
%% channel three: S3
%% P = in?x -> in?y -> three.Three.x.y?z -> STOP
%% \end{cspm}
%% %
%% The processes $P$ do not satisfy equation~\ref{eqn:failures}, in general.
%% Suppose 
%% \[
%% T_1 = \set{A,B,C}, \quad T_2 = \set{C,D}, \quad 
%% f(A) = f(B) = D, \quad f(C) = C.
%% \]
%% Then $P[T_2]$ does not have the failure
%% \[
%% (\trace{\M{in}.D, \M{in}.D},\; \set{\M{three.Three}.D.D.C}),
%% \]
%% since, after that trace, the ``|?z|'' ranges over all values in~$T_1$,
%% including~$C$.  However,  $P[T_1]$ does have a corresponding failure
%% \[
%% (\trace{\M{in}.A, \M{in}.B},\; 
%%   \set{\M{three.Three}.x.y.C \| x \in \set{A,B}, y \in \set{A,B}}),
%% \]
%% because, after that trace, the ``|?z|'' ranges over $\set{A}$.  In effect, the
%% declaration of the type of~|three| has introduced an equality test.
%% ** Simplified below

\paragraph{Restricting sets}

We need to place various restrictions upon how sets based on~$t$ are defined.
In particular, we will have to place restrictions on the use of set
comprehensions. 

We need restrictions on the types of channels.  Consider the following (in
addition to the declarations in Figure~\ref{fig:t-allowed}).\footnote{The use
  of the {\cspmfont PairB} type and {\cspmfont Two} type constructor is
  necessary to meet FDR's requirements that the alphabet of each channel is
  rectangular, i.e.~the cartesian product of sets of complete values; they
  aren't otherwise important for this example.}
\begin{cspm}
datatype PairB = Two.t.t.Bool
channel two : union({Two.x.x.true | x <- t}, {Two.x.y.false | x <- t, y <- t})
P = in?x -> in?y -> two.Two.x.y?b -> STOP
\end{cspm}
%
The $P$ processes do not satisfy equation~(\ref{eqn:failures}), in general.
Suppose 
\[
T_1 = \set{A,B}, \quad T_2 = \set{C}, \quad f(A) = f(B) =  C.
\]
Then $P[T_1]$ has a failure
\[
(\trace{\M{in}.A, \M{in}.B},\; 
  \set{\M{two.Two}.x.y.\M{true} \| x \in \set{A,B}, y \in \set{A,B}}),
\]
because, after that trace, the ``|?b|'' ranges over $\set{\M{false}}$.  
However,  $P[T_2]$ does not have the corresponding failure
\[
(\trace{\M{in}.C, \M{in}.C},\; \set{\M{two.Two}.C.C.\M{true}}),
\]
because, after that trace, the ``|?b|'' ranges over $\set{\M{true,false}}$.
In effect, the declaration of the type of~|two| has introduced an equality
test: the ``|?b|'' is equivalent to 
``\CSPM{?b:(if x==y then \{true,false\} else \{false\})}'', which is not
allowed because of the equality test. 

To avoid situations like this, we will require that in each set
comprehension $\{ e \| stmts\}$ each  identifier $x$ that appears in $e$ and
that depends on~$t$ occurs at most once in~$e$.   This would ban the
expression \CSPM{\{Two.x.x.true \| x <- t\}} that caused problems above,
because~|x| appears twice in the term.

We need a similar restriction for other sets.  Consider the following process,
using hiding. 
\begin{cspm}
P = (in?x -> in?y -> c2.x.y -> STOP) \ {c2.x.x | x <- t}
\end{cspm}
Let $T_1$, $T_2$ and~$f$ be as above.  Then $P[T_2]$ has trace $\seq{in.A,
in.B, c2.A.B}$, but $P[T_1]$ doesn't have the corresponding trace $\seq{in.C,
in.C, c2.C.C}$, contrary to equation~(\ref{eqn:traces}).  We avoid situations
like this in the same way as above.

A similar example shows that we need likewise to restrict the sets used for
alphabets or synchronisation sets in parallel compositions.



%% For simplicity, we do not allow any set expressions using~$t$, other than
%% expressions of the form $\eset{c_1, \ldots, c_k}$ for channel
%% identifiers~$c_1, \ldots, c_k$, or of the form $\eset{C_1, \ldots, C_k}$ for
%% datatype constructors~$C_1, \ldots, C_k$.  \framebox{???}

%% \framebox{**} This could probably be weakened.  I think it would probably be
%% ok to restrict set comprehensions so that each name that is bound by a
%% generator over a set built from~$t$ is used at more once.  This would ban the
%% expression \CSPM{\{Two.x.x.true \| x <- t\}} that caused problems above.  It's
%% enough to ensure $T$-invariance. 
%% %
%% Note we also prevent such bound names appearing within the range of another
%% generator, to prevent expressions such as  
%% \CSPM{\{Two.x.y.true \| x <- t, y <- \{x\}\}}.

%% We  need to restrict operations over sets that are built from~$t$.  We
%% do allow sets to be built that contain values built from~$t$, and allow the
%% \CSPM{union} and \CSPM{empty} functions to be applied to such sets.  However,
%% we do not allow the function \CSPM{member} to be applied to a set built
%% from~$t$, because it can be used to simulate equality tests: the expression
%% $\CSPMM{member}(x, \set{y})$ is equivalent to $x = y$.  Likewise, we do not
%% allow the function \CSPM{card} to be applied to a set built from~$t$: the
%% expression $\CSPMM{card} \set{x,y} = 1$ is equivalent to $x = y$.  Similarly,
%% we do not allow pattern matching over sets built from~$t$ that are based on
%% the cardinality of the set.  We also do not allow the \CSPM{inter} and
%% \CSPM{diff} functions since (when combined with \CSPM{empty}) they can be used
%% to simulate equality tests.

%% We can build sets using comprehensions over sets based on~$t$, for example,
%% the set |T1| in Figure~\ref{fig:t-allowed}.  However, we cannot build
%% sequences in the same way.  An expression such as \CSPM{< x.3 \| x <- t >}
%% doesn't type check, since |t| is a set, not a sequence.  And, as noted above,
%% we cannot use the function |seq| to convert a set into a sequence.  


\begin{definition}
\label{def:invariant}
We say that a definition of a set is \emph{invariant} if it is of one of the
following forms:
%
\begin{enumerate}
\item A definition that is independent of~$t$.

\item A set comprehension $\{e \| stmts\}$ or $\eset{e \| stmts}$, such  that
for every identifier $x$ that appears in $e$ and that depends on~$t$:
  % 
  \begin{itemize}
  \item $x$~is bound by a generator $x \,\M{<-}\, s_x$ in $stmts$, where $s_x$ is
  itself invariant and uses no identifier bound by an earlier generator;
  \item $x$~occurs only once in~$e$; and
  \item $x$ is not used in any predicate in $stmts$.
  \end{itemize}

\item The name of a datatype (including $t$). 

\item \CSPM{union(S1, S2)} where |S1| and |S2| are invariant forms. 

\item ......
\end{enumerate}
\end{definition}

I think the set comprehension case includes simple set definitions.

Can the generator range over a larger set?  Would need to prevent any other
generator depending on~$x$. 

Note that the restriction that the set of a generator doesn't use an earlier
bound variable prevents expressions such as 
\CSPM{\{c2.x.y \| x <- t, y <- \{x\}\}}. 


%% \framebox{Omit following?}

%% We restrict all channel declarations
%% \begin{cspm}
%% channel c : A£$_1$£.£$\cdots$£.A£$_k$£
%% \end{cspm}
%% and all branches of datatypes
%% \begin{cspm}
%% datatype D = £$\cdots$£ | C.A£$_1$£.£$\cdots$£.A£$_k$£ | £$\cdots$£
%% \end{cspm}
%% such that each set |A|$_i$ is either:
%% %
%% \begin{itemize}
%% \item a set that is independent of~$t$; 

%% \item a datatype; or

%% \item an expression of the form $\eset{C_1, \ldots, C_j}$, where $C_1, \ldots,
%%   C_j$ are constructors of some datatype.
%% \end{itemize}

%% Conjecture: for every top-level set-valued expression $e$ in the script,
%% \[
%% \eval ~ \rho_1 ~ e = f\inverse \banana{\eval ~ (f\after\rho_1)~e}.
%% \]
%% Prove that if $\rho(x) = f\inverse \banana{f(\rho(x))}$ for every identifier~$x$
%% associated with a set, then the above holds for~$\rho$.  Effectively this
%% means piece-wise symmetric in~$t$. 

%% ................


We also need a restriction for renamings.  Consider
%
\begin{cspm}
channel c : t.t
P = (in?x -> in?y -> c.x.y -> STOP)[[ c.x.x <- d | x <- t ]]
\end{cspm}
%
Let $T_1 = \set{A,B}$,\, $T_2 = \set{C}$ and $f(A) = f(B) = C$.  Then $P[T_1]$
has trace $\trace{in.A, in.B, c.A.B}$.  But $P[T_2]$ does not have the
corresponding trace $\trace{in.C, in.C, c.C.C}$.  We prevent renamings such as
``\CSPM{c.x.x <- d}'', which contains an implicit equality test. 




%% stable failure \[ (\trace{in.A, in.B}, \set{d}).  \] However, $P[T_2]$ does
%% not have the corresponding stable failure \[ (\trace{in.C, in.C}, \set{d}) \]
