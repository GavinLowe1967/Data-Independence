\section{Evaluation}

We assume a well-typed error-free script.  \framebox{**} Error-free for~$T_1$? 

We begin by defining the set of values associated with expressions in a
script.  The set $Value$ contains the union of the following. 
%
\begin{itemize}
\item Basic values: integers, booleans, and characters.

\item Datatype constructors and channel names.

\item Dotted tuples $v_1.\ldots.v_n$: in such values, we assume each
  component~$v_i$ is not a dotted term; i.e.~we treat the dot operator as
  associative, and ``flatten'' nested dots.  We sometimes write $\varepsilon$
  for the empty dotted tuple, i.e.~the unit of dot.

\item Sequences $\seq{v_1,\ldots,v_n}$, sets
  $\set{v_1,\ldots,v_n}$, tuples $(v_1,\ldots,v_n)$ and mappings
  $\set{v_1 \mapsto v_1', \ldots, v_n \mapsto v_n'}$, where each $v_i, v_i'$
  is a value.

\item Functions, considered as sets of maplets ($v \mapsto v'$); we sometimes
  represent functions using lambda abstractions.

\item Augmented LTSs (defined below), representing processes.

\item For each channel declaration $\CSPMM{channel}~c : e_1\M.\ldots\M.e_n$, a
  channel value $\channel S$, where $S$ is the set of values that can be
  passed on~$c$, i.e.~all values of the form $v_1.\ldots.v_n$ associated with
  $e_1.\ldots.e_n$.  In the case $n=0$, we use the value
  $\channel \set\varepsilon$.  

\item For each datatype declaration
\[
\CSPMM{datatype}~D = 
  A_1.e_{1,1}.\ldots.e_{1,m_1} \mid \ldots \mid A_n.e_{n,1}.\ldots.e_{n,m_n}
\]
datatype constructor values $\dtcons S_i$ corresponding to~$A_i$, for
$i = 1,\ldots,n$, where $S_i$ is the set of values of the form
$v_1.\ldots.v_{m_i}$ associated with $e_{i,1}.\ldots.e_{i,m_i}$.  In the case
$m_i=0$, we use the value $\dtcons\set\varepsilon$.

\item Distinguished values $\error$ representing an error, and $\bottom$
  representing a non-terminating computation.
\end{itemize}

We use an \emph{environment} mapping identifiers (variables) to values:
\[
Env = Var \pfun Value.
\]
We write $\rho$, $\rho'$, etc., for environments.  An environment can
associate top-level names with values, for example corresponding to the
declaration of channels.  If $c$ is the name of a channel, we will sometimes
write $v \in \rho(c)$ to mean $v \in S$ where $\rho(c) = \channel S$.
In addition, an environment can give the value of, for example:
%
\begin{itemize}
\item a name that corresponds to a formal parameter of a function;
\item a name bound by a ``|let| \ldots |within|'' clause;
\item a name bound by an input or nondeterministic choice in a process;
\item a name for an indexing variable in an indexed process operator;
\item a name bound by a generator in a set comprehension.
\end{itemize}


The type $Expr$ represents \CSPm\ expressions.  The semantics of expressions is
defined using a function $\eval : Env \fun Expr \pfun Value$ such that
$\eval \rho~e$ gives the value of expression~$e$ in environment~$\rho$.

For later convenience, we define the semantics of processes in terms of
\emph{augmented LTSs}, where each state is an element of $(Proc, Env)$,
i.e.~a \CSPm\ syntactic process expression (maybe with free variables) and
an environment giving values to free variables.  For example, three successive
states of a process might be
\[
\begin{array}{rl}
(\M{in?x -> in?y -> c2.x.y -> STOP}, & \rho_1) \\
(\M{in?y -> c2.x.y -> STOP}, & \rho_1 \oplus \set{\M{x} \mapsto v_1}) \\
(\M{c2.x.y -> STOP}, & \rho_1 \oplus \set{\M x \mapsto v_1, \M y \mapsto v_2}),
\end{array}
\]
where $\rho_1$ is the environment corresponding to top-level declarations, for
example declarations of channels.  For a process expression~$P$, the value
$\eval \rho~P$ is an LTS with initial state $(P,\rho)$.  For simplicity, we
assume that different identifiers (e.g.~in different parallel components) have
distinct names: this can be achieved via an alpha-renaming.

We define a standard partial order~$\sqsubseteq$ over $Value$, with bottom
element~$\bottom$, and such that all the datatype constructors are continuous.

Let $f : T_1 \fun T_2$.  We extend $f$ to other
values in the obvious way.
\begin{itemize}
\item For values~$x$ not depending on~$t$ we have $f(x) = x$: this includes
  basic values (e.g.integers and booleans), names of channels and datatype
  constructors, $\varepsilon$, $\bottom$ and $\error$.

\item We lift $f$ to dotted tuples ---including events--- by
  $f(v_1.\ldots.v_n) = f(v_1).\ldots.f(v_n)$.  This is well defined provided
  \framebox{???} (see below).

\item We lift $f$ to tuples, sets, sequences and maps by point-wise
  application.

\item We lift $f$ to functions, considered as sets of maplets, by
  \framebox{???} 
  %% $f (g) = \set{ f(x) \mapsto f(y) \mid x \mapsto y \in g }$;
  %% equivalently, in terms of a lambda-abstraction,
  %% $f(g) = \lambda z \cdot f(g(f\inverse(z)))$.

\item For channel and datatype constructor values we define
  $f(\channel S) = \channel (f(S))$, and
  $f(\dtcons S) = \dtcons (f(S))$.

\item For convenience, we lift $f$ to environments by functional
  composition, i.e.~$f(\rho) = f\after\rho$.

\item We lift $f$ to LTS states by point-wise application, so $f(Q,\rho) =
  (Q, f\after\rho)$. 

\item We lift $f$ to augmented LTSs by application of~$f$ to the states
  and to the events of the transitions.  Note that, as $f$ is not necessarily
  injective, this might merge some states. 
\end{itemize}

The following definition captures our main assumption about the CSP script.
%
\begin{definition}
\label{defn:data-independent}
A CSP script is \emph{data-independent} for~$t$ if
\begin{enumerate}
\item The only constants from~$t$ that appear are within the definition
  of the instantiation of~$t$.

\item There are no equality or inequality tests over values built from~$t$.

\item When two processes synchronise on a channel built from~$t$, for each
  field from~$t$ in the corresponding events, at least one of the processes is
  willing to accept an arbitrary value.

\item\label{item:di-invariant} Every expression that produces a set is
  invariant (Definition~\ref{def:invariant}) for: (1) top-level declarations
  (which includes declarations used for channels and datatypes); (2) sets used
  in alphabets and synchronisation sets for parallel composition, hiding, and
  |?x:S| fields in prefixing.

\item\label{item:di-renaming} For each renaming $\rename{e_1 \lArrow e_1',
  \ldots, e_k \lArrow e_k' ~ \M\| stmts}$, and for each~$i \in
  \set{1,\ldots,k}$, every identifier that appears in $e_i$ and that depends
  on~$t$:
%
\begin{itemize}
\item is bound by a generator $x \,\M{<-}\, t$ in $stmts$, \framebox{more
  general?}

\item occurs at most once in~$e_i$, and

\item is not used in any predicate in $stmts$.
\end{itemize}
\framebox{cf. Definition~\ref{def:invariant}}

\item The script does not apply any of the built-in functions
  %%  \CSPM{member}, \CSPM{card}, \CSPM{inter}, \CSPM{diff}, \CSPM{seq},
  %% \CSPM{mapToList},
  \CSPM{mtransclose} or \CSPM{show} to a value built from~$t$.
  \framebox{\ldots}. 

\item The script makes no use of any compression function.
  %% s \CSPM{deter},
  %% \CSPM{chase} or \CSPM{chase_nocache}.
\end{enumerate}
\end{definition}

%%%%%%%%%%%%%%%%%%%%%%%%%%%%%%%%%%%%%%%%%%%%%%%%%%%%%%%

\subsection{Invariance of sets}

Recall from the introduction that we need to place restrictions on expressions
that define sets, mainly concerning the use of bound variables in set
comprehensions.  In this section, we present a corresponding semantic
condition on the set produced, and show that it is satisfied under the
conditions of Definition~\ref{def:invariant}.

Informally, we want to say that, for each instantiation~$T$ of~$t$, if a value
$A \in T$ can appear in a particular place in a channel or datatype value,
then replacing that instance of~$A$ by a different value~$B$ also gives a
valid channel or datatype value.  For example, for the definitions in
Figure~\ref{fig:t-allowed}, if $A \in T$, then $\M{c2}.A.A$ is a valid value;
and so is $\M{c2}.A.B$ for $B \in T$.  The following definition captures this
formally.
%
\begin{definition}
Let $T$ be an instantiation of~$t$, let $\unit$ be a fresh value, and let
$\unitf : T \fun \set{\unit}$ be the function that maps each element of~$T$
to~$\unit$. Extend~$\unitf$ to other values in the normal way.  Then a set $S$
is \emph{$T$-invariant} if
\[
\forall w, w' \spot
  \unitf(w) = \unitf(w') \land w \in S  \implies w' \in S.
\]
\end{definition}

%%%%%

\begin{example}
Suppose $\M{c2}.A.A \in \eset{\M{c2}}$.  Then 
\[
\unitf(\M{c2}.A.A) = \M{c2}.\unit.\unit = \unitf(\M{c2}.A.B),
\]
so $\set{A,B}$-invariance of $\eset{\M{c2}}$ would require $\M{c2}.A.B \in
\eset{\M{c2}}$, which is the case, for $B \in T$.

However, recall the declaration (which is forbidden by our assumptions)
%
\begin{cspm}
channel two : union({Two.x.x.true | x <- t}, {Two.x.y.false | x <- t, y <- t})
\end{cspm}
%
The set $\eset{\M{two}}$ is not $\set{A,B}$-invariant: it includes
|two.Two.A.A.true|, but not |two.Two.A.B.true|; and
\[
\unitf(\M{two.Two.A.A.true}) = \M{two.Two}.u.u.\M{true} =
   \unitf(\M{two.Two.A.B.true}).
\]
\end{example}

%%%%%

\begin{definition}
An environment~$\rho$ is \emph{$T$-invariant} if
%
\begin{itemize}
\item For every channel identifier~$c$, either $\rho(c) = \bottom$, or
  $\rho(c) = \channel S$ for a set~$S$ that is $T$-invariant;

\item For every datatype constructor~$C$, either $\rho(C) = \bottom$, or
  $\rho(C) = \dtcons S$ for a set~$S$ that is $T$-invariant;

\item For every name~$x$ such that $\rho(x)$ is a set, $rho(x)$ is
  $T$-invariant. 
\end{itemize}
\end{definition}

%%%%%

Fix an instantiation $T$ of $t$.  We show that the syntactic property of
invariance implies the semantic property of $T$-invariance. 

\begin{lemma}
\label{lem:invariant}
Let $T$ be an instantiation of~$t$.  Let $\rho$ be a $T$-invariant environment
such that $\rho(t) = T$. 
%
Consider a particular expression~$s$ that defines a set and is invariant.
Then $\eval \rho~s$ is $T$-invariant.
\end{lemma}

\begin{proof}
We prove the result by induction over the structure of~$s$, and perform a case
analysis over the form of~$s$, following Definition~\ref{def:invariant}.  We
concentrate on the interesting cases.

Consider a set comprehension $s = \M\{ e ~ \M\| stmts \M\}$.  Let $S = \eval
\rho~s$.  Suppose $w \in S$ and $\unitf(w) = \unitf(w')$.  Consider the fields
$w_1,\ldots,w_k$ in~$w$ of types that depend upon~$t$.  By the assumption of
invariance, these correspond to distinct identifiers $x_1,\ldots,x_k$ in~$e$,
each of which is bound by a generator $x_i \,\M{<-}\, s_i$ in $stmts$, where
$s_i$ is itself invariant.  Let $S_i = \eval \rho_i~s_i$, where $\rho_i$ be
the environment~$\rho$ augmented corresponding to values taken by earlier
generators in the production of~$w$; so $w_i \in S_i$.  By induction, $S_i$ is
$T$-invariant.

Let $w_1',\ldots,w_k'$ be the corresponding fields in~$w'$.  We show that the
identifiers $x_1,\ldots,x_k$ could also be bound to $w_1',\ldots,w_k'$ by the
generators, with all other generators being bound as in the production of~$w$.
Consider a generator $x_i \,\M{<-}\, s_i$.  Let $\rho_i'$ be the
environment~$\rho$ augmented corresponding to values taken by earlier
generators.  But, by assumption, $s_i$ uses none of the earlier bound
variables~$x_1,\ldots,x_{i-1}$, and so $\eval \rho_i'~s_i = \eval \rho_i~s_i =
S_i$.  And $w_i' \in S_i$, since $S_i$ is $T$-invariant, so this generator can
indeed produce~$w_i'$.  Further, the identifiers $x_1,\ldots,x_k$ do not
appear in predicates in $stmts$, so each such predicate will still be true
with this binding.  Then evaluating $e$ produces~$w'$, so $w' \in S$, as
required.

Now consider an extensions set comprehension $\M{\{\|} e ~ \M\| stmts \M{\|\}}$.
Necessarily, the expression~$e$ represents either a channel or a partial value
from a datatype.  This expression represents the same set as a set
comprehension where the ``missing'' fields are added with extra generators;
for example, the expression \CSPM{\{\|c2o\|\}}  defines the same set as
\CSPM{\{c2o.x.y \| x <- t, y <- Option\}}.  By the assumption that $\rho$ is
$T$-invariant, the generators range over sets that are themselves
$T$-invariant.  Hence we can apply the result for standard set comprehensions
to deduce the result for the extensions set comprehension. 

Other cases from Definition~\ref{def:invariant} are straightforward.
\end{proof}




%% \begin{definition}
%% Consider a script instantiating~$t$ with~$T$.  Let $\rho$ be a corresponding
%% environment.  We say that $\rho$ is \emph{$T$-invariant} if:
%% %
%% \begin{itemize}
%% \item For every channel~$c$ declared in the script, $\eval \rho ~ \eset{c}$
%%   is $T$-invariant;

%% \item For every datatype~$d$ declared in the script, $\eval \rho ~ d$ is
%%   $T$-invariant;

%% \item For every constructor $C$ of a datatype declared in the script,
%%   $\eval \rho ~ \eset{C}$  is $T$-invariant.
%% \end{itemize}
%% \end{definition}


%% \begin{lemma}
%% \label{lem:T-invariant}
%% Consider a script instantiating~$t$ with~$T$.  Then  the environment~$\rho_1$
%% formed from the top-level declarations is $T$-invariant. 
%% %
%% %% \begin{itemize}
%% %% \item For every channel~$c$ declared in the script, $\eval \rho_1 ~ \eset{c}$
%% %%   is $T$-invariant;
%% %% \item For every datatype~$d$ declared in the script, $\eval \rho_1 ~ d$ is
%% %%   $T$-invariant;
%% %% \item For every constructor $C$ of a datatype declared in the script,
%% %%   $\eval \rho_1 ~ \eset{C}$  is $T$-invariant.
%% %% \end{itemize}
%% %
%% Channel and datatypes are defined only at the top level; hence the same is
%% true of each subsequent environment. 
%% \end{lemma}

%%%%%

%% \begin{proof}
%% We prove the result by induction over the declarations in the script.  We
%% prove the case for datatypes; the cases for channels and constructors are
%% similar but more straightforward.

%% Suppose $w \in \eval \rho_1 ~ d$ and $\unitf(w) = \unitf(w')$.  Suppose $w$
%% comes from a branch $C.S_1.\ldots.S_k$ of the definition of~$d$, so $w$ is of
%% the form $C.w_1.\ldots.w_k$ where $w_i \in \eval \rho_1 ~ S_i$ for each~$i$.
%% Since $\unitf(w) = \unitf(w')$,\, $w'$ must use the same constructor, and be
%% of the form $C.w_1'.\ldots.w_k'$, where $\unitf(w_i) = \unitf(w_i')$ for
%% each~$i$.  We show $w_i' \in \eval \rho_1 ~ S_i$, for each~$i$, via a case
%% analysis.
%% %
%% \begin{itemize}
%% \item If $S_i$ is a set that is independent of~$t$, when $w_i = \unitf(w_i) =
%%   \unitf(w_i') = w_i'$, so $w_i' \in \eval \rho_1 ~ S_i$.

%% \item If $S_i$ is a datatype, then, by induction, $\eval \rho_1 ~ S_i$ is
%%   $T$-invariant; hence  $w_i' \in \eval \rho_1 ~ S_i$. 

%% \item If $S_i$ is of the form $\eset{C_1, \ldots, C_j}$, where $C_1, \ldots,
%%   C_j$ are constructors of some datatype, then by induction, each of $\eval
%%   \rho_1 ~ \eset{C_1}, \ldots, \eval \rho_1 ~ \eset{C_j}$ is $T$-invariant;
%%   hence again $w_i' \in \eval \rho_1 ~S_i$.
%% \end{itemize}
%% %
%% Hence $w' = C.w_1'.\ldots.w_k' \in \eval \rho d$, as required.
%% \end{proof}

%%%%%%%%%%

  
Consider a data-independent script with top-level declarations $decls$.
Lemma~\ref{lem:invariant} had a premise that the environment is $T$-invariant.
We now use this to show that the environment~$\rho_1$ formed from $decls$ is
$T$-invariant.

The semantics of a declaration is defined by a function $\bindDecl : Env \fun
Decl \pfun Env$, where $Decl$ is the type of declarations: $\bindDecl
\rho~decl$ gives the environment formed by updating~$\rho$ according to~$decl$. 
%
The declarations might be mutually recursive, so their semantics is defined as
a fixed point.  One step of the iteration for that fixed point is given by the
function
%
\begin{eqnarray*}
F(\rho) & = & \rho \oplus \Union \set{\bindDecl \rho ~ decl \| decl \in decls}.
\end{eqnarray*}
%
Let $\rho^\bottom$ be the environment that maps all identifiers bound by
$decls$ to~$\bottom$.  Then the semantics of $decls$ is the least fixed point
of~$F$: 
\begin{eqnarray*}
\rho_1 & = & \bigsqcup_{n \in \nat} F^n(\rho^\bottom).
\end{eqnarray*}

Clearly $\rho^\bottom$ is $T$-invariant.  
%
Suppose $\rho$ is invariant, and consider a top-level declaration $x = s$;
so $s$ is invariant, by item~\ref{item:di-invariant} of
Definition~\ref{defn:data-independent}.  Then Lemma~\ref{lem:invariant} shows
that $\eval \rho~s$ is $T$-invariant, so 
\begin{eqnarray*}
\bindDecl \rho~(x = s) & = & \rho \oplus \set{x \mapsto \eval \rho~s}
\end{eqnarray*}
is $T$-invariant.  A similar argument applies to channel and datatype
declarations.  Hence if $\rho$ is $T$-invariant, then so is $F(\rho)$.
%
Then a straightforward induction shows that $F^n(\rho^\bottom)$ is
$T$-invariant, for all~$n$.  It follows that $\rho_1$ is $T$-invariant.




%%%%%%%%%%%%%%%%%%%%%%%%%%%%%%%%%%%%%%%%%%%%%%%%%%%%%%%

%% \begin{lemma}
%% Consider a set comprehension $s = \{ e \| stmts\}$.  Suppose that every
%% identifier $x$ that appears in $e$ and that depends on~$t$:
%% %
%% \begin{itemize}
%% \item is bound by a generator $x \,\M{<-}\, t$ in $stmts$,
%% \item occurs only once in~$e$, and
%% \item is not used in any predicate in $stmts$.
%% \end{itemize}
%% %
%% Then $\eval \rho~s$ is $T$-invariant for every well-typed environment~$\rho$.
%% \end{lemma}

%% %%%%%

%% \begin{proof}
%% Let $S = \eval \rho~s$.  Suppose $w \in S$ and $\unitf(w) = \unitf(w')$.
%% %% Necessarily, $w = \eval \rho_w~e$ for some $\rho_w \in \evalStmts \rho~stmts$.
%% Consider the fields $w_1,\ldots,w_k$ in~$w$ of type~$t$.  Necessarily, these
%% correspond to distinct identifiers $x_1,\ldots,x_k$ in~$e$, each of which is
%% bound by a generator $x_i \,\M{<-}\, t$ in $stmts$.  Let $w_1',\ldots,w_k'$ be
%% the corresponding fields in~$w'$.  Then the identifiers $x_1,\ldots,x_k$ could
%% also be bound to $w_1',\ldots,w_k'$ by the generators.  These identifiers do
%% not appear in predicates in $stmts$, so each such predicate will still be true
%% with this binding.  Then evaluating $e$ produces~$w'$, so $w' \in S$, as
%% required.
%% \end{proof}



%%%%%%%%%%%%%%%%%%%%%%%%%%%%%%%%%%%%%%%%%%%%%%%%%%%%%%%

%% \framebox{***} Also need hidden sets and alphabets to be $T$-invariant


%%%%%%%%%%%%%%%%%%%%%%%%%%%%%%%%%%%%%%%%%%%%%%%%%%%%%%%%%%%%

\subsection{Main result}

The following is the main result of this section.
%
\begin{prop}
\label{prop:expressions}
Suppose a script is data-independent for~$t$, let $T_1$ and~$T_2$ be two
instantiations of~$t$, and let $f : T_1 \fun T_2$ be surjective.  Then for
every expression~$e$ in the script, and every environment~$\rho$,
\begin{eqnarray*}
f(\eval \rho ~ e) & = \eval (f \after \rho) e .
\end{eqnarray*}
\end{prop}
%
The proof is in Appendix~\ref{sec:proof}.
