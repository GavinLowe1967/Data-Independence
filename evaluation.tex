\section{Evaluation}

We now describe how \CSPm\ expressions are evaluated.  In
Section~\ref{sec:values} we define the set of values that an expression can
represent.  In Section~\ref{sec:lifting-f} we describe how a function $f: T_1
\fun T_2$ is lifted to apply to other values.  Then in
Section~\ref{sec:semantics} we describe the functions that define the
semantics of \CSPm\ expressions.  
%
We assume a well-typed script throughout.

%%%%%%%%%%%%%%%%%%%%%%%%%%%%%%%%%%%%%%%%%%%%%%%%%%%%%%%

\subsection{Values}
\label{sec:values}

We begin by defining the set of values associated with expressions in a
script.  The set $Value$ contains the union of the following. 
%
\begin{itemize}
\item Basic values: integers, booleans, characters and strings.

\item Datatype constructors and channel names.

\item Dotted tuples $v_1.\ldots.v_n$: in such values, we assume each
  component~$v_i$ is not a dotted term; i.e.~we treat the dot operator as
  associative, and ``flatten'' nested dots.  We sometimes write $\varepsilon$
  for the empty dotted tuple, i.e.~the unit of dot.

\item Sequences $\seq{v_1,\ldots,v_n}$, sets
  $\set{v_1,\ldots,v_n}$, tuples $(v_1,\ldots,v_n)$ and mappings
  $\set{v_1 \mapsto v_1', \ldots, v_n \mapsto v_n'}$, where each $v_i, v_i'$
  is a value.

\item Functions, considered as sets of maplets ($v \mapsto v'$).

\item Augmented labelled transition systems (defined below), representing
  processes.

\item For each channel declaration $\CSPMM{channel}~c : e_1\M.\ldots\M.e_n$, a
  channel value $\channel S$, where $S$ is the set of values that can be
  passed on~$c$, i.e.~all values of the form $v_1.\ldots.v_n$ associated with
  $e_1.\ldots.e_n$; in the case $n=0$, we use the value $\channel
  \set\varepsilon$.

\item For each datatype declaration
\[
\CSPMM{datatype}~D = 
  A_1.e_{1,1}.\ldots.e_{1,m_1} \mid \ldots \mid A_n.e_{n,1}.\ldots.e_{n,m_n}
\]
datatype constructor values $\dtcons S_i$ corresponding to~$A_i$, for $i =
1,\ldots,n$, where $S_i$ is the set of values of the form $v_1.\ldots.v_{m_i}$
associated with $e_{i,1}.\ldots.e_{i,m_i}$; in the case $m_i=0$, we use the
value $\dtcons\set\varepsilon$.

\item Distinguished values $\error$ representing an error, and $\bottom$
  representing a non-terminating computation.
\end{itemize}


We define a standard partial order~$\sqsubseteq$ over $Value$, with bottom
element~$\bottom$, and such that all the datatype constructors are continuous.

%%%%%%%%%%


We use an \emph{environment} mapping identifiers (names) to values:
\[
Env = Name \pfun Value.
\]
We write $\rho$, $\rho'$, etc., for environments.  An environment can
associate top-level names with values, for example corresponding to the
declaration of channels.  If $c$ is the name of a channel, we will sometimes
write $v \in \rho(c)$ to mean $v \in S$ where $\rho(c) = \channel S$.
In addition, an environment can give the value of, for example:
%
\begin{itemize}
\item the name of a datatype or a constructor of a datatype, including the
  distinguished type~|T|;
\item a name that corresponds to a formal parameter of a function;
\item a name bound by a ``|let| \ldots |within|'' clause;
\item a name bound by an input or nondeterministic choice in a process;
\item a name for an indexing variable in a replicated process operator;
\item a name bound by a generator in a set comprehension.
\end{itemize}

For later convenience, we define the semantics of processes in terms of
\emph{augmented labelled transition systems (ALTS)}.  An ALTS is a labelled
transition system, where each state is an element of $Proc \cross Env$, i.e.~a
\CSPm\ syntactic process expression (maybe with free variables) and an
environment giving values to (at least) the free variables.  For example,
three successive states of a process might be
\[
\begin{array}{rl}
(\M{in?x -> in?y -> c2.x.y -> STOP}, & \rho_1) \\
(\M{in?y -> c2.x.y -> STOP}, & \rho_1 \oplus \set{\M{x} \mapsto v_1}) \\
(\M{c2.x.y -> STOP}, & \rho_1 \oplus \set{\M x \mapsto v_1, \M y \mapsto v_2}),
\end{array}
\]
where $\rho_1$ is the environment corresponding to top-level declarations, for
example declarations of channels.  

More precisely, an ALTS is a tuple $(S,E,s_0)$, where
%
\begin{itemize}
\item $S \subseteq Proc \cross Env$ is a set of states;

\item $E \subseteq S \cross (\Sigma \union \set{\tau, \tick}) \cross S$ is a
  set of edges, labelled with events;

\item $s_0 \in S$ is the initial state.
\end{itemize}
%
The value of a process expression~$P$ in environment~$\rho$, then, is an ALTS
with initial state $(P,\rho)$ (and with transitions defined in a standard
way).  Consider a state $(Q,\rho')$ in this ALTS, and consider the sub-ALTS
with initial state $(Q,\rho')$; that sub-ALTS is itself the value of
process~$Q$ in environment~$\rho'$.  We assume that each ALTS is connected:
every state is reachable from the initial state. 


For simplicity, we assume that different identifiers (e.g.~in different
parallel components) have distinct names: this can be achieved via an
alpha-renaming.


%%%%%%%%%%%%%%%%%%%%%%%%%%%%%%%%%%%%%%%%%%%%%%%%%%%%%%%

\subsection{Lifting $f$}
\label{sec:lifting-f}

Let $f : T_1 \fun T_2$.  We extend $f$ to other
values in the obvious way.
\begin{itemize}
\item For values~$x$ not depending on~|T| we have $f(x) = x$: this includes
  basic values (e.g.~integers and booleans), names of channels, names of
  datatype constructors other than those in~$T_1$, and the distinguished values
  $\varepsilon$, $\bottom$ and $\error$.

\item We lift $f$ to dotted tuples, including events, by point-wise
  application: $f(v_1.\ldots.v_n) = f(v_1).\ldots.f(v_n)$.  A consequence of
  our main result in Section~\ref{sec:main-result} is that if $v_1.\ldots.v_n$
  is an event on channel~$v_1$ in the script associated with~$T_1$, then
  $f(v_1).\ldots.f(v_n)$ is an event on channel~$f(v_1)$ in the script
  associated with~$T_2$, and similarly for elements of datatypes.

\item We lift $f$ to tuples, sets, sequences and maps by point-wise
  application.

\item For channel and datatype constructor values we define
  $f(\channel S) = \channel (f(S))$, and
  $f(\dtcons S) = \dtcons (f(S))$.

\item We lift $f$ to functions, considered as sets of maplets, by pointwise
  application: $f (g) = \set{ f(x) \mapsto f(y) \mid x \mapsto y \in g }$.  In
  Section~\ref{sec:functions} we present conditions on~$g$ for this to be well
  defined, i.e.~giving a function; and we show that all functions defined in
  the script satisfy these conditions as part of our main result in
  Section~\ref{sec:main-result}. 

\item We lift $f$ to environments as for other functions: $f(\rho) =
  \set{{f(x) \mapsto f(y)} \| x \mapsto y \in \rho}$.  Note that for all
  names~$x$ other than elements of the distinguished type, $f(x) = x$.  When
  applied to a name from the distinguished type, the maplet $x \mapsto
  \dtcons\set\epsilon$ is replaced by $f(x) \mapsto \dtcons\set\epsilon$; note
  that this makes $f(\rho)$ well defined, i.e.~a function.

\item We lift $f$ to a state of an ALTS by point-wise application, so
  $f(Q,\rho) = (Q, f(\rho))$.

\item We lift $f$ to a transition on an ALTS by point-wise application, so
  $f(S,a,S') = (f(S), f(a), f(S'))$.

\item We lift $f$ to an ALTS by application of~$f$ to the states and to the
  transitions of the transitions.  Note that, as $f$ is not necessarily
  injective, this might merge some states.
\end{itemize}


%%%%%%%%%%%%%%%%%%%%%%%%%%%%%%%%%%%%%%%%%%%%%%%%%%%%%%%

\subsection{Semantics}
\label{sec:semantics}

\CSPm\ uses the following syntactic categories.
%
\begin{itemize}
\item The type $Expr$ represents \CSPm\ expressions, both process expressions
  and non-process expressions.

\item The type $Pat$ represents  patterns used in pattern matching, including
  constant patterns (e.g.~|3|), variable patterns (e.g.~|x|), wildcard
  patterns (\CSPM{\_}), tuple patterns (e.g.~|(3,x)|), dotted patterns
  (e.g.~\CSPM{3.x.\_}), set patterns (e.g.~\CSPM{\{x\}}), sequence patterns
  (e.g.~|<x> ^ xs|), and double patterns~(e.g.~\CSPM{pair @@ (x,y)}).

\item The type |Decl| represents  declarations (e.g.~|P = STOP|,\, \CSPM{f(x) =
  x+1},\, \CSPM{datatype T = A \| B}, or |channel c: T|.

\item The type |Stmt| represents statements in set or sequence comprehensions,
  and replicated operators: either generators (e.g.~|x <- s|), or predicates
  (e.g.~|x < 3|).  Note that in some cases, a statement produces a
  \emph{sequence} (and each generator must range over a sequence): we call
  these \emph{sequence statements}.  And in other cases, a statement produces
  a \emph{set} (and each generator must range over a set): we call these
  \emph{set statements}.
\end{itemize}

%% The semantics of expressions is
%% defined using a function $\eval : Env \fun Expr \pfun Value$ such that
%% $\eval \rho~e$ gives the value of expression~$e$ in environment~$\rho$.

The semantics is defined using the following functions:
%
\begin{itemize}
\item $\eval : Env \fun Expr \pfun Value$ such that $\eval \rho~e$ gives the
  value of expression~$e$ in environment~$\rho$.

\item $\matches : Env \fun Pat \fun Value \fun Bool$ such that
  $\matches \rho~pat ~ v$ tests whether pattern~$pat$ can be matched by
  value~$v$.

\item $\bind_0, \bind : Env \fun Pat \fun Value \pfun Env$ such that
  $\bind_0 \rho~pat~v$ gives the updates to the environment by binding
  pattern~$pat$ to value~$v$, and is defined when $\matches \rho~ pat~v$; and 
  $\bind \rho~pat~v$ gives the result of those updates, i.e.
  \begin{eqnarray*}
  \bind \rho~pat~v & = & \rho \oplus \bind_0 \rho~pat~v.
  \end{eqnarray*}
% Do we need $\rho$ for $bind_0$?  Yes, we do: we need to know whether an
% identifier represents a datatype constructor or a channel. 

\item $\bindDecl : Env \fun Decl \pfun Env$ such that $\bindDecl \rho~decl$,
  gives the update to the environment corresponding to the
  declaration~$decls$.  And $\bindDecls : Env \fun Decl^* \pfun Env$ such that
  $\bindDecls \rho~decls$, updates $\rho$ corresponding to the list of
  declarations $decls$, assuming that all names bound by the declarations are
  distinct, and that each pattern matches the corresponding expression.  The
  declarations might be mutually recursive, so the semantics is defined as a
  least fixed point:
  \[
  \begin{align}
  \bindDecls \rho~decls = \bigsqcup_{n \in \nat} F^n(\rho^\bottom) \\
  \mbox{where } F(\rho_1)  =  
    \rho_1 \oplus \Union \set{\bindDecl \rho_1 ~ decl \| decl \in decls},
  \end{align}
  \]
  where $\rho^\bottom$ extends $\rho$ by mapping all names bound by $decls$
  to~$\bottom$. 

\item $\evalStmt: Env \fun Stmt \fun Env^*$, and $\evalStmts : Env \fun Stmt^*
  \fun Env^*$, such that $\evalStmt \rho~stmt$ gives the sequence of
  environments resulting from evaluating the sequence statement~$stmt$, and
  $\evalStmts$ does likewise for a list of statements.

\item $\evalStmtSet : Env \fun Stmt \fun \power(Env)$, and $\evalStmtsSet :
  Env \fun Stmt^* \fun \power(Env)$ that operate similarly over set statements
  (and produce a set of environments).
\end{itemize}

%%%%%%%%%%%%%%%%%%%%%%%%%%%%%%%%%%%%%%%%%%%%%%%%%%%%%%%

We now consider the role of the declaration of the distinguished type~|T| in
a script.
%
\begin{definition}
\label{def:initial-env}
Consider a script that declares the distinguished type |T| as
\begin{cspm}
  datatype T = A£$_1$£ | ... | A£$_m$£
\end{cspm}
and let $T = \set{A_1,\ldots, A_m}$.  Then we write $\rho^{T}$ for the
environment that results from evaluating just that declarations:
\begin{eqnarray*}
\rho^T = \set{\M T \mapsto T, \M A_1 \mapsto \dtcons\set\epsilon, \ldots,
   \M A_m \mapsto \dtcons\set\epsilon}. \\
\end{eqnarray*}
\end{definition}
%
The semantics of the script is equal to the result of evaluating (using
$\bindDecls$) the remainder of the declarations in the script in~$\rho^T$.

Now consider two scripts that are identical other than their definitions of the
distinguished type~|T|.
The following lemma captures a relationship between the initial environments.
%
\begin{lemma}
\label{lem:distinguished-type-declaration}
Let $\rho^{T_1}$ and $\rho^{T_2}$ be environments corresponding to
declarations of~|T| as $T_1$ and $T_2$, respectively.  Suppose $f: T_1 \fun
T_2$ is surjective.  Then $f(\rho^{T_1}) = \rho^{T_2}$.
\end{lemma}
%
\begin{proof}
In $f(\rho^{T_1})$, the maplet $\M T \mapsto T_1$ is replaced by $\M T \mapsto
f(T_1) = \M T \mapsto T_2$ (since $f$ is surjective), matching $\rho^{T_2}$.
And each maplet $A \mapsto \dtcons\set\epsilon$, for $A \in T_1$, is replaced
by a maplet $f(A) \mapsto \dtcons\set\epsilon$; but these match the
corresponding maplets in $\rho^{T_2}$ (again, because $f$ is surjective).
\end{proof}

%%%%%%%%%%

We show below that other values defined in the two scripts, including all
subsequent environments, are likewise related by~$f$.

Part of our main result, proved as part of Proposition~\ref{prop:expressions},
is that for every expression~$e$ in the script, and for every
environment~$\rho$:
\begin{eqnarray*}
f(\eval \rho ~ e) & = \eval (f(\rho)) ~e .
\end{eqnarray*}
%
In particular, this implies $f(\eval \rho_1 ~ e) = \eval \rho_2 ~e$.  Note that
the former is built using~$T_1$ then renamed; the latter is built using~$T_2$.

In the case of a process expression~$P$, we will require a slightly stronger
result, namely that each transition of $\eval \rho ~ P$ is matched by a
transition of $\eval (f(\rho)) ~P$, and vice versa.  The definition below
captures the relationship formally.  Here and henceforth, when dealing with
ALTSs defined over~$T_1$ and~$T_2$, we write, respectively, $\trans{}_{T_1}$
and $\trans{}_{T_2}$ for their transition relations.  
%
\begin{definition}
\label{def:f-bisim}
Given ALTSs $L_1$ and $L_2$ over~$T_1$ and~$T_2$, respectively, we say $L_1$
and~$L_2$ are \emph{$f$-bisimilar}, written $L_1 \bisim_f L_2$ if
%We say that $(P,\rho)$ \emph{respects~$f$} if:
%
\begin{enumerate}
\item For every transition $(P, \rho) \trans{a}_{T_1} (P', \rho')$ in~$L_1$,
  there is a transition $(P, f(\rho)) \trans{f(a)}_{T_2} (P', f(\rho'))$
  in~$L_2$;

\item For every transition $(P, f(\rho)) \trans{b}_{T_2} (P', \rho')$
  in~$L_2$, there is a transition $(P, \rho) \trans{a}_{T_1} (P', \rho'')$
  in~$L_1$ such that $f(a) = b$ and $f(\rho'') = \rho'$.
%% \[
%% \exists a, \rho'' \spot
%%   (P, \rho) \trans{a}_{T_1} (P', \rho'') \land f(a) = b \land f(\rho'') = \rho'.
%% \]

\item if $s_1$ and $s_2$ are the initial states of $L_1$ and $L_2$,
   then $f(s_1) = s_2$. 
\end{enumerate}
%We say that an ALTS respects~$f$ if every state respects~$f$. 
\end{definition}

%%%%%%%%%%

As part of Proposition~\ref{prop:expressions}, we show $\eval \rho ~ P
\bisim_f \eval (f(\rho)) ~P$ for each process expression~$P$.  This
will allow us to deduce equations~(\ref{eqn:traces})--(\ref{eqn:fd}),
concerning the denotational semantics of that process.

The following lemma shows that if two ALTSs are $f$-bisimilar, there is an
obvious correspondence between the states.  Recall that we assume the ALTSs
are connected: every state is reachable from the initial state.
%
\begin{lemma}
\label{lem:f-bisim-implies-states}
If $L_1 \bisim_f L_2$ then
%
\begin{enumerate}
\item For each state $(Q,\rho)$ in $L_1$, there is a state $(Q,f(\rho))$
  in~$L_2$;

\item For each state $(Q,\rho')$ in $L_2$, there is a state $(Q,\rho)$
  in~$L_1$ such that $\rho' = f(\rho)$.
\end{enumerate}
\end{lemma}
%
\begin{proof}
Each clause can be proven by induction over the length of the path leading to
the presumed state, and constructing a corresponding path to the required
state in the other ALTS. 
\end{proof}

%%%%%%%%%%

\begin{lemma}
\label{lem:ALTS-respects-f-implication}
If $\eval \rho~P \bisim_f \eval (f(\rho))~P$, then $f(\eval \rho~P) = \eval
(f(\rho))~P$.
\end{lemma}

\begin{proof}
Each of $f(\eval \rho~P)$ and $\eval (f(\rho))~P$ has initial state
$(P,f(\rho))$.  We show that the transitions of the ALTSs match, and hence the
subsequent states match.

Suppose in  $f(\eval \rho~P)$ there is a transition $(Q,\rho_1)
\trans{b} (Q',\rho_1')$.  Then in $\eval \rho~P$ there is a transition
$(Q,\rho_0) \trans{a}_{T_1} (Q',\rho_0')$ such that $f(a) = b$,\, $f(\rho_0) =
\rho_1$ and $f(\rho_0') = \rho_1'$.  But $\eval \rho~P \bisim_f \eval
(f(\rho))~P$
% $(Q,\rho_0)$ respects~$f$ 
so $(Q, f(\rho_0)) \trans{f(a)}_{T_2} (Q',f(\rho_0'))$, i.e.~$(Q,\rho_1)
\trans{b}_{T_2} (Q',\rho_1')$.

Now suppose in $\eval (f(\rho))~P$ there is a transition $(Q,\rho_1)
\trans{b}_{T_2} (Q',\rho_1')$.  By Lemma~\ref{lem:f-bisim-implies-states},
there is a  state $(Q,\rho_0)$ in $\eval \rho~P$ such that $\rho_1 =
f(\rho_0)$.  But $\eval \rho~P \bisim_f \eval (f(\rho))~P$,
% $(Q,\rho_0)$ respects $f$,
so there is a transition $(Q,\rho_0) \trans{a}_{T_1} (Q',\rho_0')$ such that
$f(a) = b$ and $f(\rho_0') = \rho'$.  So $f(\eval \rho~P)$ has a transition
$(Q,f(\rho_0)) \trans{f(a)} (Q',f(\rho_0'))$, i.e., $(Q,\rho_1) \trans{b}
(Q',\rho_1')$.
\end{proof}

%%%%%

The following example shows that the converse of the above lemma doesn't hold
in general.  It considers a process~$P$ that satisfies $f(\eval \rho~P) =
\eval (f(\rho))~P$, but $\eval \rho~P \not\bisim_f \eval (f(\rho))~P$.  The
example considers a process that does not satisfy our assumptions of
Definition~\ref{defn:data-independent}: this is necessary since part of
Proposition~\ref{prop:expressions} shows that for every process~$P$ that
satisfies Definition~\ref{defn:data-independent}, $\eval \rho~P
\bisim_f \eval (f(\rho))~P$.
%
\begin{example}
Consider the process 
\begin{eqnarray*}
P = in?x?y \then \If x = y \Then a \then STOP \Else STOP,
\end{eqnarray*}
and let $T_1 = \set{A,B}$, and $f(A) = f(B) = C$.  
%
Consider the state $(Q,\rho_1)$, for~$T_1$, where
%
\begin{eqnarray*}
Q & = & \If x = y \Then a \then STOP \Else STOP, \\
\rho_1 & =  & \set{x \mapsto A, y \mapsto B}.
\end{eqnarray*}
%
Then the state $(Q, \rho_1)$ has no transition to match the transition
\[
(Q, f(\rho_1)) = (Q, \set{x \mapsto C, y \mapsto C}) \trans{a}_{T_2} 
  (STOP, \set{x \mapsto C, y \mapsto C}),
\]
so  $\eval \rho~P \not\bisim_f \eval (f(\rho))~P$.

Nevertheless, $f(\eval \rho~P) = \eval (f(\rho))~P$.  For example, applying
$f$ to the ALTS merges the two states, $(Q,\rho_1)$ and $(Q,\rho_2)$, where
$\rho_2 = \set{x \mapsto A,\linebreak[1] {y \mapsto A}}$, since $f(\rho_1) =
f(\rho_2)$: this gives a state with a transition 
\[
(Q, \set{{x \mapsto C}, y \mapsto C}) \trans{a}
   (STOP, \set{x \mapsto C, y \mapsto C}),
\]
matching the transition in $\eval (f(\rho))~P$.
\end{example}

%%%%%%%%%%

Before we are able to state our main result, as
Proposition~\ref{prop:expressions}, we need some additional results.

Recall that we assume that set expressions that appear in particular places
satisfy the property of invariance (Definition~\ref{def:invariant}).  In
Section~\ref{sec:invariance} we show that the sets defined by those
expressions satisfy a corresponding semantic property, informally that
elements of the distinguished type are interchangable; in particular, this
includes the alphabet of all channels.  This result will be useful
subsequently. 

In Section~\ref{sec:functions} we identify a particular property of each
function defined in the script, which implies that applying~$f$ to that
function returns a function; part of Proposition~\ref{prop:expressions} shows
that all functions defined by the script do indeed satisfy this property.

