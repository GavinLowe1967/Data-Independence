\section{Main proposition}
\label{sec:proof}

%% \begin{prop}
%% \label{prop:expressions-1}
%% Suppose a script is data-independent for~$t$.  Let $T_1$ and~$T_2$ be two
%% instantiations of~$t$, let $\rho: Var \fun T_1$ be an environment
%% corresponding to the script (so $\rho$ is $T_1$-invariant), and let $f : T_1
%% \fun T_2$ be surjective.  Then:
%% %
%% \begin{enumerate}
%% \item\label{item:matches} Let $pat$ be a pattern, and $v$ a value; then
%% \begin{eqnarray*}
%%   \matches \rho~ pat~v & \iff & \matches (f(\rho)) ~ pat~(f(v)).
%% \end{eqnarray*}

%% \item\label{item:bind}
%% Let $pat$ be a pattern, and $v$ a value; then
%% \begin{eqnarray*}
%% f  (\bind_0 \rho~pat~v) &  = & \bind_0 (f(\rho))~pat~(f(v)),
%% \end{eqnarray*}
%% and hence
%% \begin{eqnarray*}
%% f  (\bind \rho~pat~v) & = & \bind (f(\rho))~pat~(f(v)).
%% \end{eqnarray*}

%% \item\label{item:bindDecls} Suppose $decls$ is a list of declarations of
%%   values, channels and datatypes, with disjoint names.  Then
%%   \begin{eqnarray*}
%%   f (\bindDecl \rho~decls) & = & \bindDecl (f(\rho))~decls.
%%   \end{eqnarray*}
%%   and hence
%%   \begin{eqnarray*}
%%   f (\bindDecls \rho~decls) & = & \bindDecls (f(\rho))~decls.
%%   \end{eqnarray*}

%% Also every function produced respects~$f$. 

%% \item\label{item:eval} For every expression~$e$ in the script,
%% \begin{eqnarray*}
%% f(\eval \rho ~ e) & = \eval (f(\rho)) ~e .
%% \end{eqnarray*}

%% \item\label{item:evalStmtsSet}
%% For every sequence~$stmts$ of statements (generators or qualifiers), and every
%% environment~$\rho$\,  
%% \begin{eqnarray*}
%% f (\evalStmtsSet \rho ~ stmts) & = & \evalStmtsSet (f(\rho))~stmts.
%% \end{eqnarray*}
%% %
%% \framebox{***} As sets, but not as sequences. 

%% \item \label{item:evalStmts}
%% Under some circumstances, the previous also holds for sequences and
%% multisets. 
%% \begin{eqnarray*}
%% f (\evalStmts \rho ~ stmts) & = & \evalStmts (f(\rho))~stmts.
%% \end{eqnarray*}

%% \end{enumerate}
%% \end{prop}

%%%%%%%%%%

In this appendix, we prove Proposition~\ref{prop:expressions}, by induction
over the syntax of \CSPm.  The proof is split over the following subsections.
Several cases are as in~\cite{symmetry-TR}, but we include them for
completeness.

%%%%%%%%%%
