\subsection{Declarations}
\label{sec:bindDecls}

We now prove clause~\ref{item:bindDecls}.  We consider the various subclauses
in the subsections below.

%%%%%%%%%%

\subsubsection{Single declarations}

We start with $\bindDecl$, and show
%
\begin{eqnarray*}
  f(\bindDecl \rho~decl) & = & \bindDecl (\frho)~decl.
\end{eqnarray*}
%
We proceed by a case analysis over~$decl$.  

\paragraph{Case $pat = e$.}

We have
\begin{calc}
  & f(\bindDecl \rho~(pat = e)) \\
  = & f(
        \begin{align}
        \Let v = \eval \rho~e \In \\
        \If \matches \rho~pat~v \Then \bind_0 \rho~pat~v \Else \error)
        \end{align} \\
  = & \begin{align}
      \Let v = \eval \rho~e \In \\
        \If \matches \rho~pat~v \Then f(\bind_0 \rho~pat~v)
        \Else f(\error)
      \end{align}  \\
  = & \com{clauses~\ref{item:matches} and \ref{item:bind}; 
    $f(error) = error$} \\
  & \begin{align}
    \Let v = \eval \rho~e \In \\
     \If \matches (\frho)~pat~(f(v)) 
      \Then \bind_0 (\frho)~pat~(f(v)) 
        \Else \error
    \end{align} \\
  = & \com{letting $w = f(v)$} \\
    & \begin{align}
      \Let w = f(\eval \rho~e) \In \\
        \If \matches (\frho)~pat ~w 
        \Then \bind_0 (\frho)~pat~w
        \Else \error
      \end{align}  \\
  = & \com{clause~\ref{item:eval}} \\
    & \begin{align}
      \Let w = \eval (\frho)~e \In \\
      \If \matches (\frho)~pat ~w \Then \bind_0 (\frho)~pat~w
        \Else \error
      \end{align} \\
  = & \bindDecl (\frho)~(pat=e).
\end{calc}

%%%%%%%%%%

\paragraph{Case of a function definition.}

Consider, first, a function~$g$ of a single argument with $m>0$ sub-arguments.
The definition can use $n>0$ clauses, each clause using patterns for the
sub-arguments.  We write $\vec{pat}_i$ for the tuple of $m$~patterns used in
the $i$th clause; necessarily, for each~$i$, the variables of the patterns in
$\vec{pat}_i$ are disjoint.  Let $decl$ be the declaration
\[
\begin{array}{c}
g(\vec{pat}_1) = e_1 \\
\ldots \\
g(\vec{pat}_n) = e_n.
\end{array}
\]
We write $\vec{v}$ for a tuple of $m$ values (supplied as actual parameters).

We extend $\matches$ and $\bind$ to tuples in the obvious way:
\[
\begin{align}
\matches \rho~(pat_1,\ldots,pat_m)~(v_1,\ldots, v_m)  = \\
\qquad  \matches \rho~pat_1~v_1 \land \ldots \land \matches \rho~pat_m~v_m, \\
\bind \rho~(pat_1,\ldots,pat_m)~(v_1,\ldots, v_m)  = \\
\qquad  \bind(\ldots(\bind (\bind \rho~pat_1~v_1)~pat_2~v_2)\ldots)~pat_m~v_m.
\end{align}
\]
It is clear that parts~\ref{item:matches} and~\ref{item:bind} imply
corresponding results for these generalised versions.  

Suppose $g$ in environment~$\rho$ evaluates to a function over type~$A$; then
$g$ in environment~$\frho$ evaluates to a function over~$f(A)$.
%
We proceed as follows.
%
\begin{calc}
& f(\bindDecl \rho ~ decl) \\
= & f(\set{g \mapsto
  \begin{align}
  \set{\vec{v} \mapsto  
    \begin{align}  
      \If \matches \rho~ \vec{pat}_1~\vec{v} 
      \Then \eval (\bind \rho~\vec{pat}_1~\vec{v})~ e_1 \\
      \Else \ldots \\
      \Else \If \matches \rho~ \vec{pat}_n~\vec{v} 
      \Then \eval (\bind \rho~\vec{pat}_n~\vec{v})~ e_n \\
      \Else \error \| 
    \end{align} \\
  \quad \vec{v} \in A} })
  \end{align} \\
= & \com{applying~$f$} \\
& \set{g \mapsto
  \begin{align}
  \set{f(\vec{v}) \mapsto 
    \begin{align}  
      \If \matches \rho~ \vec{pat}_1~\vec{v} 
      \Then f(\eval (\bind \rho~\vec{pat}_1~\vec{v})~ e_1) \\
      \Else \ldots \\
      \Else \If \matches \rho~ \vec{pat}_n~\vec{v} 
      \Then f(\eval (\bind \rho~\vec{pat}_n~\vec{v})~ e_n) \\
      \Else f(\error) \| 
    \end{align} \\
  \quad \vec{v} \in A} }
  \end{align} \\
= & \com{clauses~\ref{item:matches} and~\ref{item:eval}; $f(error) = error$} \\
& \set{g \mapsto
  \begin{align}
  \set{f(\vec{v}) \mapsto 
    \begin{align}  
      \If \matches (\frho)~ \vec{pat}_1~(f(\vec{v})) \\
      \Then \eval (f(\bind \rho~\vec{pat}_1~\vec{v})) ~ e_1 \\
      \Else \ldots \\
      \Else \If \matches (\frho)~ \vec{pat}_n~(f(\vec{v})) \\
      \Then \eval (f(\bind \rho~\vec{pat}_n~\vec{v} )) ~ e_n \\
      \Else \error \| 
    \end{align} \\
  \quad \vec{v} \in A} }
  \end{align} \\
= & \com{clause~\ref{item:bind}} \\
& \set{g \mapsto
  \begin{align}
  \set{f(\vec{v}) \mapsto 
    \begin{align}  
      \If \matches (\frho)~ \vec{pat}_1~(f(\vec{v})) \\
      \Then \eval (\bind (\frho)~\vec{pat}_1~(f(\vec{v}))) ~ e_1 \\
      \Else \ldots \\
      \Else \If \matches (\frho)~ \vec{pat}_n~ (f(\vec{v})) \\
      \Then \eval (\bind (\frho) ~\vec{pat}_n~ (f(\vec{v}))) ~ e_n \\
      \Else \error \| 
    \end{align} \\
  \quad \vec{v} \in A} }
  \end{align} \\
= & \com{letting $\vec{w} = f(\vec{v})$} \\
& \set{g \mapsto
  \begin{align}
  \set{\vec{w} \mapsto 
    \begin{align}  
      \If \matches (\frho)~ \vec{pat}_1~\vec{w} 
      \Then \eval  (\bind (\frho)~\vec{pat}_1~ \vec{w} ) ~ e_1 \\
      \Else \ldots \\
      \Else \If \matches (\frho)~ \vec{pat}_n~ \vec{w} 
      \Then \eval  (\bind (\frho) ~\vec{pat}_n~ \vec{w}) ~ e_n \\
      \Else \error \| 
    \end{align} \\
  \quad \vec{w} \in f(A)} }
  \end{align} \\
= & \bindDecl (\frho) ~ decl.
\end{calc}

The extension to a function with $k$ curried arguments ---where each argument
has several sub-arguments, presented as patterns--- is straightforward but
notationally messy. 

%%%%%%%%%%%%%%%%%%

\paragraph{Case of a channel declaration.}

Consider a declaration 
\[
\CSPMM{channel}~ c : e_1.\ldots.e_n. 
\]
Then
%
\begin{calc}
  & f(\bindDecl \rho~ (\CSPMM{channel}~ c : e_1.\ldots.e_n)) \\
  = & f(\set{c \mapsto 
         \channel\set{v_1.\ldots.v_n \mid 
            v_1 \in \eval \rho~e_1, \ldots, v_n \in\eval \rho~e_n}}) \\
  = & \set{c \mapsto \channel \set{
        \begin{align}
        f(v_1).\ldots.f(v_n) \mid \\
        \quad    v_1 \in \eval \rho~e_1, \ldots, v_n \in\eval \rho~e_n}}
        \end{align} \\
  = & \com{letting $w_i = f(v_i)$} \\
    & \set{c \mapsto \channel\set{
        \begin{align}
        w_1.\ldots.w_n \mid \\
        \quad w_1 \in f(\eval \rho~e_1), \ldots, 
           w_n \in f(\eval \rho~e_n)}}
        \end{align} \\
  = & \com{clause~\ref{item:eval}} \\
    & \set{c \mapsto \channel\set{
        \begin{align}
        w_1.\ldots.w_n \mid \\
        \quad w_1 \in \eval (\frho)~e_1, \ldots,
        w_n \in \eval (\frho)~e_n}}
        \end{align} \\
  = & \bindDecl (\frho)~ (\CSPMM{channel}~ c : e_1.\ldots.e_n).
\end{calc}%
%
Note, in particular, that $v_1.\ldots.v_n$ is a complete value if and only if
$w_1.\ldots.w_n = f(v_1.\ldots.v_n$ is a complete value, by
Lemma~\ref{lem:complete}.

%%%%%%%%%%

\paragraph{Case of a datatype declaration.}

The case of the declaration of the distinguished type~|T| was dealt with
in Lemma~\ref{lem:distinguished-type-declaration}.

Let $decl$ be the following declaration, not of the distinguished type~|T|:
\[
\CSPMM{datatype}~D = 
A_1.e_{1,1}.\ldots.e_{1,m_1} \mid \ldots \mid A_n.e_{n,1}.\ldots.e_{n,m_n} \;.
\]
Then we calculate as follows, writing $S_i$ for the set of values associated
with constructor $A_i$:
%
\begin{calc}
&  f(\bindDecl \rho~decl) \\
= & 
  \begin{align}
  \Let S_i = \set{v_{i,1}.\ldots.v_{i,m_i} \mid 
         v_{i,j} \in \eval \rho ~ e_{i,j},\, j  = 1,\ldots, m_i},\\
  \qquad\qquad\qquad \mbox{for $i = 1,\ldots, n$}      \In    \\
  f(
    \begin{align}
    \set{A_i \mapsto \dtcons S_i \mid i \in \set{1,\ldots,n}} \union\null \\
    \set{D \mapsto 
      \set{A_i.v_i \mid i \in \set{1,\ldots,n},  v_i \in S_i } } )
    \end{align}
  \end{align} \\
= & 
  \begin{align}
  \Let S_i = \set{v_{i,1}.\ldots.v_{i,m_i} \mid 
         v_{i,j} \in \eval \rho ~ e_{i,j},\, j  = 1,\ldots, m_i},\\
  \qquad\qquad\qquad \mbox{for $i = 1,\ldots, n$}      \In    \\
  \begin{align}
    \set{f(A_i) \mapsto \dtcons f(S_i) \mid i \in \set{1,\ldots,n}} 
    \union\null \\
    \set{D \mapsto 
      \set{f(A_i).f(v_i) \mid i \in \set{1,\ldots,n},  v_i \in S_i } } )
    \end{align}
  \end{align} 
\\
= & \com{$f(A_i) = A_i$ since this is not the distinguished type; 
      letting $w_i = f(v_i)$} 
\\
& \begin{align}
  \Let S_i = \set{v_{i,1}.\ldots.v_{i,m_i} \mid 
         v_{i,j} \in \eval \rho ~ e_{i,j},\, j  = 1,\ldots, m_i},\\
  \qquad\qquad\qquad \mbox{for $i = 1,\ldots, n$}      \In    \\
  \begin{align}
    \set{A_i \mapsto \dtcons f(S_i) \mid i \in \set{1,\ldots,n}} \union \null \\
    \set{D \mapsto 
      \set{A_i.w_i \mid i \in \set{1,\ldots,n},  w_i \in f(S_i) } } )
    \end{align}
  \end{align} \\
= & \com{letting $S_i' = f(S_i)$} 
\\
& \begin{align}
  \Let S_i' = \set{f(v_{i,1}).\ldots.f(v_{i,m_i}) \mid 
         v_{i,j} \in \eval \rho ~ e_{i,j},\, j  = 1,\ldots, m_i},\\
  \qquad\qquad\qquad \mbox{for $i = 1,\ldots, n$}      \In    \\
  \begin{align}
    \set{A_i \mapsto \dtcons S_i' \mid i \in \set{1,\ldots,n}} \union \null \\
    \set{D \mapsto 
      \set{A_i.w_i \mid i \in \set{1,\ldots,n},  w_i \in S_i' } } )
    \end{align}
  \end{align} \\
= & \com{letting $w_{i,j} = f(v_{i,j})$} 
\\
& \begin{align}
  \Let S_i' = \set{w_{i,1}.\ldots.w_{i,m_i} \mid 
         w_{i,j} \in f(\eval \rho ~ e_{i,j}),\, j  = 1,\ldots, m_i},\\
  \qquad\qquad\qquad \mbox{for $i = 1,\ldots, n$}      \In    \\
  \begin{align}
    \set{A_i \mapsto \dtcons S_i' \mid i \in \set{1,\ldots,n}} \union \null \\
    \set{D \mapsto 
      \set{A_i.w_i \mid i \in \set{1,\ldots,n},  w_i \in S_i' } } )
    \end{align}
  \end{align} \\
= & \com{clause \ref{item:eval}} 
\\
& \begin{align}
  \Let S_i' = \set{w_{i,1}.\ldots.w_{i,m_i} \mid 
         w_{i,j} \in \eval (\frho) ~ e_{i,j},\, j  = 1,\ldots, m_i},\\
  \qquad\qquad\qquad \mbox{for $i = 1,\ldots, n$}      \In    \\
  \begin{align}
    \set{A_i \mapsto \dtcons S_i' \mid i \in \set{1,\ldots,n}} \union \null \\
    \set{D \mapsto 
      \set{A_i.w_i \mid i \in \set{1,\ldots,n},  w_i \in S_i' } } )
    \end{align}
  \end{align} \\
= & \bindDecl (\frho) ~ decl
\end{calc}
%
Note, in particular, that $A_i.v_i$ is a complete value iff $A_i.w_i$ is a
complete value, by Lemma~\ref{lem:complete}.

%%%%%%%%%%%%%%%%%%%%%%%%%%%%%%%%%%%%%%%%%%%%%%%%%%%%%%%

\subsubsection{Defined functions respect $f$}

Recall that we need to show that if a declaration, based on~$\rho$, binds a
name~$h$ to a function~$g$, then $g$ respects~$f$.

Consider the case of a point-free declaration of the form $h = e$, for example
where $e$ is a lambda abstraction.  But clause~\ref{item:eval} of the inductive
hypothesis shows that $g = \eval \rho~e$ respects~$f$, as required.

%% Suppose the environment~$\rho$ binds a name~$h$ to a function~$g$.  We require
%% that $g$ respects~$f$.  We prove this as an extension of the proof in
%% Section~\ref{sec:bindDecls}.  

For the case of a function defined by pattern matching, the proof is a
straightforward extension of the proof for lambda abstractions in
Section~\ref{sssec:function-decl}.

%% Now consider the case of a function defined by pattern matching:
%% \[
%% \begin{align}
%% h(pat_1)  =  e_1 \\
%% h(pat_2)  =  e_2 \\
%% \ldots \\
%% h(pat_k) = e_k
%% \end{align}
%% \]
%% where $k \ge 1$.  Then, letting $A$ be the argument type of~$h$, the
%% environment binds~$h$ to the function
%% %
%% \[
%% \set{(v,
%%   \begin{align}
%%   \If \matches \rho~ pat_1~v \Then \eval (\bind \rho~pat_1~v)~e \\
%%   \Else \If \matches \rho~ pat_2~v \Then \eval (\bind \rho~pat_2~v)~e \\
%%   \ldots \\
%%   \Else \If \matches \rho~ pat_k~v \Then \eval (\bind \rho~pat_k~v)~e 
%%   \Else error ) \| \\
%%   \qquad v \in A }.
%%   \end{align}
%% \]
%% The proof that the defined function respects~$f$ is very similar to the proof
%% for lambda extractions in Section~\ref{sssec:function-decl}.
%% %
%% The proof extends to curried functions in a straightforward (but notationally
%% messy) way.


%%%%%%%%%%%%%%%%%%%%%%%%%%%%%%%%%%%%%%%%%%%%%%%%%%%%%%%%%%%%

\subsubsection{Sets of declarations}

Now consider a set of declarations $decls$, which necessarily bind disjoint
variables.  Recall that
\[
\begin{align}
\bindDecls \rho~decls = \bigsqcup_{n \in \nat} F^n(\rho^\bottom) \\
\mbox{where } F(\rho_1)  =  
  \rho_1 \oplus \Union \set{\bindDecl \rho_1 ~ decl \| decl \in decls},
\end{align}
\]
where $\rho^\bottom$ extends $\rho$ by mapping all names bound by $decls$
to~$\bottom$. 

%% Recall that the proof of Proposition~\ref{prop:initial-env-invariant} showed
%% that each $F^n(\rho^\bottom)$ is $T_1$-invariant, under the assumption that
%% $\rho$ is. 
%% $T_1$-invariant

We have, for  environment~$\rho_1$:
%
\begin{calc}
& f(F (\rho_1)) \\
= & f(\rho_1) \oplus \Union 
    \set{f(\bindDecl \rho_1 ~decl) \mid decl \in decls} \\
= & \com{result for $\bindDecl$} \\
  & f(\rho_1) \oplus \Union 
    \set{\bindDecl (f(\rho_1)) ~decl \mid decl \in decls} \\
= & F (f(\rho_1)).
\end{calc}
%
A simple induction then shows that $f(F^n(\rho^\bottom)) =
F^n(f(\rho^\bottom))$.  Hence (using continuity of~$f$):
%
\begin{calc}
& f(\bindDecls \rho~ decls) \\
 = & f(\bigsqcup_{n \in \nat}  F^n(\rho^\bottom) ) \\
 = & \bigsqcup_{n \in \nat}  f(F^n(\rho^\bottom) ) \\
 = & \bigsqcup_{n \in \nat}  F^n(f(\rho^\bottom) ) \\
 = & \bindDecl (f(\rho))~ decls.
\end{calc}
