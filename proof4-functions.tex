\subsubsection{Functions}
\label{sssec:function-decl}

We show that a function application~$g(e)$ satisfies clause~\ref{item:eval}.
Note that we can assume inductively that $g$ evaluates to a function that
respects~$f$. 
%
\begin{calc}
& f(\eval \rho~(g(e))) \\
= & f((\eval \rho~g)~ (\eval \rho~e)) \\
= & \com{$\eval \rho~g$ respects $f$; Lemma \ref{lem:function-respects}} \\
  & (f(\eval \rho~g))~ (f(\eval \rho~e)) \\
= & \com{inductive hypothesis} \\
  & (\eval (f(\rho))~g) ~(\eval (f(\rho))~e) \\
= & \eval (f(\rho)) (g(e)).
\end{calc}

It remains to show that for any expression that defines a function: (1)~the
resulting function respects~$f$; and (2)~the expression satisfies
clause~\ref{item:eval}.  The following sections discharge these requirements.

%%%%%%%%%%

\paragraph{Lambda abstractions.}

Consider the expression $\lambda pat \spot e$.  Let $g = \eval \rho (\lambda
pat \spot e)$.  Suppose this function takes arguments of type~$A$.  Then
%
\begin{eqnarray*}
g & = & 
  \set{(v,\; 
  \If \matches \rho~pat~v \Then \eval(\bind \rho~pat~v)~e \Else error) \|
    v \in A}.
\end{eqnarray*}
%
We show that $g$ respects~$f$.  Suppose $(v,w), (v',w') \in g$ and $f(v) =
f(v')$.  Then
\begin{calc}
& f(w) \\
= & \If \matches \rho~pat~v \Then  f(\eval(\bind \rho~pat~v)~e)
    \Else f(error) \\
= & \com{inductive hypothesis, clauses \ref{item:matches} and \ref{item:eval};
  $f(error) = error$} \\
& \If \matches (f(\rho))~pat~(f(v))
       \Then \eval(f(\bind \rho~pat~v))~e 
    \Else error \\
= & \com{inductive hypothesis, clause~\ref{item:bind}} \\
&  \If \matches (f(\rho))~pat~(f(v))
       \Then \eval(\bind (f(\rho))~pat~(f(v)))~e 
    \Else error \\
= & \com{$f(v) = f(v')$} \\
& \If \matches (f(\rho))~pat~(f(v'))
       \Then \eval(\bind (f(\rho))~pat~(f(v')))~e 
    \Else error \\
= & \com{as above} \\
& f(w'),
\end{calc}
%
as required. 

We prove the lambda extraction satisfies clause~\ref{item:eval} as follows.
%
\begin{calc}
& f(\eval \rho~(\lambda pat \spot e)) \\
= & \com{as above} \\
& \begin{align}
  \set{(f(v),\; 
    \begin{align}
    \If \matches (f(\rho))~pat~(f(v))
       \Then \eval(\bind (f(\rho))~pat~(f(v)))~e \\
    \Else error) \| 
    \end{align} \\
    \qquad v \in A} 
    \end{align} \\
= & \com{letting $v' = f(v)$} \\& 
  \begin{align}
  \set{(v',\; 
    \begin{align}
    \If \matches (f(\rho))~pat~v'
       \Then \eval(\bind (f(\rho))~pat~v')~e 
    \Else error) \| 
    \end{align} \\
    \qquad v' \in f(A)} 
    \end{align} \\
= & \eval~(f(\rho))~(\lambda pat \spot e).
\end{calc}

%%%%%%%%%%

\paragraph{Bound function names.}

Consider a name~$h$ declared as a function in the script, and suppose $\rho$
maps~$h$ to a function~$g$.  By clause~\ref{item:bindDecls} of the inductive
hypothesis, $g$ respects~$f$.  Further
\[
f(\eval \rho~h) = f(g) = \eval (f(\rho))~h.
\]

%%%%%%%%%%

\paragraph{Built-in functions.}

Consider a built-in function~|g|.  Let $f: T_1 \fun T_2$, and suppose the
semantics of~|g| over types built from~$T_1$ is a function~$g_1$, and  the
semantics of~|g| over types built from~$T_2$ is a function~$g_2$.  We seek to
show $g_2 = f(g_1)$.  Note that we need to distinguish between $g_1$ and $g_2$
to ensure correct typing.  

%
%% We consider which built-in functions~$g$ respect~$f$, i.e.~if $f(v) = f(v')$
%% then $f(g(v)) = f(g(v'))$.  The following lemma will be useful.
%
\begin{lemma}
\label{lem:built-in-functions}
Suppose $\dom g_2 = f(\dom g_1)$ and $f \circ g_1 = g_2 \circ f$.  Then
$g_1$ respects~$f$, and $f(g_1) = g_2$.
\end{lemma}
%
\begin{proof}
We first show $g_1$ respects~$f$.  Suppose $f(v) = f(v')$.  Then $f(g_1(v)) =
g_2(f(v)) = g_2(f(v')) = f(g_1(v'))$, as required.

We prove $f(g_1) = g_2$ as follows.
%
\begin{calc}
& f(g_1) \\ 
= & \set{ (f(v), f(g_1(v))) \| v \in \dom g_1 } \\ 
= & \set{ (f(v), g_2(f(v))) \| v \in \dom g_1 } \\
= & \com{letting $w = f(v)$} \\ 
& \set{ (w, g_2(w)) \| w \in f(\dom g_1) }  \\
= & \set{ (w, g_2(w)) \| w \in \dom g_2 } \\
= & g_2.
\end{calc}
\end{proof}

%%%%%

Under the condition $f(g_1) = g_2$, we can proof condition~\ref{item:eval}
holds for the built-in function~|g|:
\[
f(\eval \rho ~ \M{g}) = f(g_1) = g_2 = \eval (f(\rho)) ~ \M{g}.
\]

It remains to consider which built-in functions satisfy the premises of
Lemma~\ref{lem:built-in-functions}.  The condition $\dom g_2 = f(\dom g_1)$
holds for all built-in functions.  For non-polymorphic functions, $g_1$
and~$g_2$ are the same functions, operating over values independent of~$t$,
and $f$ is the identity on the resulting values.  For polymorphic functions,
the functions are defined over \emph{all} types with the relevant type
constraint (e.g.~$Set~t$ for types whose members can be included in sets), and
corresponding types built from~$T_1$ and~$T_2$ satisfy the same type
constraints.

If a function |g| (or infix binary operator) can be applied only to values
that do not use $t$, and so necessarily returns a result that it independent
of~$t$, then its semantics is independent of the underlying types, i.e.~$g_1 =
g_2$.  Further, composing it with~$f$ has no effect.  Hence $f \circ g_1 = g_1
= g_2 = g_2 \circ f$.  Examples include operators over integers
(e.g.~$+$,~$<=$) or booleans~(e.g.~|and|,~|!|).  The same is true for
instances of polymorphic functions that are independent of~$t$.  Examples
include equality or inequality tests over values independent of $t$; or
application of the set operators to sets of values that are independent
of~$t$.

It is straightforward to show that each of the following polymorphic built-in
functions satisfies the condition $f \circ g_1 = g_2 \circ f$.
%
\begin{itemize}
\item the set functions |empty|, |union|, |Union|, |Set| and |Seq|;

\item the sequence functions \CSPM{^} (concatenation), \CSPM{#} (length),
  |concat|, |head|, |length|, |null|, |set| and~|tail|;

\item the map functions |emptyMap|, |mapDelete|, |mapFromList|, |mapLookup|,
  |mapMember|, |mapUpdate|, |mapUpdateMultiple| and~|Map| (recall that we
  assume that the domain type of each map is independent of~$t$);

\item the relation functions \CSPM{relational_image},
  \CSPM{relational_inverse_image}, and \CSPM{transpose}.
\end{itemize}

However, the condition $f \circ g_1 = g_2 \circ f$ does not hold for all
polymorphic built-in functions when applied to arguments of type~$t$.  For
example, it does not hold for the set-cardinality functions (denoted $\#_1$,
$\#_2$) corresponding to the built-in function |card|; if $f(A) = f(B) = C$
then:
\[
f(\#_1\set{A,B}) = f(2) = 2 \ne 1 = \#_2\set{C} = \#_2(f(\set{A,B})).
\]
For a binary operator with denotations~$\oplus_1$ and $\oplus_2$ (written as
infix operators), the required condition can be stated as $f(v \oplus_1 v') =
f(v) \oplus_2 f(v')$.  The condition does not hold for the set-intersection
functions.
\[
f(\set{A} \inter_1 \set{B}) = f(\set{}) = \set{} \ne \set{C} 
  = \set{C} \inter_2 \set{C} = f(\set{A}) \inter_2 f(\set{B}).
\]
All the built-in functions for which the condition does not hold were
disallowed by condition~\ref{item:built-in-functions} of
Definition~\ref{defn:data-independent}.

%% But not
%% \begin{itemize}
%% \item the set functions |card|, |diff|, |inter|, |Inter|, |member| and |seq|;
%%   or the subset relations |<=|, |<|, |>=| and |>|;

%% \item the sequence function |elem|;

%% \item the map function |mapToList|;

%% \item the function~|show|.
%% \end{itemize}

%%%%%%%%%%%%%%%%%%%%%%%%%%%%%%%%%%%%%%%%%%%%%%%%%%%%%%%

\subparagraph{{\cspmInlineStyle extensions} and {\cspmInlineStyle productions}.}

We now consider the |extensions| function, which, given a partially completed
channel or datatype value~$a$, gives all values~$v$ such that $a.v$ is a
complete event or datatype value (here $a$ and $v$ might be dotted values).
Given an environment~$\rho$, we write $\extendables(\rho)$ for all the
extendable values in~$\rho$, i.e.~partially constructed channels or datatype
values.\footnote{Here we use ``$\prefix$'' to denote the prefix relation over
  dotted tuples.}
%
\begin{eqnarray*}
\extendables(\rho) & = & 
  \set{ c.v \|
    \exists S \spot 
      \begin{align}
      (\rho(c) = \channel S \lor \rho(c) = \dtcons(S)) \land \\
      \exists w \in S \spot v \prefix w }.
      \end{align}
\end{eqnarray*}
%
Here and in the following, $c$ denoted a non-dotted term, so either a channel
name or a datatype constructor.  The semantics of the |extensions| function in
environment~$\rho$ is then the function:
\[
\set{ c.v \mapsto \set{ w \| v.w \in \rho(c) } \| 
     c.v \in \extendables(\rho) }.
\]

%We will need the following two lemmas.
% 
%% \begin{lemma}
%% \label{lem:extendables-f-rho}
%% $\extendables (f(\rho)) = f(\extendables(\rho))$.
%% \end{lemma}

%% \begin{proof}
%% We prove a set-inclusion in each direction.
%% %
%% \begin{description}
%% \item[($\supseteq$)]  Suppose $c.v \in \extendables(\rho)$, and let $S$ be such
%%   that $\rho(c)$ is either $\channel S$ or $\dtcons S$.  Then for some $w \in
%%   S$,\, $v \prefix w$.  But then $f(\rho)(f(c))$ is either $\channel (f(S))$
%%   or $\dtcons (f(S))$; and $f(w) \in f(S)$ and $f(v) \prefix f(w)$.  Hence
%%   $f(c.v) = f(c).f(v) \in \extendables(f(\rho))$.

%% \item[($\subseteq$)] Suppose $f(c.v) = f(c).f(v) \in \extendables(f(\rho))$.
%%   Then $\rho(c)$ must be either $\channel S$ or $\dtcons S$, for some~$S$, and
%%   $f(\rho)(f(c))$ is either $\channel (f(S))$ or $\dtcons (f(S))$,
%%   respectively.  Further, there is some $f(w) \in f(S)$ such that $f(v)
%%   \prefix f(w)$.  Let $u$ be such that $f(w) = f(v).f(u) = f(v.u)$.  But then
%%   Lemma~\ref{lem:T-invariant-inclusion} implies that $v.u \in S$, and so $c.v
%%   \in \extendables(\rho)$.
%% \end{description}
%% \end{proof}

%% \begin{lemma}
%% \label{lem:extensibles-vs-f}
%% $c.v \in \extendables(\rho) \iff f(c).f(v) \in f(\extendables(\rho))$.
%% \end{lemma}

%% \begin{proof}
%% The left-to-right implication is trivial.  For the right-to-left implication,
%% suppose $f(c).f(v) \in f(\extendables(\rho))$.
%% %
%% \begin{itemize}
%% \item If $c$ is a member of the distinguished datatype, then $f(v) = \epsilon$
%%   (since $\rho(c) = \dtcons\set\epsilon$).  Hence $v = \epsilon$, and $c.v = c
%%   \in \extendables(\rho)$.

%% \item Otherwise $f(c) = c$.  Let $S$ be such that $\rho(c)$ is either
%%   $\channel S$ or $\dtcons S$.  Then there is some $v'$ such that $c.v' \in
%%   \extendables(\rho)$ and $f(v) = f(v')$.  Hence there is some $w' \in S$ such
%%   that $v' \prefix w'$.  Let $u$ be such that $w' = v'.u$, and let $w = v.u$.
%%   Then $f(w) = f(v).f(u) = f(v').f(u) = f(v'.u) = f(w')$.  But $S$ is
%%   $T$-invariant, so $w \in S$, by Lemma~\ref{lem:T-invariant-vs-f}.  And $v
%%   \prefix w$, so $c.v \in \extendables(\rho)$, as required.
%% \end{itemize}
%% \end{proof}

%%%%%%%%%%

We will need the following lemma.
%
\begin{lemma}\label{lem:extendables-f-rho}
$c.v \in \extendables(\rho) \iff f(c).f(v) \in \extendables(f(\rho))$.
\end{lemma}

\begin{proof}
For the left-to-right direction, suppose $c.v \in \extendables(\rho)$.  Let
$S$ be such that $\rho(c)$ is either $\channel S$ or $\dtcons S$.  Then for
some $w \in S$,\, $v \prefix w$.  But then $f(\rho)(f(c))$ is either $\channel
(f(S))$ or $\dtcons (f(S))$; and $f(w) \in f(S)$ and $f(v) \prefix
f(w)$. Hence $f(c.v) = f(c).f(v) \in \extendables(f(\rho))$.

For the right-to-left direction, suppose $f(c).f(v) \in
\extendables(f(\rho)))$.
\begin{itemize}
\item If $c$ is a member of the distinguished datatype, $f(\rho)(f(c)) =
  \dtcons\set\epsilon$, so $f(v) = \epsilon$, and hence $v = \epsilon$.  But
  then $\rho(c) = \dtcons\set\epsilon$, so $c.v = c.\epsilon \in
  \extendables(\rho)$.

\item Otherwise, $f(c) = c$.  Let $S$ be such that $\rho(c)$ is either
  $\channel S$ or $\dtcons S$.  Hence $f(\rho)(c)$ is either $\channel (f(S))$
  or $\dtcons (f(S))$, and there is some $w' \in f(S)$ such that $f(v) \prefix
  w'$.  Let $u$ be such that $w' = f(v).f(u) = f(v.u)$, and let $w = v.u$.
  But $S$ is $T$-invariant, so $w \in S$, by Lemma~\ref{lem:T-invariant-vs-f}.
  And $v \prefix w$, so $c.v \in \extendables(\rho)$, as required.
\end{itemize}
\end{proof}


%% Then $f(w') \in f(S)$, and so ..........

%%  Then $\extendables(\rho)$
%% contains a value $d.v'$ such that $f(c) = f(d)$ and $f(v) = f(v')$.  
%% \begin{itemize}
%% \item If $d$ is a member of the distinguished datatype, then so is~$c$, and
%%   vice versa.  And then $v = v' = \epsilon$, and $c.v = c \in
%%   \extendables(\rho)$.

%% \item Otherwise, $c = d = f(c) = f(d)$.  Let $S$ be such that $\rho(c)$ is
%%   either $\channel S$ or $\dtcons S$.  Then, since $d.v' = c.v' \in
%%   \extendables(\rho)$,\, $S$ contains a value $w'$ such that $v' \prefix w'$.
%%   Let $w' = v'.u'$.  
%% \end{itemize}
 
We can now prove clause~\ref{item:eval} for the |extensions| function.
%
\begin{calc}
& f(\eval \rho~\M{extensions}) \\
= & f(\set{ c.v \mapsto \set{ w \| v.w \in \rho(c) } \| 
     c.v \in \extendables(\rho) }) \\
= & \com{applying~$f$; Lemma \ref{lem:extendables-f-rho}} \\
& \set{ f(c).f(v) \mapsto \set{ f(w) \| v.w \in \rho(c) } \| 
      f(c).f(v) \in \extendables(f(\rho))) } \\
= & \com{letting $v' = f(v)$ and $w' = f(w')$} \\
 & \set{ f(c).v' \mapsto \set{ w' \| v'.w' \in f(\rho(c)) } \| 
      f(c).v' \in f(\extendables(\rho)) } \\
= & \com{letting $c' = f(c)$; $f(\rho(c)) = f(\rho)(f(c)) = f(\rho)(c')$} \\
%%  & \set{ f(c).v' \mapsto \set{ w' \| v'.w' \in f(\rho)(f(c)) } \| 
%%       f(c).v' \in \extendables(f(\rho)) } \\
%% = & \com{letting $c' = f(c)$} \\
 & \set{ c'.v' \mapsto \set{ w' \| v'.w' \in f(\rho)(c') } \| 
      c'.v' \in \extendables(f(\rho)) } \\
= & \eval (f(\rho)) ~\M{extensions}.
\end{calc}


We now show that $\eval \rho~\M{extensions}$ respects~$f$.  Suppose $f(c.v) =
f(c'.v')$, so $f(c) = f(c')$ and $f(v) = f(v')$.  Then, using Lemma
\ref{lem:extendables-f-rho}:
\[
\begin{array}{rcl}
c.v \in \extendables(\rho) &  \iff & f(c.v)  \in \extendables(f(\rho)) \\
 & \iff & f(c'.v') \in \extendables(f(\rho)) \\
 & \iff  & c'.v' \in \extendables(\rho).
\end{array}
\]
Suppose both sides of the above implications hold, so $c.v$ and $c'.v'$ are
both in the domain of $\eval \rho~\M{extensions}$.  We  show 
\begin{eqnarray*}
\eval \rho~\M{extensions} (c.v) & = & \eval \rho~\M{extensions} (c'.v'),
\end{eqnarray*}
which implies that $\eval \rho~\M{extensions}$ respects~$f$.  If $c$ and~$c'$
are members of the distinguished type~$t$, then $v = v' = \epsilon$, and both
sides of the above equation equal $\set\epsilon$.  Otherwise $c = c'$.  Let
$S$ be such that $\rho(c) = \rho(c')$ is either $\channel S$ or $\dtcons S$.
Then
%
\begin{calc}
& \eval \rho~\M{extensions} (c.v) \\
= &  \set{ w \| v.w \in S} \\
= & \com{%
  \begin{tabular}{@{}l@{}}
  $f(v.w) = f(v).f(w) = f(v').f(w) = f(v'.w)$, and $S$ is $T$-invariant, \\
  so  $v.w \in S \iff v'.w  \in S$, by Lemma \ref{lem:T-invariant-vs-f}
  \end{tabular}} \\
 &  \set{ w \| v'.w \in S} \\
= & \eval \rho~\M{extensions} (c.v),
\end{calc}
%
as required. 

%% Now, $f(v.w) = f(v'.w)$, and $S$ is $T$-invariant, so $v.w \in S \iff v'.w \in
%% S$, by Lemma \ref{lem:T-invariant-vs-f}.  Hence
%% \[
%% \begin{array}{rcl}
%% \eval \rho~\M{extensions} (c.v) & = & \set{ w \| v.w \in S} \\
%% & = & \set{ w \| v'.w \in S} \\
%% & = & \eval \rho~\M{extensions} (c'.v'),
%% \end{array}
%% \]
%% as required.

%% \framebox{*** But also that it respects $f$}


%%%%%

The proofs for |productions| follow easily, noting that $\eval \rho~
\M{productions}$ and $\eval \rho~\M{extensions}$ have the same domains, and,
for $a$ in that domain,
\begin{eqnarray*}
\eval \rho~ \M{productions}(a) & = & 
  \set{ a.w \| w \in \eval \rho ~ \M{extensions}(a)}.
\end{eqnarray*}
%

%%%%%%%%%%%%%%%%%%%%%%%%%%%%%%%%%%%%%%%%%%%%%%%%%%%%%%%

%% \subparagraph{...}

%% \framebox{???}  Compression functions; relation functions.  |extensions|
%% |productions|.  


%% \begin{lemma}
%% If $\dom g_2 = f(\dom g_1)$ and  $f \circ g_1 = g_2 \circ f$, then
%% $f(g_1) = g_2$.
%% \end{lemma}

%% \begin{proof}
%% \begin{calc}
%% & f(g_1) \\ 
%% = & \set{ (f(v), f(g_1(v))) \| v \in \dom g_1 } \\ 
%% = & \set{ (f(v), g_2(f(v))) \| v \in \dom g_1 } \\
%% = & \com{letting $w = f(v)$} \\ 
%% & \set{ (w, g_2(w)) \| w \in f(\dom g_1) }  \\
%% = & \set{ (w, g_2(w)) \| w \in \dom g_2 } \\
%% = & g_2.
%% \end{calc}
%% \end{proof}
%
%% The converse does not hold.  Indeed, if $f : T_1 \fun T_2$ (suitably lifted),
%% and $g : T_1 \fun T_1$, then $g \circ f$ might not even be well-typed.  The
%% above lemma requires $g$ to be suitably polymorphic --- as is the case with
%% most built-in functions.

%% Further, if a built in function with name~|g| has denotation~$g$, then
%% \[
%% f(\eval \rho \M{g}) 
%% = f(g) 
%% = \set{ (f(v), f(g(v))) \| v \in \dom g } 
%% = \set{ (f(v), g(f(v))) \| v \in \dom g } 
%% = \set{ (w, g(w)) \| w \in f(\dom g) } 
%% = ?? 
%% f(\eval (f(\rho)) \M{g}) 
%% \]

%% Note that if $g$ returns results that are independent of~$t$ (on
%% which values $f$ is the identity), the consequent is equivalent $g(v) =
%% g(v')$.  
%% For a binary operator with denotation~$\oplus$, the required
%% condition can be stated as $f(v \oplus v') = f(v) \oplus f(v')$. 
%% : if $f(v) = f(v')$ and $f(w) = f(w')$ then
%% $f(v \oplus w) = f(v' \oplus w')$.


%% Consider a built-in function |g| with semantics~$g$ such that $f(g(x)) =
%% g(f(x))$ for all~$x$.  Then  clause~\ref{item:eval} holds for an application
%% of~|g|: 
%% %
%% \begin{calc}
%% & f(\eval \rho ~ (\M{g}(e))) \\
%% = &  f(g(\eval \rho e)) \\
%% = & g(f(\eval \rho e)) \\
%% = & g(\eval (f(\rho)) e) \\
%% = & \eval (f(\rho)) ~(\M{g}(e)).
%% \end{calc}
%% %
%% For a binary operator with semantics $\oplus$, we can prove a corresponding
%% result provided $f(x \oplus x') = f(x) \oplus f(x')$, for all $x, x'$. 

