\subsubsection{Functions}

We now consider functions defined and/or applied in the script.
The representation of a function~$g$ is a set of pairs $(v,w)$ such that $g$
maps~$v$ to~$w$, with the normal property of being a function, i.e.
\[
\forall v, w, w' \spot (v,w) \in g \land (v,w') \in g \implies w=w'.
\]
We will use standard notation for function application, writing $g(v)$ for the
unique $w$ such that $(v,w) \in g$.
%
Note that each \CSPm\ function is considered total, although it might map
certain arguments to one of the special values~$\bottom$ or $error$. 

Applying $f$ to a function~$g$ corresponds to applying $f$ to each element of
each pair, i.e.~$f(g) = \set{(f(v), f(w)) \| (v,w) \in g}$.  We show below
that $f(g)$ is a function.  Here and below, if $f: T_1 \fun T_2$, then $g$ is
a function over a type~$A$ built from~$T_1$, and $f(g)$ is a function over the
corresponding type $f(A)$ built from~$T_2$.

The following definition will be useful.
%
\begin{definition}
We say that $g$ \emph{respects~$f$} if
\[
\forall v, w, v', w' \spot 
  (v,w) \in g \land (v',w') \in g \land f(v) = f(v') \implies f(w) = f(w'). 
\]
Equivalently:
\[
\forall v,  v' \spot f(v) = f(v') \implies f(g(v)) = f(g(v')). 
\]
\end{definition}
%
We show that if $g$ respects~$f$ then $f(g)$ is a function.  Later we show
that all functions that can be defined using \CSPm\ respect~$f$. 
%
\begin{lemma}
\label{lem:function-respects}
Suppose $g$ respects $f$.  Then $f(g)$ is a function, and $f(g(x)) =
(f(g))(f(x))$.
\end{lemma}
%
\begin{proof}
We first show $f(g)$ is a function.  Suppose $(v,w) \in f(g)$ and $(v,w') \in
f(g)$.  Then for some $v_0, v_0', w_0, w_0'$ we have
\[
\begin{align}
(v_0,w_0) \in g \land f(v_0) = v \land f(w_0) = w \land \\
(v_0',w_0') \in g \land f(v_0') = v \land f(w_0') = w'.
\end{align}
\]
But $g$ respects~$f$, and $f(v_0) = f(v_0')$, so $f(w_0) = f(w_0')$, and hence
$w = w'$.

Now suppose $g(v) = w$, i.e.~$(v,w) \in g$.  Then $(f(v),f(w)) \in f(g)$ so
$(f(g))(f(v)) = f(w) = f(g(v))$, as required.
\end{proof}

%%%%%

We can now show that a function application~$g(e)$ satisfies 
clause~\ref{item:eval}, assuming $g$ evaluates to a function that
respects~$f$:
%
\begin{calc}
& f(\eval \rho~(g(e))) \\
= & f((\eval \rho~g)~ (\eval \rho~e)) \\
= & \com{Lemma \ref{lem:function-respects}} \\
  & (f(\eval \rho~g))~ (f(\eval \rho~e)) \\
= & \com{inductive hypothesis} \\
  & (\eval (f\after\rho)~g) ~(\eval (f\after\rho)~e) \\
= & \eval (f\after\rho) (g(e)).
\end{calc}

It remains to show that for any expression that defines a function: (1)~the
resulting function respects~$f$; and (2)~the expression satisfies
clause~\ref{item:eval}.  The following sections discharge these requirements.

%%%%%%%%%%

\paragraph{Lambda abstractions.}

Consider the expression $\lambda pat \spot e$.  Let $g = \eval \rho (\lambda
pat \spot e)$.  Suppose this function takes arguments of type~$A$.  Then
%
\begin{eqnarray*}
g & = & 
  \set{(v,\; 
  \If \matches \rho~pat~v \Then \eval(\bind \rho~pat~v)~e \Else error) \|
    v \in A}.
\end{eqnarray*}
%
We show that $g$ respects~$f$.  Suppose $(v,w), (v',w') \in g$ and $f(v) =
f(v')$.  Then
\begin{calc}
& f(w) \\
= & \If \matches \rho~pat~v \Then  f(\eval(\bind \rho~pat~v)~e)
    \Else f(error) \\
= & \com{inductive hypothesis, clauses \ref{item:matches} and \ref{item:eval};
  $f(error) = error$} \\
& \If \matches (f\after\rho)~pat~(f(v))
       \Then \eval(f\after(\bind \rho~pat~v))~e 
    \Else error \\
= & \com{inductive hypothesis, clause~\ref{item:bind}} \\
&  \If \matches (f\after\rho)~pat~(f(v))
       \Then \eval(\bind (f\after\rho)~pat~(f(v)))~e 
    \Else error \\
= & \com{$f(v) = f(v')$} \\
& \If \matches (f\after\rho)~pat~(f(v'))
       \Then \eval(\bind (f\after\rho)~pat~(f(v')))~e 
    \Else error \\
= & \com{as above} \\
& f(w'),
\end{calc}
%
as required. 

We prove the lambda extraction satisfies clause~\ref{item:eval} as follows.
%
\begin{calc}
& f(\eval \rho~(\lambda pat \spot e)) \\
= & \com{as above} \\
& \begin{align}
  \set{(f(v),\; 
    \begin{align}
    \If \matches (f\after\rho)~pat~(f(v))
       \Then \eval(\bind (f\after\rho)~pat~(f(v)))~e \\
    \Else error) \| 
    \end{align} \\
    \qquad v \in A} 
    \end{align} \\
= & \com{letting $v' = f(v)$} \\& 
  \begin{align}
  \set{(v',\; 
    \begin{align}
    \If \matches (f\after\rho)~pat~v'
       \Then \eval(\bind (f\after\rho)~pat~v')~e 
    \Else error) \| 
    \end{align} \\
    \qquad v' \in f(A)} 
    \end{align} \\
= & \eval~(f\after\rho)~(\lambda pat \spot e).
\end{calc}

%%%%%%%%%%

\paragraph{Bound function names.}

Suppose the environment~$\rho$ binds a name~$h$ to a function~$g$.  We require
that $g$ respects~$f$.  We prove this as an extension of the proof in
Section~\ref{sec:bindDecls}.  Suppose $h$ is defined by pattern matching:
\[
\begin{align}
h(pat_1)  =  e_1 \\
h(pat_2)  =  e_2 \\
\ldots \\
h(pat_k) = e_k
\end{align}
\]
where $k \ge 1$.  Then, letting $A$ be the argument type of~$h$, the
environment binds~$h$ to the function
%
\[
\set{(v,
  \begin{align}
  \If \matches \rho~ pat_1~v \Then \eval (\bind \rho~pat_1~v)~e \\
  \Else \If \matches \rho~ pat_2~v \Then \eval (\bind \rho~pat_2~v)~e \\
  \ldots \\
  \Else \If \matches \rho~ pat_k~v \Then \eval (\bind \rho~pat_k~v)~e 
  \Else error ) \| \\
  \qquad v \in A }.
  \end{align}
\]
The proofs that this function respects~$f$, and that clause{item:eval} is
satisfied, are very similar to the proofs for lambda extractions, above.

The proofs extend to curried functions in a straightforward (but notationally
messy) way.

%%%%%%%%%%

\paragraph{Built-in functions}

Consider a built-in function~|g|.  Let $f: T_1 \fun T_2$, and suppose the
semantics of~|g| over types built from~$T_1$ is a function~$g_1$, and  the
semantics of~|g| over types built from~$T_2$ is a function~$g_2$.  We seek to
show $g_2 = f(g_1)$.  Note that we need to distinguish between $g_1$ and $g_2$
to ensure correct typing.  

%
%% We consider which built-in functions~$g$ respect~$f$, i.e.~if $f(v) = f(v')$
%% then $f(g(v)) = f(g(v'))$.  The following lemma will be useful.
%
\begin{lemma}
\label{lem:built-in-functions}
Suppose $\dom g_2 = f(\dom g_1)$ and $f \circ g_1 = g_2 \circ f$.  Then
$g_1$ respects~$f$, and $f(g_1) = g_2$.
\end{lemma}
%
\begin{proof}
We first show $g_1$ respects~$f$.  Suppose $f(v) = f(v')$.  Then $f(g_1(v)) =
g_2(f(v)) = g_2(f(v')) = f(g_1(v'))$, as required.

We prove $f(g_1) = g_2$ as follows.
%
\begin{calc}
& f(g_1) \\ 
= & \set{ (f(v), f(g_1(v))) \| v \in \dom g_1 } \\ 
= & \set{ (f(v), g_2(f(v))) \| v \in \dom g_1 } \\
= & \com{letting $w = f(v)$} \\ 
& \set{ (w, g_2(w)) \| w \in f(\dom g_1) }  \\
= & \set{ (w, g_2(w)) \| w \in \dom g_2 } \\
= & g_2.
\end{calc}
\end{proof}

%%%%%

Under the condition $f(g_1) = g_2$, we can proof condition~\ref{item:eval}
holds for the built-in function~|g|:
\[
f(\eval \rho ~ \M{g}) = f(g_1) = g_2 = \eval (f\after\rho) ~ \M{g}.
\]

It remains to consider which built-in functions satisfy the premises of
Lemma~\ref{lem:built-in-functions}.  The condition $\dom g_2 = f(\dom g_1)$
holds for all built-in functions.  For non-polymorphic functions, $g_1$
and~$g_2$ are the same functions, operating over values independent of~$t$,
and $f$ is the identity on the resulting values.  For polymorphic functions,
the functions are defined over \emph{all} types with the relevant type
constraint (e.g.~$Set~t$ for types whose members can be included in sets), and
corresponding types built from~$T_1$ and~$T_2$ satisfy the same type
constraints.

If a function |g| (or infix binary operator) can be applied only to values
that do not use $t$, and so necessarily returns a result that it independent
of~$t$, then its semantics is independent of the underlying types, i.e.~$g_1 =
g_2$.  Further, composing it with~$f$ has no effect.  Hence $f \circ g_1 = g_1
= g_2 = g_2 \circ f$.  Examples include operators over integers
(e.g.~$+$,~$<=$) or booleans~(e.g.~|and|,~|!|).  The same is true for
instances of polymorphic functions that are independent of~$t$.  Examples
include equality or inequality tests over values independent of $t$; or
application of the set operators to sets of values that are independent
of~$t$.

It is straightforward to show that each of the following polymorphic built-in
functions satisfies the condition $f \circ g_1 = g_2 \circ f$.
%
\begin{itemize}
\item the set functions |empty|, |union|, |Union|, |Set| and |Seq|;

\item the sequence functions |concat|,  |head|, |length|, |null|, |set|
and~|tail|; 

\item the map functions |emptyMap|, |mapDelete|, |mapFromList|, |mapLookup|,
  |mapMember|, |mapUpdate|, |mapUpdateMultiple| and~|Map| (recall that we
  assume that the domain type of each map is independent of~$t$);
\end{itemize}

However, the condition $f \circ g_1 = g_2 \circ f$ does not hold for all
polymorphic built-in functions when applied to arguments of type~$t$.  For
example, it does not hold for the set-cardinality functions (denoted $\#_1$,
$\#_2$) corresponding to the built-in function |card|; if $f(A) = f(B) = C$
then:
\[
f(\#_1\set{A,B}) = f(2) = 2 \ne 1 = \#_2\set{C} = \#_2(f(\set{A,B})).
\]
For a binary operator with denotations~$\oplus_1$ and $\oplus_2$ (written as
infix operators), the required condition can be stated as $f(v \oplus_1 v') =
f(v) \oplus_2 f(v')$.  The condition does not hold for the set-intersection
functions.
\[
f(\set{A} \inter_1 \set{B}) = f(\set{}) = \set{} \ne \set{C} 
  = \set{C} \inter_2 \set{C} = f(\set{A}) \inter_2 f(\set{B}).
\]
All the built-in functions for which the condition does not hold were
disallowed by condition~\ref{item:built-in-functions} of
Definition~\ref{defn:data-independent}.

%% But not
%% \begin{itemize}
%% \item the set functions |card|, |diff|, |inter|, |Inter|, |member| and |seq|;
%%   or the subset relations |<=|, |<|, |>=| and |>|;

%% \item the sequence function |elem|;

%% \item the map function |mapToList|;

%% \item the function~|show|.
%% \end{itemize}

\framebox{???} Compression functions; relation functions.  |extensions|
|productions|.


%% \begin{lemma}
%% If $\dom g_2 = f(\dom g_1)$ and  $f \circ g_1 = g_2 \circ f$, then
%% $f(g_1) = g_2$.
%% \end{lemma}

%% \begin{proof}
%% \begin{calc}
%% & f(g_1) \\ 
%% = & \set{ (f(v), f(g_1(v))) \| v \in \dom g_1 } \\ 
%% = & \set{ (f(v), g_2(f(v))) \| v \in \dom g_1 } \\
%% = & \com{letting $w = f(v)$} \\ 
%% & \set{ (w, g_2(w)) \| w \in f(\dom g_1) }  \\
%% = & \set{ (w, g_2(w)) \| w \in \dom g_2 } \\
%% = & g_2.
%% \end{calc}
%% \end{proof}
%
%% The converse does not hold.  Indeed, if $f : T_1 \fun T_2$ (suitably lifted),
%% and $g : T_1 \fun T_1$, then $g \circ f$ might not even be well-typed.  The
%% above lemma requires $g$ to be suitably polymorphic --- as is the case with
%% most built-in functions.

%% Further, if a built in function with name~|g| has denotation~$g$, then
%% \[
%% f(\eval \rho \M{g}) 
%% = f(g) 
%% = \set{ (f(v), f(g(v))) \| v \in \dom g } 
%% = \set{ (f(v), g(f(v))) \| v \in \dom g } 
%% = \set{ (w, g(w)) \| w \in f(\dom g) } 
%% = ?? 
%% f(\eval (f\after\rho) \M{g}) 
%% \]

%% Note that if $g$ returns results that are independent of~$t$ (on
%% which values $f$ is the identity), the consequent is equivalent $g(v) =
%% g(v')$.  
%% For a binary operator with denotation~$\oplus$, the required
%% condition can be stated as $f(v \oplus v') = f(v) \oplus f(v')$. 
%% : if $f(v) = f(v')$ and $f(w) = f(w')$ then
%% $f(v \oplus w) = f(v' \oplus w')$.


%% Consider a built-in function |g| with semantics~$g$ such that $f(g(x)) =
%% g(f(x))$ for all~$x$.  Then  clause~\ref{item:eval} holds for an application
%% of~|g|: 
%% %
%% \begin{calc}
%% & f(\eval \rho ~ (\M{g}(e))) \\
%% = &  f(g(\eval \rho e)) \\
%% = & g(f(\eval \rho e)) \\
%% = & g(\eval (f\after\rho) e) \\
%% = & \eval (f\after\rho) ~(\M{g}(e)).
%% \end{calc}
%% %
%% For a binary operator with semantics $\oplus$, we can prove a corresponding
%% result provided $f(x \oplus x') = f(x) \oplus f(x')$, for all $x, x'$. 

