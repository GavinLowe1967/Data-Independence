\section{Denotational semantics}

We now build on Proposition~\ref{prop:expressions} to prove the denotational
results claimed at the start of the paper.  

Recall that we write $P[T]$ for the process~$P$ interpreted in the script
where the distinguished type~|T| is given value~$T$.  We make this more
precise.  Recall that we write $\rho^T$ for the environment that results from
evaluating just the declaration of~|T| with value~$T$
(Definition~\ref{def:initial-env}).  Suppose $decls$ represents the top-level
declarations in the script.  Then the environment formed from these
declarations is $\bindDecls \rho^{T_1}~decls$.  So the value of~$P$ in the
script is
\begin{eqnarray*}
P[T] & \defs &  \eval (\bindDecls \rho^{T}~decls)~P.
\end{eqnarray*}
%
The following lemma relates the value of~$P$ in the two different scripts.
%
\begin{lemma}
Let $f: T_1 \fun T_2$ be surjective.  Then $P[T_1] \bisim_f P[T_2]$. 
\end{lemma}
%
\begin{proof}
We have:
\begin{calc}
& P[T_1] \\
= & \eval (\bindDecls \rho^{T_1}~decls)) ~P \\
\bisim_f & \com{clause~\ref{item:eval} of Proposition~\ref{prop:expressions}} \\
& \eval (f(\bindDecls \rho^{T_1}~decls)) ~P \\
= & \com{clause~\ref{item:bindDecls} of Proposition~\ref{prop:expressions}} \\
  & \eval (\bindDecls (f(\rho^{T_1}))~decls) ~P \\
= & \com{Lemma~\ref{lem:distinguished-type-declaration}} \\
= & \eval (\bindDecls \rho^{T_2}~decls)~P \\
= & P[T_2].
\end{calc}
%% & P[T_2] \\
%% = & \eval (\bindDecls \rho^{T_2}~decls)~P \\
%% = & \com{Lemma~\ref{lem:distinguished-type-declaration}} \\
%%   & \eval (\bindDecls (f(\rho^{T_1}))~decls) ~P \\
%% = & \com{clause~\ref{item:bindDecls} of Proposition~\ref{prop:expressions}} \\
%%   & \eval (f(\bindDecls \rho^{T_1}~decls)) ~P \\
%% = & \com{clause~\ref{item:eval}  of Proposition~\ref{prop:expressions}} \\
%%   & f( \eval (\bindDecls \rho^{T_1}~decls)) ~P \\
%% = & f(P[T_1]) .
%% \end{calc}
\end{proof}


Consider an ALTS with transition relation $\trans{}$, and let $s = \trace{a_1,
  \ldots, a_n}$ be a sequence of events from~$\Sigma \union \set{\tau,\tick}$.
We write $(P,\rho) \transd{s} (P',\rho')$ to indicate that there is a sequence
of transitions, labelled with $a_1, \ldots a_n$, leading from $(P,\rho)$
to~$(P',\rho')$, i.e.~there exist states~$(P_0,\rho_0) =
(P,\rho),\linebreak[1] (P_1,\rho_1),\linebreak[1] \ldots,\linebreak[1]
(P_n,\rho_n) = (P',\rho')$ such that $(P_i,\rho_i) \trans{a_{i+1}}
(P_{i+1},\rho_{i+1})$ for $i = 0,\ldots,n-1$.  Below we write
$\transd{}_{T_1}$ and $\transd{}_{T_2}$ for this relation corresponding to
ALTSs built using $T_1$ and~$T_2$, respectively.

The following lemma is proved by a straightforward induction, using the
definition of $f$-bisimulation.
%
\begin{lemma}
Suppose $L_1 \sim_f L_2$.  Then:
%
\begin{enumerate}
\item If $(P,\rho) \transd{s}_{T_1} (P',\rho')$ in~$L_1$, then $(P,f(\rho))
  \transd{f(s)}_{T_2} (P',f(\rho'))$ in~$L_2$;  

\item If $(P,f(\rho)) \transd{s'}_{T_2} (P',\rho')$ in~$L_2$, then $(P,\rho)
  \transd{s}_{T_1} (P',\rho'')$ for some $s$ such that $f(s) = s'$, and
  some~$\rho''$ such that $f(\rho'') = \rho'$.
\end{enumerate}
\end{lemma}


%%%%%%%%%%%%%%%%%%%%%%%%%%%%%%%%%%%%%%%%%%%%%%%%%%%%%%%

%% \subsection{Cut?}

%% It immediately follows that for each transition of $P[T_1]$ using event~$a$,
%% there is a corresponding transition of~$P[T_2]$ using event $f(a)$.  
%% %
%% \begin{lemma}
%% If
%% \[
%% (P,\rho) \trans{a} (Q,\rho_2)  \quad \mbox{in $P[T_1]$}\,,
%% \]
%% then
%% \[
%% (P,f(\rho)) \trans{f(a)} (Q,f(\rho_2))  \quad \mbox{in $P[T_2]$}.
%% \]
%% \end{lemma}

%% The proof of the converse is made more difficult by the fact that applying~$f$
%% to $\eval \rho~P$ can merge states.  A particular state $(Q,\rho_1)$ gets
%% mapped to $(Q,f(\rho_1))$, but so does any other state $(Q,\rho_2)$ such that
%% $f(\rho_2) = f(\rho_1)$.  The resulting state has transitions corresponding to
%% the union of the transitions from $(Q,\rho_1)$ and $(Q,\rho_2)$ (remapped
%% by~$f$).  We need to show that each transition of $(Q,f(\rho_1))$ does
%% correspond to a transition of~$(Q,\rho_1)$. 

%% %% \begin{lemma}
%% %% Suppose $(Q, \rho)$ is a state in $P[T_1]$, and suppose there is a transition
%% %% \[
%% %% (Q, f(\rho)) \trans{a} (Q',\rho_1)  \quad \mbox{in $P[T_2] = f(P[T_1])$.}
%% %% \]
%% %% Then there is a transition 
%% %% \[
%% %% (Q, \rho) \trans{a'} (Q', \rho_1') \quad \mbox{in $P[T_1]$}
%% %% \]
%% %% such that $f(a') = a$, and $f(\rho_1') = \rho_1$.
%% %% \end{lemma}

%% %% \begin{proof}
%% %% The sub-ALTS rooted at $(Q,\rho)$ and $(Q, f(\rho))$ are equal to $\eval \rho~Q$
%% %% and $\eval Q f(\rho)~Q = f(\eval \rho~Q)$, respectively.  

%% %% , by
%% %% Proposition~\ref{prop:expressions}.  

%% %% \end{proof}



%% %%%%%%%%%%%%%%%%%%%%%%%%%%%%%%%%%%%%%%%%%%%%%%%%%%%%%%%

%% \subsection{Ignore?? }



%% %
%% The following sequence of lemmas proves the converse.
%% %
%% \begin{lemma}
%% \label{lem:f-sim-expr}
%% Suppose $f(\rho) = f(\rho')$.  Then for every expression~$e$,\,
%% \[
%% f(\eval \rho~e) =f(\eval \rho'~e).
%% \]
%% \end{lemma}
%% %
%% \begin{proof}
%% We have
%% \(
%% f(\eval \rho~e) = \eval (f(\rho))~e = \eval (f(\rho'))~e = f(\eval \rho'~e)
%% \),
%% using Proposition~\ref{prop:expressions}.
%% \end{proof}

%% %%%%%%%%%%

%% \begin{lemma}
%% \label{lem:f-sim-proc}
%% Consider two ALTS states $(Q,\rho)$ and $(Q,\rho')$ such that $f(\rho) =
%% f(\rho')$.  Suppose there is a transition
%% \[
%% (Q,\rho) \trans{a} (Q',\rho_1),
%% \]
%% Then there is also a transition
%% \[
%% (Q,\rho') \trans{a'} (Q',\rho_1'),
%% \]
%% such that $f(a) = f(a')$ and $f(\rho_1) = f(\rho_1')$.
%% \end{lemma}
%% %
%% \begin{proof}
%% The ALTSs rooted at the two states are equal to $\eval \rho ~ Q$ and $\eval
%% \rho'~ Q$, respectively.  By Lemma~\ref{lem:f-sim-expr}, $f(\eval \rho~ Q) =
%% f(\eval \rho' ~Q)$.  Then \framebox{????}
%% \end{proof}


%% %%%%%%%%%%%%%%%%%%%%%%%%%%%%%%%%%%%%%%%%%%%%%%%%%%%%%%%

%% %% Consider, the case of a process expression~$P$.  Recall that $\eval \rho~P$ is
%% %% an ALTS with initial state~$(P,\rho)$; we write $ALTS_{P,\rho}$ for this
%% %% ALTS\@.  The following lemma follows immediately from
%% %% Lemma~\ref{lem:f-sim-expr}.
%% %% %
%% %% \begin{lemma}
%% %% \label{lem:f-sim-proc}
%% %% Suppose $f(\rho) = f(\rho')$.  Then $f(ALTS_{P,\rho}) = f(ALTS_{P,\rho'})$.
%% %% In particular, for every transition 
%% %% \[
%% %% (P,\rho) \trans{a} (Q,\rho_2)  \quad \mbox{in $ALTS_{P,\rho}$}\,,
%% %% \]
%% %% there is a corresponding transition 
%% %% \[
%% %% (P,\rho') \trans{a'} (Q,\rho_2')  \quad \mbox{in $ALTS_{P,\rho'}$}
%% %% \]
%% %% such that $f(a) = f(a')$, and $f(\rho_2) = f(\rho_2')$, and vice versa.
%% %% \end{lemma}


%% %%%%%%%%%%

%% \begin{lemma}
%% Suppose $(Q, \rho)$ is a state in $P[T_1]$, and suppose there is a transition
%% \[
%% (Q, f(\rho)) \trans{a} (Q',\rho_1)  \quad \mbox{in $P[T_2] = f(P[T_1])$.}
%% \]
%% Then there is a transition 
%% \[
%% (Q, \rho) \trans{a'} (Q', \rho_1') \quad \mbox{in $P[T_1]$}
%% \]
%% such that $f(a') = a$, and $f(\rho_1') = \rho_1$.
%% \end{lemma}

%% \framebox{XXX}

%% I don't think the above holds from what I've proved so far.  Consider a
%% process~$Q$ equivalent to $\If x = y \Then a \then STOP \Else STOP$, and let
%% $\rho = \set{x \mapsto A, y \mapsto B}$.  Then $(Q,\rho)$ does not have a
%% transition on~$a$.  But if $f(A) = f(B) = C$, then $(Q,f(\rho))$ does have a
%% transition on~$a$.  This doesn't contradict Proposition~\ref{prop:expressions}
%% as written, because $f$ merges $(Q,\rho)$ with $(Q,\rho')$ where $\rho' =
%% \set{x \mapsto A, y \mapsto B}$, which does have a transition on~$a$. 


%% \begin{proof}
%% The transition in $f(P[T_1])$ results from the application of~$f$ to a
%% transition in $P[T_1]$; that transition must be of the form
%% \[
%% (Q, \rho') \trans{a''} (Q', \rho'') % \quad \mbox{in $ALTS_{P,\rho}$}
%% \]
%% such that $f(\rho') = f(\rho)$,\, $f(a'') = a$, and $f(\rho'') = \rho_1$.
%% Then Lemma~\ref{lem:f-sim-proc} implies the existence of a transition
%% \[
%% (Q, \rho) \trans{a'} (Q', \rho_1') % \quad \mbox{in $ALTS_{P,\rho}$}
%% \]
%% such that $f(a') = f(a'') = a$, and $f(\rho_1') = f(\rho'') = \rho_1$, as
%% required. 
%% \end{proof}





%% %%%%%%%%%%%%%%%%%%%%%%%%%%%%%%%%%%%%%%%%%%%%%%%%%%%%%%%


%% %% Suppose  there is a transition 
%% %% \[
%% %% (Q_1,\rho_1) \trans{a} (Q_2,\rho_2) \quad \mbox{in $f(ALTS_{P,\rho})$.}
%% %% \]
%% %% Then there is a transition
%% %% \[
%% %% (Q_1,\rho_1') \trans{a'} (Q_2,\rho_2') \quad \mbox{in $ALTS_{P,\rho}$}
%% %% \]
%% %% such that $f(\rho_1') = \rho_1$,\, $f(a') = a$, and $f(\rho_2') = \rho_2$.
%% %% Suppose also $f(\rho_1'') = f(\rho_1') = \rho_1$.  Then
%% %% there is a transition
%% %% \[
%% %% (Q_1,\rho_1'') \trans{a''} (Q_2,\rho_2'') \quad \mbox{in $ALTS_{P,\rho}$}
%% %% \]
%% %% such that $f(a'') = f(a') = a$ and $f(\rho_2'') = f(\rho_2') = \rho_2$.

%% Not apply induction.
