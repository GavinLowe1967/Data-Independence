\section{Invariance of sets}

Recall from the introduction that we place a restriction, \emph{invariance}
(Definition~\ref{def:invariant}), on expressions that define sets, mainly
concerning the use of bound variables in set comprehensions.  In this section,
we present a corresponding semantic condition, \emph{$T$-invariance}, on the
set produced, and show that it is satisfied. 

Informally, we want to say that, for each instantiation~$T$ of~$t$, if a value
$A \in T$ can appear in a particular place in a channel or datatype value,
then replacing that instance of~$A$ by a different value~$B \in T$ also gives
a valid channel or datatype value.  For example, for the definitions in
Figure~\ref{fig:t-allowed}, if $A \in T$, then $\M{c2}.A.A$ is a valid value;
and so is $\M{c2}.A.B$ for $B \in T$.  The following definition captures this
formally.
%
\begin{definition}
Let $T$ be an instantiation of~$t$, let $\unit$ be a fresh value, and let
$\unitf : T \fun \set{\unit}$ be the function that maps each element of~$T$
to~$\unit$. Extend~$\unitf$ to other values in the normal way.  Then a set $S$
is \emph{$T$-invariant} if
\[
\forall w, w' \spot
  \unitf(w) = \unitf(w') \land w \in S  \implies w' \in S.
\]
\end{definition}

%%%%%

\begin{example}
Suppose $\M{c2}.A.A \in \eset{\M{c2}}$.  Then 
\[
\unitf(\M{c2}.A.A) = \M{c2}.\unit.\unit = \unitf(\M{c2}.A.B),
\]
so $\set{A,B}$-invariance of $\eset{\M{c2}}$ would require $\M{c2}.A.B \in
\eset{\M{c2}}$.  This holds for the declarations in
Figure~\ref{fig:t-allowed}.

However, recall the declaration (which is not invariant, and so forbidden by
our assumptions)
%
\begin{cspm}
channel two : union({Two.x.x.true | x <- t}, {Two.x.y.false | x <- t, y <- t})
\end{cspm}
%
The set $\eset{\M{two}}$ is not $\set{A,B}$-invariant: it includes
|two.Two.A.A.true|, but not |two.Two.A.B.true|; and
\[
\unitf(\M{two.Two.A.A.true}) = \M{two.Two}.u.u.\M{true} =
   \unitf(\M{two.Two.A.B.true}).
\]
\end{example}

%%%%%

The following definition captures that channel identifiers, datatype
constructors or identifiers representing sets are either mapped to
$T$-invariant values, or to improper values, $\bottom$ or~$error$; our main
interest is in the former case, but we need the latter for completeness.
%
\begin{definition}
An environment~$\rho$ is \emph{$T$-invariant} if
%
\begin{itemize}
\item For every channel identifier~$c$, either $\rho(c) \in \set{\bottom,
  error}$, or $\rho(c) = \channel S$ for a set~$S$ that is $T$-invariant;

\item For every datatype constructor~$C$, either $\rho(C) \in \set{\bottom,
  error}$, or $\rho(C) = \dtcons S$ for a set~$S$ that is $T$-invariant;

\item For every name~$x$ that represents a set, either $\rho(x) \in
  \set{\bottom, error}$ or $\rho(x)$ is $T$-invariant.
\end{itemize}
\end{definition}

%%%%%

Fix an instantiation $T$ of $t$.  We show that the syntactic property of
invariance implies the semantic property of $T$-invariance. 

\begin{lemma}
\label{lem:invariant}
Let $T$ be an instantiation of~$t$.  Let $\rho$ be a $T$-invariant environment
such that $\rho(t) = T$. 
%
Consider a particular expression~$s$ that defines a set and is invariant.
Then $\eval \rho~s$ is $T$-invariant.
\end{lemma}

\begin{proof}
We prove the result by induction over the structure of~$s$, and perform a case
analysis over the form of~$s$, following Definition~\ref{def:invariant}.  We
concentrate on the interesting cases.

Consider a set comprehension $s = \M\{ e ~ \M\| stmts \M\}$.  Suppose $s$
evaluates correctly to a set $S = \eval \rho~s$ (so $\rho$ binds identifiers
used in~$s$ to proper values, and errors do not occur).  Suppose $w \in S$ and
$\unitf(w) = \unitf(w')$.  Consider the identifiers $x_1,\ldots,x_k$ in~$e$
that depend on~$t$, and let $w_1,\ldots,w_k$ be the corresponding fields
in~$w$.  By the assumption of invariance, the identifiers $x_1,\ldots,x_k$
each appears only once in~$e$, and each is bound by a generator $x_i
\,\M{<-}\, s_i$ in $stmts$, where $s_i$ is itself invariant.  Let $S_i = \eval
\rho_i~s_i$, where $\rho_i$ is the environment~$\rho$ augmented corresponding
to values taken by earlier generators in the production of~$w$; so $w_i \in
S_i$.  By induction, $S_i$ is $T$-invariant.

Let $w_1',\ldots,w_k'$ be the corresponding fields in~$w'$.  We show that the
identifiers $x_1,\ldots,x_k$ could also be bound to $w_1',\ldots,w_k'$ by the
generators, with all other generators being bound as in the production of~$w$.
Consider a generator $x_i \,\M{<-}\, s_i$.  Let $\rho_i'$ be the
environment~$\rho$ augmented corresponding to values taken by earlier
generators.  But, by assumption, $s_i$ uses none of the earlier bound
variables~$x_1,\ldots,x_{i-1}$, and so $\eval \rho_i'~s_i = \eval \rho_i~s_i =
S_i$.  And $w_i' \in S_i$, since $S_i$ is $T$-invariant, so this generator can
indeed produce~$w_i'$.  Further, the identifiers $x_1,\ldots,x_k$ do not
appear in predicates in $stmts$, so each such predicate will still be true
with this binding.  Then evaluating $e$ produces~$w'$, so $w' \in S$, as
required.

Now consider a productions set comprehension $\M{\{\|} e ~ \M\| stmts \M{\|\}}$.
Necessarily, the expression~$e$ represents either a channel or a partial value
from a datatype.  This expression represents the same set as a set
comprehension where the ``missing'' fields are added with extra generators;
for example, the expression \CSPM{\{\|c2o\|\}}  defines the same set as
\CSPM{\{c2o.x.y \| x <- t, y <- Option\}}.  By the assumption that $\rho$ is
$T$-invariant, the generators range over sets that are themselves
$T$-invariant.  Hence we can apply the result for standard set comprehensions
to deduce the result for the productions set comprehension. 

Other cases from Definition~\ref{def:invariant} are straightforward.
\end{proof}

%%%%%%%%%%


Consider a data-independent script with top-level declarations $decls$.
Lemma~\ref{lem:invariant} had a premise that the environment is $T$-invariant.
We now use this to show that the environment~$\rho_1$ formed from $decls$ is
$T$-invariant.

Recall that $\bindDecl \rho~decl$ gives the environment formed by
updating~$\rho$ according to~$decl$.  The declarations might be mutually
recursive, so their semantics is defined as a fixed point.  One step of the
iteration for that fixed point is given by the function
%
\begin{eqnarray*}
F(\rho) & = & \rho \oplus \Union \set{\bindDecl \rho ~ decl \| decl \in decls}.
\end{eqnarray*}
%
Let $\rho^\bottom$ be the environment that maps all top-level identifiers
to~$\bottom$ (and maps no other identifier).  Then the environment~$\rho_1$
defined by $decls$ is the least fixed point of~$F$:
\begin{eqnarray*}
\rho_1 & = & \bigsqcup_{n \in \nat} F^n(\rho^\bottom).
\end{eqnarray*}

%%%%%

\begin{prop}
The initial environment~$\rho_1$ is $T$ invariant.
\end{prop}

\begin{proof}
Clearly $\rho^\bottom$ is $T$-invariant.  
%
Suppose $\rho$ is invariant, and consider a top-level declaration $x = s$;
so $s$ is invariant, by item~\ref{item:di-invariant} of
Definition~\ref{defn:data-independent}.  Then Lemma~\ref{lem:invariant} shows
that $\eval \rho~s$ is $T$-invariant, so 
\begin{eqnarray*}
\bindDecl \rho~(x = s) & = & \rho \oplus \set{x \mapsto \eval \rho~s}
\end{eqnarray*}
is $T$-invariant.  A similar argument applies to channel and datatype
declarations.  Hence if $\rho$ is $T$-invariant, then so is $F(\rho)$.
%
Then a straightforward induction shows that $F^n(\rho^\bottom)$ is
$T$-invariant, for all~$n$.  It follows that $\rho_1$ is $T$-invariant.
\end{proof}

%%%%%

We can immediately deduce that the parts of all subsequent environments
concerning channel identifiers and datatype constructors is $T$-invariant,
because channel identifiers and datatype constructors are declared only at the
top level.
%
\begin{corollary}
\label{cor:invariant}
Suppose the initial environment is well-formed, i.e.~no name gets mapped to
$\bottom$ or $error$.  Then for each subsequent environment~$\rho$:
\begin{itemize}
\item For every channel identifier~$c$,\, $\rho(c) = \channel S$ for a set~$S$
  that is $T$-invariant;

\item For every datatype constructor~$C$,\, $\rho(C) = \dtcons S$ for a
  set~$S$ that is $T$-invariant.
\end{itemize}
\end{corollary}
%
From now on, we will assume the result of the above corollary for all
environments subsequent to the initial environment. 

%%%%%%%%%%%%%%%%%%%%%%%%%%%%%%%%%%%%%%%%%%%%%%%%%%%%%%%%%%%%

\subsection{Channel types}

We'll need the following lemma concerning $T$-invariant sets.
%
\begin{lemma}
\label{lem:T-invariant-inclusion}
Suppose $S$ is $T$-invariant.  Then 
%
\begin{eqnarray*}
v \in S & \iff & f(v) \in f(S).
\end{eqnarray*}
\end{lemma}

%%%%%

\begin{proof}
The left-to-right implication is immediate.  For the right-to-left
implication, suppose $f(v) \in f(S)$.  Then there is some $v' \in S$ such that
$f(v) = f(v')$.  Note that $\unitf \circ f = \unitf$: each maps elements
of~$T$ to~$\unit$, and is lifted in the normal way to other values.  Hence
\[
\unitf(v) = \unitf(f(v)) = \unitf(f(v')) = \unitf(v').
\]
Hence $v \in S$, since $S$ is $T$-invariant.
\end{proof}

%%%%%

\begin{corollary}
\label{cor:channel-types}
Consider a script instantiating~$t$ with $T$.  Consider a channel
identifier~$c$ and environment~$\rho$, and suppose $\rho(c) = \channel S$.
Then for each value~$v$,
\begin{eqnarray*}
v \in S & \iff & f(v) \in f(S).
\end{eqnarray*}
%
Likewise for a datatype constructor~$C$ such that $\rho(C) = \dtcons S$. 
\end{corollary}

\begin{proof}
By Corollary~\ref{cor:invariant}, $S$ is $T$-invariant (in each case).  The
result then follows from Lemma~\ref{lem:T-invariant-inclusion}.
\end{proof}

%%%%%%%%%%

Values may be either \emph{incomplete} or \emph{complete}, depending on
whether additional fields may be added to them using the dot ($.$) operator.
For example, with the declarations of Figure~\ref{fig:t-allowed}, the values
|c2| and |c2.A| are incomplete, but |c2.A.B| is complete.  

\begin{lemma}
\label{lem:complete}
If $v$ is a complete value in environment~$\rho$ (subsequent to the initial
environment), then $f(v)$ is a complete value in environment $f\after\rho$.
\end{lemma}

\begin{proof}
Suppose $v$ is an event on channel~$c$, so $v = c.w$ for some $w \in S$ where
$\rho(c) = \channel S$.  Then $(f\after\rho)(c) = \channel f(S)$, and $S$ is
$T$-invariant, by Corollary~\ref{cor:invariant}.  Hence $f(w) \in f(S)$ by
Corollary~\ref{cor:channel-types}, and so $f(v) = c.f(w)$ is a complete
value. 
\end{proof}
