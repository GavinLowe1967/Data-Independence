\paragraph{Renaming}

The semantics of a renaming is a set of pairs of events $(e,e')$ indicating
that~$e$ is renamed to~$e'$.  
%% We start by considering renamings where each individual renaming \CSPM{e <-
%% e'} applies to complete events; we then extend to channels.
We start by showing the renaming itself satisfies clause~\ref{item:eval}.
Later, we use this to show that the application of a renaming to a process
also satisfies clause~\ref{item:eval}.
%
\begin{lemma}
\label{lem:eval-renaming}
Let $\rename{ren}$ be a renaming.  Then 
\begin{eqnarray*}
f(\eval \rho~\rename{ren}) & = & \eval (f(\rho))~ \rename{ren}.
\end{eqnarray*}
\end{lemma}
%
\begin{proof}
Let $\rename{ren} = \rename{ e_1 \lArrow e_1', \ldots, e_k \lArrow e_k' \|
stmts}$.  Suppose, for the moment, that each individual renaming $e_i \lArrow
e_i'$ is complete, i.e.~it renames an event as opposed to a channel.  Below,
``$id_1$'' represents the identity function over events based on~$T_1$, and
``$id_2$'' represents the identity function over events based on~$T_2$.
%
\begin{calc}
& f(\eval \rho~
   \rename{ e_1 \leftarrow e_1', \ldots, e_k \leftarrow e_k' \|  stmts}) \\
= & f(id_1 \oplus \set{ (\eval \rho'~e_i, \eval \rho'~e_i') \|
     \rho' \in \evalStmtsSet \rho~stmts, i \in \set{1,\ldots,k} }) \\
= & id_2 \oplus\null 
    \begin{align} \set{ (f(\eval \rho'~e_i), f(\eval \rho'~e_i')) \| \\
    \qquad \rho' \in \evalStmtsSet \rho~stmts, i \in \set{1,\ldots,k} }
     \end{align} \\
= & \com{inductive hypothesis} \\
  & id_2 \oplus\null  
     \begin{align}
     \set{ (\eval (f(\rho'))~e_i, \eval (f(\rho'))~e_i') \| \\
     \qquad \rho' \in \evalStmtsSet \rho~stmts, i \in \set{1,\ldots,k} }
     \end{align} \\
= & \com{letting $\rho'' = f(\rho')$} \\ 
  & \begin{align}
    id_2 \oplus \set{ (\eval \rho''~e_i, \eval \rho''~e_i') \| \\
    \qquad \rho'' \in f(\evalStmtsSet \rho~stmts), i \in \set{1,\ldots,k} }
    \end{align} \\
= & \com{inductive hypothesis, clause~\ref{item:evalStmts}} \\
  & id_2 \oplus\null
    \begin{align} \set{ (\eval \rho''~e_i, \eval \rho''~e_i') \| \\
    \qquad  \rho'' \in \evalStmtsSet (f(\rho))~stmts, i \in \set{1,\ldots,k} }
    \end{align} \\
= & \eval (f(\rho))
       \rename{ e_1 \leftarrow e_1', \ldots, e_k \leftarrow e_k' \|  stmts}.
\end{calc}

Finally, if an individual renaming $e_i \lArrow e_i'$ is not complete, we can
replace it by a complete renaming of the form $e_i.x_1.\ldots.x_k \lArrow
e_i'.x_1.\ldots.x_k$, with additional generators for $x_1,\ldots,x_k$ (cf.~the
proof of Lemma~\ref{lem:invariant}), and then apply the above result.
\end{proof}

%%%%%

Recall that each renaming satisfies condition~\ref{item:di-renaming} of
Definition~\ref{defn:data-independent}.  We will need the following lemma
(analogous to Lemma~\ref{lem:invariant}),
which shows that the value of a renaming is consistent with~$f$.
%
\begin{lemma}
\label{lem:renaming-invariant}
Consider a renaming $\rename{r} = \rename{ e_1 \lArrow e_1', \ldots, e_k
  \lArrow e_k' \| stmts}$, and let $R = \eval \rho~ \rename{r}$.  Then
%
\begin{eqnarray*}
%\label{eqn:renaming-invariant}
(v,v') \in R \land f(v) = f(w) & \implies &
  \exists w' \spot (w,w') \in R \land f(v') = f(w').
\end{eqnarray*}
\end{lemma}

%%%%%

\begin{proof}
Without loss of generality, we can assume that each individual renaming
$e_i \lArrow e_i'$ is complete, i.e.~it renames an event as opposed to a
channel (otherwise we can use an argument similar to that at the end of the
proof of Lemma~\ref{lem:eval-renaming}).

Suppose $(v,v') \in R$ and $f(v) = f(w)$.  Suppose $(v,v')$ is produced by the
individual renaming $e_j \lArrow e_j'$.  

Consider the fields $v_1,\ldots,v_k$ in~$v$ of types that depend upon~|T|.  By
condition~\ref{item:di-renaming} of Definition~\ref{defn:data-independent},
these correspond to distinct identifiers $x_1,\ldots,x_k$ in~$e_j$, each of
which is bound by a generator $x_i \lArrow s_i$ in $stmts$, where $s_i$ is
invariant.  Let $S_i = \eval \rho_i~s_i$, where $\rho_i$ is the
environment~$\rho$ augmented corresponding to values taken by earlier
generators in the production of~$(v,v')$; so $v_i \in S_i$.  By
Lemma~\ref{lem:invariant}, $S_i$~is invariant.

Let the corresponding fields in~$w$ be $w_1,\ldots,w_k$, so $f(v_i) = f(w_i)$
for each~$i$.  We show that the identifiers $x_1,\ldots,x_k$ could also be
bound to $w_1,\ldots,w_k$ by the generators, with all other generators being
bound as in the production of~$(v,v')$.  
Consider a generator $x_i \,\M{<-}\, s_i$.  Let $\rho_i'$ be the
environment~$\rho$ augmented corresponding to values taken by earlier
generators.  But, by assumption, $s_i$ uses none of the earlier bound
variables~$x_1,\ldots,x_{i-1}$, and so $\eval \rho_i'~s_i = \eval \rho_i~s_i =
S_i$.   And $w_i \in S_i$, since $S_i$ is $T_1$-invariant, so this generator can
indeed produce~$w_i$.  Further, the identifiers $x_1,\ldots,x_k$ do not
appear in predicates in $stmts$, so each such predicate will still be true
with this binding.  Then  the individual renaming $e_j \lArrow e_j'$
produces $(w,w')$ for some $w'$ such that $f(v') = f(w')$.
\end{proof}

%%%%%%%%%%%%%%%%%%%%%%%%%%%%%%%%%%%%%%%%%%%%%%%%%%%%%%%%%%%%
 

Consider a syntactic renaming $\rename{ren}$, which corresponds to a
semantic renaming $R$ (i.e.~a set of pairs).  When applied to a LTS~$P$, each
state $(Q,\rho)$ of~$P$ is replaced by $(Q\M{[[}ren\M{]]}, \rho)$, and the
transitions renamed according to~$R$.  We denote this renamed LTS
$P\rename{R}^{ren}$.
%
\begin{lemma}
\label{lem:renaming-f}
Let $P$ be an LTS, $\rename{ren}$ a syntactic renaming, and $R$ the
corresponding renaming relation.  Then
%
\begin{eqnarray*}
f(P\rename{R}^{ren}) & = & (f(P))\rename{f(R)}^{ren}.
\end{eqnarray*}
%
\end{lemma}

%%%%%

\begin{proof}
For each state $(Q,\rho)$ of~$P$, each side of the equation will have a
state $(Q\rename{ren}, f(\rho))$.

%% On each side, each state $Q$ of $P$ is replaced by $f(Q)$ (here each state $Q$
%% will be a pair $(QQ\M{[[}ren\M{]]}, \rho)$, where $QQ$ is a syntactic
%% process, \M{[[}ren\M{]]} is a syntactic renaming, and $\rho$ is an
%% environment). 

Suppose that $P$ contains a transition $(Q,\rho) \trans{e} (Q',\rho')$.  Then
$P\rename{R}^{ren}$ contains a transition $(Q\rename{ren},\rho) \trans{e'}
(Q'\rename{ren},\rho')$ for every $(e,e') \in R$; and so the left-hand side
contains a transition $(Q\rename{ren}, f(\rho)) \trans{f(e')}
(Q'\rename{ren}, {f(\rho')})$.  And $f(P)$ contains a transition $(Q,
f(\rho)) \trans{f(e)} (Q', f(\rho'))$; so the right-hand side contains
a transition $(Q\rename{ren}, f(\rho)) \trans{e''} (Q'\rename{ren},
{f(\rho')})$ for every $e''$ such that $(f(e), e'') \in f(R)$.  We show
that these transitions correspond.
%
\begin{itemize}
\item Consider such a transition of the left-hand side labelled with~$f(e')$.
Then $(e,e') \in R$ implies $(f(e),f(e')) \in f(R)$, so the right-hand side
  has a corresponding transition.

\item Consider such a transition of the right-hand side labelled with~$e''$.
  If $(f(e), e'') \in f(R)$ then there is some $(e_1,e_2) \in R$ such that
  $f(e) = f(e_1)$ and $e'' = f(e_2)$.  But then
  Lemma~\ref{lem:renaming-invariant} implies $(e,e_3) \in R$ for some $e_3$
  such that $f(e_3) = f(e_2) = e''$.  Hence the left-hand side has a
  corresponding transition labelled with $f(e_3) = e''$.
\end{itemize}
\end{proof}

%%%%%

Let $\rename{ren} = \rename{ e_1 \leftarrow e_1', \ldots, e_k \leftarrow
e_k' \| stmts}$ be a renaming.  We show that clause~\ref{item:eval} is
satisfied for the application of~$ren$ to a process~$P$.
%
\begin{calc}
& f(\eval \rho ~ (P~\rename{ren})) \\
= & f( (\eval \rho~P) \rename{ \eval \rho~ren }^{ren} ) \\
= & \com{Lemma \ref{lem:renaming-f}} \\
  &  (f(\eval \rho~P)) \rename{ f(\eval \rho~ren) }^{ren} \\
= & \com{inductive hypothesis; Lemma \ref{lem:eval-renaming}} \\
  &  (\eval (f(\rho))~P)) \rename{ \eval (f(\rho))~ren }^{ren} \\
= & \eval (f(\rho)) (P~\rename{ren}).
\end{calc}
