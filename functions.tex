\section{On functions}
\label{sec:functions}

We now consider functions defined and/or applied in the script.
The representation of a function~$g$ is a set of pairs $(v,w)$ such that $g$
maps~$v$ to~$w$, with the normal property of being a function, i.e.
\[
\forall v, w, w' \spot (v,w) \in g \land (v,w') \in g \implies w=w'.
\]
We will use standard notation for function application, writing $g(v)$ for the
unique $w$ such that $(v,w) \in g$.
%
Note that each \CSPm\ function is considered total, although it might map
certain arguments to one of the special values~$\bottom$ or $error$. 

Applying $f$ to a function~$g$ corresponds to applying $f$ to each element of
each pair, i.e.~$f(g) = \set{(f(v), f(w)) \| (v,w) \in g}$.  We need to show
that $f(g)$ is a function. 
%%  Here and below, if $f: T_1 \fun T_2$, then $g$ is a function over a
%% type~$A$ built from~$T_1$, and $f(g)$ is a function over the corresponding
%% type $f(A)$ built from~$T_2$.
%
The following definition will be useful.
%
\begin{definition}
We say that $g$ \emph{respects~$f$} if
\[
\forall v, w, v', w' \spot 
  (v,w) \in g \land (v',w') \in g \land f(v) = f(v') \implies f(w) = f(w'). 
\]
Equivalently:
\[
\forall v,  v' \spot f(v) = f(v') \implies f(g(v)) = f(g(v')). 
\]
\end{definition}

We show that if $g$ respects~$f$ then $f(g)$ is a function.  
%% Later we show that all functions that can be defined using
%% \CSPm\ respect~$f$.
%
\begin{lemma}
\label{lem:function-respects}
Suppose $g$ respects $f$.  Then $f(g)$ is a function, and $f(g(x)) =
(f(g))(f(x))$.
\end{lemma}
%
\begin{proof}
We first show $f(g)$ is a function.  Suppose $(v,w) \in f(g)$ and $(v,w') \in
f(g)$.  Then for some $v_0, v_0'$ we have
\[
\begin{align}
f(v_0) = v \land f(g(v_0)) = w \land 
f(v_0') = v \land f(g(v_0')) = w'.
\end{align}
\]
But $g$ respects~$f$, and $f(v_0) = f(v_0')$, so $w = f(g(v_0)) = f(g(v_0')) =
w'$.

Now suppose $g(v) = w$, i.e.~$(v,w) \in g$.  Then $(f(v),f(w)) \in f(g)$ so
$(f(g))(f(v)) = f(w) = f(g(v))$, as required.
\end{proof}

We will show (as part of Proposition~\ref{prop:expressions}) that each
function~$g$ defined in the script respects~$f$; hence each $f(g)$ is a
function.
