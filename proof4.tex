\subsection{Expressions}

We show that for every expression~$e$ in the script,
\begin{eqnarray*}
f(\eval \rho ~ e) & = \eval (f(\rho)) e .
\end{eqnarray*}

\framebox{\ldots}

%%%%%%%%%%%%%%%%%%%%%%%%%%%%%%%%%%%%%%%%%%%%%%%%%%%%%%%

\subsubsection{Atomic values}

\paragraph{Names.}

Consider a name~$x$ bound by the environments, other than a datatype
constructor or channel name.  Then
\[
f(\eval \rho~x) = f(\rho~x) = \eval (f(\rho))~x.
\]

Not consider the name~$d$ of a datatype constructor (so both $\rho$ and
$f(\rho)$ map $c$ to a $\dtcons$ value).  Necessarily, $d$ is not a member
of~$t$, given our assumption about values from~$t$ not appearing in the
script.  Note that $d$ evaluates to itself, regardless of the environment.
Hence
\[
f(\eval \rho~d) = f(d) = d = \eval (f(\rho))~d.
\]
Note that the middle step would fail to hold for a member of~$t$: this
justifies our assumption about such values not appearing in the script. 

The case of the name of a channel is identical to the previous case. 

Names of types *** I don't think this is necessary as a separate case -- dealt
with in bindDecls.


%%%%%

\paragraph{Basic values.}

For a basic value~$v$ (an integers, boolean or character):
\[
f(\eval \rho~v) = f(v) = v = \eval(f(\rho))~v,
\]
since $v$ is independent of~$t$. 

%%%%%%%%%%

\paragraph{\ldots}


|Events|


\subsubsection{Composite values}

Case $e_1 . e_2$.
%
\begin{calc}
& f(\eval \rho ~(e_1.e_2)) \\
 = & f(\eval \rho~e_1. \eval \rho~e_2) \\
 = & f(\eval \rho~e_1) . f(\eval \rho~e_2) \\
 = & \com{inductive hypothesis} \\
  & \eval (f(\rho))~e_1 . \eval (f(\rho))~e_2 \\
 = & \eval (f(\rho)) (e_1.e_2).
\end{calc}

The case for tuples is very similar.

%%%%%%%%%%%%%%%%%%%%%%%%%%%%%%%%%%%%%%%%%%%%%%%%%%%%%%%

\subsubsection{Structural operators}

The case for local declarations is as follows.
%
\begin{calc}
& f(\eval \rho~(\M{let}~decls~\M{within}~e)) \\
= & f(\Let \rho' = \bindDecls \rho~decls \In \eval \rho'~e) \\
= & \Let \rho' = \bindDecls \rho~decls \In f(\eval \rho'~e) \\
= & \com{inductive hypothesis} \\
 &  \Let \rho' = \bindDecls \rho~decls \In \eval (f(\rho'))~e \\
= & \com{letting $\rho'' = f(\rho')$} \\
& \Let \rho'' = f (\bindDecls \rho~decls \In \eval \rho''~e) \\
= & \com{inductive hypothesis, clause~\ref{item:bindDecls}} \\
& \Let \rho'' = \bindDecls (f(\rho))~decls \In \eval \rho''~e \\
= & \eval (f(\rho))~(\M{let}~decls~\M{within}~e).
\end{calc}

The case for if statements is as follows.
%
\begin{calc}
& f(\eval \rho~(\M{if}~b~\M{then}~e_1~\M{else}~e_2)) \\
= & f(\If \eval \rho~b \Then \eval \rho~e_1 \Else \eval \rho~e_2) \\
= & \com{boolean values invariant under~$f$} \\
 & \If f(\eval \rho~b) \Then f(\eval \rho~e_1) \Else f(\eval \rho~e_2) \\
= & \com{inductive hypothesis} \\
 & \If \eval~(f(\rho))~b \Then \eval (f(\rho))~e_1) 
    \Else \eval(f(\rho))~e_2 \\
= & \eval (f(\rho)) (\M{if}~b~\M{then}~e_1~\M{else}~e_2).
\end{calc}


%%%%%%%%%%%%%%%%%%%%%%%%%%%%%%%%%%%%%%%%%%%%%%%%%%%%%%%%%%%%

\subsubsection{Functions}

We now consider functions defined and/or applied in the script.
The representation of a function~$g$ is a set of pairs $(v,w)$ such that $g$
maps~$v$ to~$w$, with the normal property of being a function, i.e.
\[
\forall v, w, w' \spot (v,w) \in g \land (v,w') \in g \implies w=w'.
\]
We will use standard notation for function application, writing $g(v)$ for the
unique $w$ such that $(v,w) \in g$.
%
Note that each \CSPm\ function is considered total, although it might map
certain arguments to one of the special values~$\bottom$ or $error$. 

Applying $f$ to a function~$g$ corresponds to applying $f$ to each element of
each pair, i.e.~$f(g) = \set{(f(v), f(w)) \| (v,w) \in g}$.  We show below
that $f(g)$ is a function.  Here and below, if $f: T_1 \fun T_2$, then $g$ is
a function over a type~$A$ built from~$T_1$, and $f(g)$ is a function over the
corresponding type $f(A)$ built from~$T_2$.

The following definition will be useful.
%
\begin{definition}
We say that $g$ \emph{respects~$f$} if
\[
\forall v, w, v', w' \spot 
  (v,w) \in g \land (v',w') \in g \land f(v) = f(v') \implies f(w) = f(w'). 
\]
Equivalently:
\[
\forall v,  v' \spot f(v) = f(v') \implies f(g(v)) = f(g(v')). 
\]
\end{definition}
%
We show that if $g$ respects~$f$ then $f(g)$ is a function.  Later we show
that all functions that can be defined using \CSPm\ respect~$f$. 
%
\begin{lemma}
\label{lem:function-respects}
Suppose $g$ respects $f$.  Then $f(g)$ is a function, and $f(g(x)) =
(f(g))(f(x))$.
\end{lemma}
%
\begin{proof}
We first show $f(g)$ is a function.  Suppose $(v,w) \in f(g)$ and $(v,w') \in
f(g)$.  Then for some $v_0, v_0', w_0, w_0'$ we have
\[
\begin{align}
(v_0,w_0) \in g \land f(v_0) = v \land f(w_0) = w \land \\
(v_0',w_0') \in g \land f(v_0') = v \land f(w_0') = w'.
\end{align}
\]
But $g$ respects~$f$, and $f(v_0) = f(v_0')$, so $f(w_0) = f(w_0')$, and hence
$w = w'$.

Now suppose $g(v) = w$, i.e.~$(v,w) \in g$.  Then $(f(v),f(w)) \in f(g)$ so
$(f(g))(f(v)) = f(w) = f(g(v))$, as required.
\end{proof}

%%%%%

We can now show that a function application~$g(e)$ satisfies 
clause~\ref{item:eval}, assuming $g$ evaluates to a function that
respects~$f$:
%
\begin{calc}
& f(\eval \rho~(g(e))) \\
= & f((\eval \rho~g)~ (\eval \rho~e)) \\
= & \com{Lemma \ref{lem:function-respects}} \\
  & (f(\eval \rho~g))~ (f(\eval \rho~e)) \\
= & \com{inductive hypothesis} \\
  & (\eval (f\after\rho)~g) ~(\eval (f\after\rho)~e) \\
= & \eval (f\after\rho) (g(e)).
\end{calc}

It remains to show that for any expression that defines a function: (1)~the
resulting function respects~$f$; and (2)~the expression satisfies
clause~\ref{item:eval}.  The following sections discharge these requirements.

%%%%%%%%%%

\paragraph{Lambda abstractions.}

Consider the expression $\lambda pat \spot e$.  Let $g = \eval \rho (\lambda
pat \spot e)$.  Suppose this function takes arguments of type~$A$.  Then
%
\begin{eqnarray*}
g & = & 
  \set{(v,\; 
  \If \matches \rho~pat~v \Then \eval(\bind \rho~pat~v)~e \Else error) \|
    v \in A}.
\end{eqnarray*}
%
We show that $g$ respects~$f$.  Suppose $(v,w), (v',w') \in g$ and $f(v) =
f(v')$.  Then
\begin{calc}
& f(w) \\
= & \If \matches \rho~pat~v \Then  f(\eval(\bind \rho~pat~v)~e)
    \Else f(error) \\
= & \com{inductive hypothesis, clauses \ref{item:matches} and \ref{item:eval};
  $f(error) = error$} \\
& \If \matches (f\after\rho)~pat~(f(v))
       \Then \eval(f\after(\bind \rho~pat~v))~e 
    \Else error \\
= & \com{inductive hypothesis, clause~\ref{item:bind}} \\
&  \If \matches (f\after\rho)~pat~(f(v))
       \Then \eval(\bind (f\after\rho)~pat~(f(v)))~e 
    \Else error \\
= & \com{$f(v) = f(v')$} \\
& \If \matches (f\after\rho)~pat~(f(v'))
       \Then \eval(\bind (f\after\rho)~pat~(f(v')))~e 
    \Else error \\
= & \com{as above} \\
& f(w'),
\end{calc}
%
as required. 

We prove the lambda extraction satisfies clause~\ref{item:eval} as follows.
%
\begin{calc}
& f(\eval \rho~(\lambda pat \spot e)) \\
= & \com{as above} \\
& \begin{align}
  \set{(f(v),\; 
    \begin{align}
    \If \matches (f\after\rho)~pat~(f(v))
       \Then \eval(\bind (f\after\rho)~pat~(f(v)))~e \\
    \Else error) \| 
    \end{align} \\
    \qquad v \in A} 
    \end{align} \\
= & \com{letting $v' = f(v)$} \\& 
  \begin{align}
  \set{(v',\; 
    \begin{align}
    \If \matches (f\after\rho)~pat~v'
       \Then \eval(\bind (f\after\rho)~pat~v')~e 
    \Else error) \| 
    \end{align} \\
    \qquad v' \in f(A)} 
    \end{align} \\
= & \eval~(f\after\rho)~(\lambda pat \spot e).
\end{calc}

%%%%%%%%%%

\paragraph{Bound function names.}

Suppose the environment~$\rho$ binds a name~$h$ to a function~$g$.  We require
that $g$ respects~$f$.  We prove this as an extension of the proof in
Section~\ref{sec:bindDecls}.  Suppose $h$ is defined by pattern matching:
\[
\begin{align}
h(pat_1)  =  e_1 \\
h(pat_2)  =  e_2 \\
\ldots \\
h(pat_k) = e_k
\end{align}
\]
where $k \ge 1$.  Then, letting $A$ be the argument type of~$h$, the
environment binds~$h$ to the function
%
\[
\set{(v,
  \begin{align}
  \If \matches \rho~ pat_1~v \Then \eval (\bind \rho~pat_1~v)~e \\
  \Else \If \matches \rho~ pat_2~v \Then \eval (\bind \rho~pat_2~v)~e \\
  \ldots \\
  \Else \If \matches \rho~ pat_k~v \Then \eval (\bind \rho~pat_k~v)~e 
  \Else error ) \| \\
  \qquad v \in A }.
  \end{align}
\]
The proofs that this function respects~$f$, and that clause{item:eval} is
satisfied, are very similar to the proofs for lambda extractions, above.

The proofs extend to curried functions in a straightforward (but notationally
messy) way.

%%%%%%%%%%

\paragraph{Built-in functions}

Consider a built-in function~|g|.  Let $f: T_1 \fun T_2$, and suppose the
semantics of~|g| over types built from~$T_1$ is a function~$g_1$, and  the
semantics of~|g| over types built from~$T_2$ is a function~$g_2$.  We seek to
show $g_2 = f(g_1)$.  Note that we need to distinguish between $g_1$ and $g_2$
to ensure correct typing.  

%
%% We consider which built-in functions~$g$ respect~$f$, i.e.~if $f(v) = f(v')$
%% then $f(g(v)) = f(g(v'))$.  The following lemma will be useful.
%
\begin{lemma}
\label{lem:built-in-functions}
Suppose $\dom g_2 = f(\dom g_1)$ and $f \circ g_1 = g_2 \circ f$.  Then
$g_1$ respects~$f$, and $f(g_1) = g_2$.
\end{lemma}
%
\begin{proof}
We first show $g_1$ respects~$f$.  Suppose $f(v) = f(v')$.  Then $f(g_1(v)) =
g_2(f(v)) = g_2(f(v')) = f(g_1(v'))$, as required.

We prove $f(g_1) = g_2$ as follows.
%
\begin{calc}
& f(g_1) \\ 
= & \set{ (f(v), f(g_1(v))) \| v \in \dom g_1 } \\ 
= & \set{ (f(v), g_2(f(v))) \| v \in \dom g_1 } \\
= & \com{letting $w = f(v)$} \\ 
& \set{ (w, g_2(w)) \| w \in f(\dom g_1) }  \\
= & \set{ (w, g_2(w)) \| w \in \dom g_2 } \\
= & g_2.
\end{calc}
\end{proof}

%%%%%

Under the condition $f(g_1) = g_2$, we can proof condition~\ref{item:eval}
holds for the built-in function~|g|:
\[
f(\eval \rho ~ \M{g}) = f(g_1) = g_2 = \eval (f\after\rho) ~ \M{g}.
\]

It remains to consider which built-in functions satisfy the premises of
Lemma~\ref{lem:built-in-functions}.  The condition $\dom g_2 = f(\dom g_1)$
holds for all built-in functions.  For non-polymorphic functions, $g_1$
and~$g_2$ are the same functions, operating over values independent of~$t$,
and $f$ is the identity on the resulting values.  For polymorphic functions,
the functions are defined over \emph{all} types with the relevant type
constraint (e.g.~$Set~t$ for types whose members can be included in sets), and
corresponding types built from~$T_1$ and~$T_2$ satisfy the same type
constraints.

If a function |g| (or infix binary operator) can be applied only to values
that do not use $t$, and so necessarily returns a result that it independent
of~$t$, then its semantics is independent of the underlying types, i.e.~$g_1 =
g_2$.  Further, composing it with~$f$ has no effect.  Hence $f \circ g_1 = g_1
= g_2 = g_2 \circ f$.  Examples include operators over integers
(e.g.~$+$,~$<=$) or booleans~(e.g.~|and|,~|!|).  The same is true for
instances of polymorphic functions that are independent of~$t$.  Examples
include equality or inequality tests over values independent of $t$; or
application of the set operators to sets of values that are independent
of~$t$.

It is straightforward to show that each of the following polymorphic built-in
functions satisfies the condition $f \circ g_1 = g_2 \circ f$.
%
\begin{itemize}
\item the set functions |empty|, |union|, |Union|, |Set| and |Seq|;

\item the sequence functions |concat|,  |head|, |length|, |null|, |set|
and~|tail|; 

\item the map functions |emptyMap|, |mapDelete|, |mapFromList|, |mapLookup|,
  |mapMember|, |mapUpdate|, |mapUpdateMultiple| and~|Map| (recall that we
  assume that the domain type of each map is independent of~$t$);
\end{itemize}

However, the condition $f \circ g_1 = g_2 \circ f$ does not hold for all
polymorphic built-in functions when applied to arguments of type~$t$.  For
example, it does not hold for the set-cardinality functions (denoted $\#_1$,
$\#_2$) corresponding to the built-in function |card|; if $f(A) = f(B) = C$
then:
\[
f(\#_1\set{A,B}) = f(2) = 2 \ne 1 = \#_2\set{C} = \#_2(f(\set{A,B})).
\]
For a binary operator with denotations~$\oplus_1$ and $\oplus_2$ (written as
infix operators), the required condition can be stated as $f(v \oplus_1 v') =
f(v) \oplus_2 f(v')$.  The condition does not hold for the set-intersection
functions.
\[
f(\set{A} \inter_1 \set{B}) = f(\set{}) = \set{} \ne \set{C} 
  = \set{C} \inter_2 \set{C} = f(\set{A}) \inter_2 f(\set{B}).
\]
All the built-in functions for which the condition does not hold were
disallowed by condition~\ref{item:built-in-functions} of
Definition~\ref{defn:data-independent}.

%% But not
%% \begin{itemize}
%% \item the set functions |card|, |diff|, |inter|, |Inter|, |member| and |seq|;
%%   or the subset relations |<=|, |<|, |>=| and |>|;

%% \item the sequence function |elem|;

%% \item the map function |mapToList|;

%% \item the function~|show|.
%% \end{itemize}

\framebox{???} Compression functions; relation functions.  |extensions|
|productions|.


%% \begin{lemma}
%% If $\dom g_2 = f(\dom g_1)$ and  $f \circ g_1 = g_2 \circ f$, then
%% $f(g_1) = g_2$.
%% \end{lemma}

%% \begin{proof}
%% \begin{calc}
%% & f(g_1) \\ 
%% = & \set{ (f(v), f(g_1(v))) \| v \in \dom g_1 } \\ 
%% = & \set{ (f(v), g_2(f(v))) \| v \in \dom g_1 } \\
%% = & \com{letting $w = f(v)$} \\ 
%% & \set{ (w, g_2(w)) \| w \in f(\dom g_1) }  \\
%% = & \set{ (w, g_2(w)) \| w \in \dom g_2 } \\
%% = & g_2.
%% \end{calc}
%% \end{proof}
%
%% The converse does not hold.  Indeed, if $f : T_1 \fun T_2$ (suitably lifted),
%% and $g : T_1 \fun T_1$, then $g \circ f$ might not even be well-typed.  The
%% above lemma requires $g$ to be suitably polymorphic --- as is the case with
%% most built-in functions.

%% Further, if a built in function with name~|g| has denotation~$g$, then
%% \[
%% f(\eval \rho \M{g}) 
%% = f(g) 
%% = \set{ (f(v), f(g(v))) \| v \in \dom g } 
%% = \set{ (f(v), g(f(v))) \| v \in \dom g } 
%% = \set{ (w, g(w)) \| w \in f(\dom g) } 
%% = ?? 
%% f(\eval (f\after\rho) \M{g}) 
%% \]

%% Note that if $g$ returns results that are independent of~$t$ (on
%% which values $f$ is the identity), the consequent is equivalent $g(v) =
%% g(v')$.  
%% For a binary operator with denotation~$\oplus$, the required
%% condition can be stated as $f(v \oplus v') = f(v) \oplus f(v')$. 
%% : if $f(v) = f(v')$ and $f(w) = f(w')$ then
%% $f(v \oplus w) = f(v' \oplus w')$.


%% Consider a built-in function |g| with semantics~$g$ such that $f(g(x)) =
%% g(f(x))$ for all~$x$.  Then  clause~\ref{item:eval} holds for an application
%% of~|g|: 
%% %
%% \begin{calc}
%% & f(\eval \rho ~ (\M{g}(e))) \\
%% = &  f(g(\eval \rho e)) \\
%% = & g(f(\eval \rho e)) \\
%% = & g(\eval (f\after\rho) e) \\
%% = & \eval (f\after\rho) ~(\M{g}(e)).
%% \end{calc}
%% %
%% For a binary operator with semantics $\oplus$, we can prove a corresponding
%% result provided $f(x \oplus x') = f(x) \oplus f(x')$, for all $x, x'$. 



%%%%%%%%%%%%%%%%%%%%%%%%%%%%%%%%%%%%%%%%%%%%%%%%%%%%%%%


\subsubsection{Collections}
\label{sec:collections}

\begin{itemize}
\item
Case of a set comprehension. Below, the set comprehensions in the first and
last line are syntax, and the other set comprehensions are semantics. 
% 
\begin{calc}
& f(\eval \rho ~ \set{ e \| stmts }) \\
= & f(\set{ \eval \rho'~e \| \rho' \in \evalStmtsSet \rho~stmts }) \\
= & \set{ f(\eval \rho'~e) \| \rho' \in \evalStmtsSet \rho~stmts } \\
= & \com{inductive hypothesis} \\
  & \set{ \eval (f(\rho'))~e) \| \rho' \in \evalStmtsSet \rho~stmts } \\
= & \com{letting $\rho'' = f(\rho')$} \\ 
  & \set{ \eval \rho''~e) \| \rho'' \in f(\evalStmtsSet \rho~stmts) } \\
= & \com{clause \ref{item:evalStmts}} \\
  & \set{ \eval \rho''~e) \| \rho'' \in \evalStmtsSet (f(\rho))~stmts } \\
= & \eval (f(\rho)) ~ \set{e \| stmts}.
\end{calc}

\item The case for  a sequence comprehension $\M{<}\, e \| stmts \,\M{>}$ is
very to that for a set comprehension, except $stmts$ are evaluated using
$\evalStmts$, which produces a sequence of environments.

\item The cases for a set $\{e_1, \ldots, e_n\}$, ranged integer set
$\set{e .. e'}$ (where $e$ and~$e'$ evaluate to integers), infinite integer
set $\set{ e .. }$, sequence $\M{<} e_1, \ldots, e_n \M{>}$, ranged integer
sequence $\M< e .. e' \M>$, and infinite integer equence $\M< e .. \M>$, are
straightforward.

\item The case for a map $(\| k_1 => v_1, ..., k_n => v_n \|)$ is also
straightforward: recall that the domain type of a map mucs be independent
of~$t$, so eack $k_i$ evaluates to a value that is invariant under~$f$. 

\item extensions sets \framebox{...}


\end{itemize}


%%%%%%%%%%%%%%%%%%%%%%%%%%%%%%%%%%%%%%%%%%%%%%%%%%%%%%%

\subsubsection{Other stuff.}

Case |extensions|, |productions|,


%%%%%%%%%%%%%%%%%%%%%%%%%%%%%%%%%%%%%%%%%%%%%%%%%%%%%%%

\subsubsection{Processes}


We now consider process expressions.

\paragraph{Basic processes.} $STOP$, $SKIP$, $div$: these are all trivial. 

%%%%%%%%%%%%%%%%%%%%%%%%%%%%%%%%%%%%%%%%%%%%%%%%%%%%%%%

\subsubsection{Prefixing}  
\label{sec:prefixing}

Consider the prefixing expression $e_c ~ f_1 \ldots f_n \then P$, where $e_c$
is an expression that should evaluate to a possibly incomplete event (i.e.~a
channel name with zero or more fields supplied), $n \ge 0$, and each~$f_i$ is
a field of one of the following forms:
  %
  \begin{itemize}
  \item $? pat$, where $pat$ is a pattern: this offers an external choice
    between all values matching $pat$ and consistent with the channel type. 

  \item $?pat : E$, where $pat$ is a pattern, and $E$ should evaluate to a
    set: this is like the previous case, but restricts values to elements
    of~$E$. 

  \item $!e$, where $e$ is an expression: this matches only the value of~$e$. 

  \item $\$ pat$: this performs an internal choice between all values matching
    $pat$ and consistent with the channel type.
 
  \item $\$ pat : E$: this is like the previous case, but restricts values to
    elements of~$E$.
  \end{itemize}
  %
  The semantics of pattern matching by fields depends upon the field's
  location: a variable pattern (e.g.~$?x$) matches a \emph{single} complete
  value, \emph{except} in the final field where it matches the whole of the
  rest of the event.  For example, the prefix construct $c?x?y$ matches the
  event $c.1.2.3$, binding $x$ to~$1$, and $y$ to~$2.3$.

  If there are one or more \$-fields, then the process has initial
  $\tau$-transitions to resolve the internal choices; each \$-field is
  replaced by a field $!x$, where $x$ is a fresh variable bound in the
  environment to the value chosen; also the environment is updated
  corresponding to any other variables bound in the field.  The definition of
  \CSPm\ requires that every \$-field precedes every !- or ?-field.

  Recall that we assume that distinct variables have distinct names.  In
  particular, this means that the initial expression~$e_c$ will use no
  variables bound by a \$-field; this is required, since we will
  evaluate~$e_c$ \emph{after} the \$-fields. 

  We perform a case analysis on whether or not the prefixing construct
  contains any \$-fields.

  %%%%%%%%%%

\paragraph{Case of no \$-fields.} 

Consider first a prefixing construct of the form ${e_c~{\it fields} \then P}$
that contains no $\$$-fields.  We define the semantics of prefixing using a
function
\[
  \evalField :: 
    \begin{align}
    Env \fun Bool \fun Value \fun Field \fun  \power (Value \cross Env).
    \end{align}
\]
If $c.v$ is an incomplete event on channel~$c$, and $field$ is a field, then
for a correct usage, $\evalField \rho~last~(c.v)~field$ gives a set of
$(w,\rho')$ pairs; for each such pair, $c.v.w$ is an extension of~$c.v$
compatible with~$c$, corresponding to adding this field, and $\rho'$ is the
updated environment caused by binding any variables in patterns in~$field$.
Compatibility will mean that if $\rho(c) = \channel S$ then $v.w$ is a prefix
of some element of~$S$; we denote this $v.w \le \rho(c)$.  However, an error
will occur if a $!e$ or $?pat:E$~field gives a value not compatible with the
channel.  The argument~$last$ of $\evalField$ indicates whether this is the
last field in the prefixing construct, which affects the semantics of pattern
matching, as explained above.

%% Recall that we assumed that the script is error-free.  This implies 

We prove the following
\begin{equation}\label{eqn:evalField}
f (\evalField \rho~last~(c.v)~field)  = 
    \evalField (f(\rho))~last~(c.f(v))~field .
\end{equation}
%
We perform a case analysis on~$field$.
%
\begin{itemize}
\item Case $!e$. 
  \begin{calc}
  &  f (\evalField \rho~last~(c.v)~(!e)) \\
  = & f(\Let w = \eval \rho~e \In 
        \If v.w \le \rho(c) \Then \set{(w, \rho)} \Else error) \\
  = & \com{Lemma~\ref{lem:invariant-prefix}} \\
  & \begin{align}
    \Let w = \eval \rho~e \In \\
    \If f(v).f(w) \le f(\rho(c)) \Then \set{(f(w), f(\rho))} \Else f(error)
    \end{align} \\
  = & \com{letting $w' = f(w)$} \\
  & \begin{align}
    \Let w' = f(\eval \rho~e) \In \\
    \If f(v).w' \le (f(\rho))(c) \Then \set{w', f(\rho))} \Else error
    \end{align} \\
  = & \com{inductive hypothesis applied to~$e$} \\
  & \begin{align}
    \Let w' = \eval (f(\rho))~e \In \\
    \If f(v).w' \le (f(\rho))(c) \Then \set{w', f(\rho))} \Else error
    \end{align} \\
  = & \evalField (f(\rho))~last~(c.f(v))~(!e).
  \end{calc}

\item Case $?pat$.  We write $arity(pat)$ for the number of fields matched by
  $pat$ in the case that this is not the final field.  Similarly, we write
  $arity(w)$ for the number of complete values within~$w$.  Below, $W$ is the
  set of candidate values that $pat$ could be matched against: it contains all
  values $w$, with the appropriate arity, that can be appended to~$v$ to give
  a prefix of an element of~$\rho(c)$.
%
  \begin{calc}
  & f (\evalField \rho~last~(c.v)~(?pat)) \\
  = & f(
    \begin{align}
    \Let W = 
      \begin{align}
      \If last \Then \set{w \mid v.w \in \rho(c)} \\
      \Else \set{w \mid v.w \prefix \rho(c), arity(pat) = arity(w)} 
      \end{align} \\
    \In 
    \set{(w, \bind \rho~pat~w) \mid w \in W, \matches \rho~pat~w})
    \end{align} \\
  = & 
    \begin{align}
    \Let W = 
      \begin{align}
      \If last \Then \set{w \mid v.w \in \rho(c)} \\
      \Else \set{w \mid v.w \prefix \rho(c), arity(pat) = arity(w)} 
      \end{align} \\
    \In 
    \set{(f(w), f(\bind \rho~pat~w)) \mid  w \in W, \matches \rho~pat~w}
    \end{align} \\
  = & \com{parts \ref{item:matches} and \ref{item:bind}} \\
        %% letting $w' = f(w)$,\, $W' = f(W)$; Corollary \ref{cor:channel-types}} \\
  & \begin{align}
    \Let W = 
      \begin{align}
      \If last \Then \set{w \mid v.w \in \rho(c)} \\
      \Else \set{w \mid v.w \prefix \rho(c), arity(pat) = arity(w)} 
      \end{align} \\
    \In 
    \set{(f(w), \bind (f(\rho))~pat~(f(w))) \mid \\
    \quad\qquad   w \in W, \matches (f(\rho))~pat~(f(w))}
    \end{align} 
  \\
  = & \com{letting $w' = f(w)$,\, $W' = f(W)$} \\
  & \begin{align}
    \Let W' = 
      \begin{align}
      \If last \Then \set{f(w) \mid v.w \in \rho(c)} \\
      \Else \set{f(w) \mid 
         \begin{align}
         v.w \prefix \rho(c), 
         arity(pat) = arity(w)} 
         \end{align}
      \end{align} \\
    \In 
    \set{(w', \bind (f(\rho))~pat~w') \mid 
       w' \in W', \matches (f(\rho))~pat~w'}
    \end{align} 
  \\
  = & \com{Lemma~\ref{lem:invariant-prefix}} \\
  & \begin{align}
    \Let W' = 
      \begin{align}
      \If last \Then \set{f(w) \mid f(v).f(w) \in (f(\rho))(c)} \\
      \Else \set{f(w) \mid 
         \begin{align}
         f(v).f(w) \prefix (f(\rho))(c), \\
         arity(pat) = arity(w)} 
         \end{align}
      \end{align} \\
    \In 
    \set{(w', \bind (f(\rho))~pat~w') \mid 
      w' \in W', \matches (f(\rho))~pat~w'}
    \end{align} 
  \\
  = & \com{letting $w' = f(w)$; $arity(w) = arity(f(w))$} \\
  & \begin{align}
    \Let W' = 
      \begin{align}
      \If last \Then \set{w' \mid f(v).w' \in (f(\rho))(c)} \\
      \Else \set{w' \mid 
         \begin{align}
         f(v).w' \prefix (f(\rho))(c), 
         arity(pat) = arity(w')} 
         \end{align}
      \end{align} \\
    \In 
    \set{(w', \bind (f(\rho))~pat~w') \mid 
       w' \in W', \matches (f(\rho))~pat~w'}
    \end{align} 
  \\
  = & \evalField (f(\rho))~last~(c.f(v))~(?pat) .
  \end{calc}

%% *** Need that $\rho(c)$ is symmetric: if $v \in \rho(c)$ and $f(v) = f(v')$
%% then $v' \in \rho(c)$.  

\item Case $?pat:E$.  This is very similar to the previous case.
  Note that $f(\eval \rho~ E) = \eval(f(\rho))~ E$, by the inductive
  hypothesis.  Recall that we do not assume $E$ is invariant.

  If, for some $w \in\eval \rho~ E$,\, $v.w$ is not a prefix of $\rho(c)$,
  then the left-hand side evaluates to $error$.  But in this case,
  $f(w) \in \eval (f(\rho))~E$, and $f(v).f(w)$ is not a prefix of
  $(f(\rho))(c)$ by Corollary~\ref{cor:channel-types}, so the right-hand
  side also evaluates to $error$.  
% 
  Conversely, if for some $w' \in \eval (f(\rho))~ E$,\, $f(v).w'$ is
  not a prefix of $(f(\rho))(c)$, then the right-hand side evaluates to
  $error$.  But then $w' = f(w)$ for some $w \in \eval \rho~E$, and $v.w$ is
  not a prefix of $\rho(c)$, so the left-hand side also evaluates to~$error$. 

  %% An error arises if any element of $\set{v.w \| w \in \eval \rho~ E}$ is not
  %% a prefix of a member of $\rho(c)$; in which case, an element of $\set{v.w'
  %%   \| w' \in \eval (f(\rho))~ E}$ is not a prefix of a member of
  %% $(f(\rho))(c)$, and vice versa, by Corollary~\ref{cor:channel-types}; both
  %% sides then evaluate to $error$.

  Otherwise, in the main set comprehensions, $w$ ranges over $W \inter \eval
  \rho~ E$.  Then $w'$ ranges over $W' \inter f(\eval \rho~ E) = W' \inter
  \eval(f(\rho))~ E$.
\end{itemize} %% End of case analysis on field

%%%%%%%%%% evalFields

We now define a function that accumulate over the fields to get the
resulting events and environments (but handling errors appropriately).
\[
\begin{align}
\evalFields :: 
  \begin{align}
  Env \fun  Value \fun Field^* \fun \\
  \qquad (\power (Value \cross Env) \union \set{error})
  \end{align} \\
\evalFields \rho~(c.v)~\seq{} = \set{(c.v,\rho)} \\
% \evalFields \rho~(c.v)~\seq{f}  =  \evalField \rho~true~(c.v)~f \\
\evalFields \rho~(c.v)~(\seq{field} \cat fs)  = \\
\qquad
  \begin{align}
  \Let S = \evalField \rho~(fs=\seq{})~(c.v)~field \In \\
  \If S = \error \Then \error \\
  \Else \Union \set{ \evalFields \rho'~(c.w)~fs \mid  (c.w, \rho') \in S }.
  \end{align}
\end{align}
\]
(The term $fs=\seq{}$ tests whether $f$ is the last field.)
We can then show 
\begin{eqnarray*}%\label{eqn:evalFields}
f(\evalFields \rho~(c.v)~fs) & = & 
  \evalFields (f(\rho))~(c.f(v))~fs ,
\end{eqnarray*}
%
by an induction on $fs$ (the proof is identical to the corresponding proof
in~\cite{symmetry-reduction}).

%%%%%%%%%%

We now consider the process $e_c~ fs \then P$.  Let $c.v = \eval \rho~e_c$.
Then $\eval (f(\rho))~e_c = c.f(v)$, by the inductive hypothesis.  Let
%
\begin{eqnarray*}
L & = & \eval \rho~ (e_c~ fs \then P), \\
S & = & \evalFields \rho~(c.v)~fs, \\
L' & = &  \eval (f(\rho))~ (e_c~fs \then P), \\
S' & = & \evalFields (f(\rho))~(c.f(v))~fs.
\end{eqnarray*}%
%
The above result shows $f(S) = S'$.  We need to show $f(L) = L'$.

If $S = error$ then $S' = error$, and vice versa; but then each of $L$, $f(L)$
and $L'$ equal error, so $f(L) = L'$, as required. 

Otherwise, $L$ is an LTS as follows.
%
\begin{itemize}
\item The initial state is $(e_c~ fs \then P, \rho)$.

\item For each $(a, \rho') \in S$, there is a transition labelled with $a$
  from the initial state.  %  to $(P, \rho')$.

\item For each $(a, \rho') \in S$, the sub-LTS following the transition of
  the previous item equals $\eval \rho'~ P$.
\end{itemize}
%
Then $f(L)$ is an  LTS as follows.
%
\begin{itemize}
\item The initial state is $(e_c~fs \then P, f(\rho))$.
  This is the same as the initial state of~$L'$.

\item From the initial state, for each $(a, \rho') \in S$, there is a
transition labelled with $f(a)$.  Letting $a' = f(a)$ and $\rho'' = f(\rho')$,
  this is equivalent to saying that for every $(a',\rho'') \in f(S) = S'$,
  there is a transition labelled with~$a'$ from the initial state.  This is
  the same as the initial transitions of~$L'$.

\item For each $(a, \rho') \in S$, the sub-LTS after the transition of the
  previous item equals $f(\eval \rho'~ P)$.  But this equals $\eval
  (f(\rho'))~P$, by the inductive hypothesis.  Letting $a' = f(a)$ and
  $\rho'' = f(\rho')$, this is equivalent to saying that for every
  $(a',\rho'') \in f(S) = S'$, the sub-LTS after this transition equals $\eval
  \rho''~P$.  This is the same as for~$L'$.
\end{itemize} 
%
Hence $f(L) = L'$, as required. 

%%%%%%%%%%

\paragraph{Case of at least one \$-field.}

Now suppose that the prefixing construct contains at least one $\$$-field.
We define the semantics using a function 
\[
\evalDollarField ::
  Env \fun Bool \fun Value \fun Field \fun  \power (Value \cross Env).
\]
If $c.v$ is an incomplete event on channel~$c$, and $field$ is a $\$$-field,
then, for a correct usage, $\evalDollarField \rho~last~(c.v)~field$ gives a
set of $(w,\rho')$ pairs; each~$w$ is a value that could be communicated in
the place of~$field$, compatible with~$c.v$, and $\rho'$ is the updated
environment caused by binding any variables in patterns in~$field$.
Compatibility again means that if $\rho(c) = \channel S$ then $v.w$ is a
prefix of some element of~$S$, denoted $v.w \le \rho(c)$.  The argument~$last$
of $\evalField$ again indicates whether this is the final field in the
prefixing construct.

We can prove the following, for $field$ a \$-field:
\[ % begin{equation}\label{eqn:evalDollarField}
  f (\evalDollarField \rho~last~(c.v)~field)  = 
    \evalDollarField (f(\rho))~last~(c.f(v))~field .
\] % end{equation}
The proof is almost identical to the cases for ?-fields, and so is omitted. 

We now define a function that, given an environment~$\rho$, an incomplete
event~$c.v$, and a list of fields~$fs$, gives a set of pairs $(fs', \rho')$
such that $fs'$ is the result of substituting each \$-field of $fs$ with a
field $!x$ where $x$ is a fresh variable, and $\rho'$ is the environment
resulting from binding~$x$ to a value~$w$ that could instantiate this field,
and also binding any variables in the \$-fields with the corresponding values.
The definition makes use of the fact that every \$-field must precede every !-
or ?-field.
\[
\begin{align}
\evalDollarFields :: 
  Env \fun  Value \fun Field^* \fun  \power (Field^* \cross Env) ,
\\
\evalDollarFields \rho~(c.v)~\seq{} = \set{(\seq{},\rho)} ,
\\
\evalDollarFields \rho~(c.v)~(\seq{field} \cat fs)  = \\
\quad
  \begin{align}
  \If \mbox{$field$ is a \$-field} \\
  \Then 
    \begin{align}
    \set{ (\seq{!x} \cat fs', \rho'') \mid 
    (w,\rho') \in \evalDollarField \rho~(fs=\seq{})~(c.v)~field, \\
    \qquad\quad (fs', \rho'') \in
       \evalDollarFields (\rho' \union \set{x \mapsto w})~ (c.v.w)~fs } \\
    \mbox{where $x$ is  fresh}
    \end{align} \\
  \Else \set{ (\seq{field} \cat fs, \rho) } .
  \end{align}
\end{align}
\]

We can then  show
\begin{eqnarray*} % \label{eqn:evalDollarFields}
f (\evalDollarFields \rho~(c.v)~fs) & = & 
  \evalDollarFields (f(\rho))~(c.f(v))~fs ,
\end{eqnarray*}
%
by an induction on $fs$ (again, identical to the corresponding proof
in~\cite{symmetry-reduction}).

We now consider the process $e_c~ fs \then P$; recall we are assuming there is
at least one \$-field.  Let $c.v = \eval \rho~e_c$.  So $\eval
(f(\rho))~e_c = f (\eval \rho~e_c) = c.f(v)$, using the inductive
hypothesis.  So let
\begin{eqnarray*}
L & = & \eval \rho ~ (e_c~fs \then P), \\
S & = & \evalDollarFields \rho ~(c.v)~ fs, \\
L' & = & \eval (f(\rho))~ (e_c~fs \then P), \\
S' & = & \evalDollarFields (f(\rho)) ~(c.f(v)) ~fs .
\end{eqnarray*}%
%
The above result shows $f(S) = S'$.  We need to show $f(L) = L'$.

If $S = \set{}$ then $S' = \set{}$, and vice versa.   But then each of $L$,
$f(L)$ and~$L'$ equals $error$, so $f(L) = L'$, as required.  The case where
$S$ and~$S'$ equal $error$ is similar.

Otherwise, $L$ is an augmented LTS as follows.
\begin{itemize}
\item The initial state is $(e_c~fs \then P, \rho)$.

\item From the initial state, for each $(fs',\rho') \in S$, there is a
  $\tau$-transition to $(e_c~fs' \then P, \rho')$.

\item For each $(fs',\rho') \in S$, the sub-LTS rooted at
  $(e_c~fs' \then P, \rho')$ equals $\eval~\rho'~(e_c~fs' \then P)$.
\end{itemize}

%
Then $f(L)$ is an augmented LTS as follows.
\begin{itemize}
\item The initial state is
  $(e_c~fs \then P, f(\rho))$.  This is the
  same as the initial state of~$L'$.

\item From this initial state, for each $(fs',\rho') \in S$, there is a
  $\tau$-transition to
  \(
  (e_c~fs' \then P, f(\rho')). 
  \)
  %since the only constants from~$\T$ must appear in~$fs'$ and~$v$.  
  Letting $\rho'' = f(\rho')$, this is equivalent to
  saying that for each $(fs',\rho'') \in f(S) = S'$, there is a
  $\tau$-transition to $(e_c~fs' \then P,\linebreak[1] \rho'')$.
  This is the same as the initial transitions of~$L'$.

\item For each $(fs',\rho') \in S$, the sub-LTS rooted at
  $(e_c~fs' \then  P,\linebreak[2]
    {f(\rho')})$ is 
  equal to 
  \( f(\eval~\rho'~(e_c~fs' \then P)) . \)
  But this equals $\eval~{(f(\rho'))} ~ (e_c~fs' \then P)$
  using the result for the case that there are no \$-fields.
  % (equation~(\ref{eqn:prefix-no-dollar})).  
  Letting 
  $\rho'' = f(\rho')$, this is equivalent to saying that for each
  $(fs',\rho'') \in f(S) = S'$, the sub-LTS rooted at
  $(e_c~fs' \then P, \linebreak[2] \rho'')$ is equal to
  $\eval~\rho''~(e_c~fs' \then P)$.  This is the same as for~$L'$.
\end{itemize}
%
Hence $f(L) = L'$, as required.





%%%%%%%%%%%%%%%%%%%%%%%%%%%%%%%%%%%%%%%%%%%%%%%%%%%%%%%

\paragraph{Non-indexed sequential operators.}

We now consider non-indexed sequential operators, other than renaming.  We
start with external choice.

Below we write $\extchoice$ for a semantic operator: ``${P \extchoice Q}$''
represents an external choice between~$P$ and~$Q$, each of which is an LTS\@.
We show that
%
\begin{eqnarray*}
f(P \extchoice Q) & = & f(P) \extchoice f(Q).
\end{eqnarray*}
%
For every transition $P \trans{e} P'$, the initial state of each of the above
processes has a transition labelled $f(e)$ to~$f(P')$; and similarly for
transitions of~$Q$.

We then have the following (for syntactic expressions~$P$ and~$Q$):
%
\begin{calc}
& f(\eval \rho ~ (P \; \M{[]} Q) \\
= & f((\eval \rho~P) \extchoice (\eval \rho~Q)) \\
= & \com{above result} \\
& f(\eval \rho~P) \extchoice f(\eval \rho~Q) \\
= & \com{inductive hypothesis} \\
& (\eval (f(\rho))~P) \extchoice (\eval (f(\rho))~Q) \\
= & \eval (f(\rho)) (P \; \M{[]} Q).
\end{calc}

The proofs for the binary operators nondeterministic choice (\CSPM{_ \|~\|
_}), sliding choice (\CSPM{_ [> _}), interrupt (\CSPM{_ /\\ _}), and
sequential composition (\CSPM{_ ; _}) are very similar.

For the exception operator (\CSPM{_ [\| E \|> _}), recall that we assume that
|E| is invariant (item~\ref{item:di-invariant} of
Definition~\ref{defn:data-independent}).  Hence $\eval \rho~E$ is
$T$-invariant, by Lemma~\ref{lem:invariant}.
%  
We write $\_ \throw{A} \_$ for the corresponding semantic operator.  We show
that if $A$ is $T$-invariant then
\begin{eqnarray*}
f(P \throw{A} Q) & = & f(P) \throw{f(A)} f(Q).
\end{eqnarray*}
%
On the left-hand side, every transition of~$P$ on an event from~$A$ is replace
by a transition to the initial state of~$Q$, and then all events are renamed
by~$f$.  On the right-hand side, the events are renamed first, and then every
transition from $f(P)$ on an event from~$f(e) \in f(A)$ is replaced by a
transition to the initial state of~$f(Q)$.  But if $f(e) \in f(A)$ then $e \in
A$, by Lemma~\ref{lem:T-invariant-inclusion}, so there is a corresponding
transition on the left-hand side.  The rest of the proof is then as for
external choice.

For hiding, note that if $A$ is $T$-invariant, then
\begin{eqnarray*}
f(P \hide A) & = & f(P) \hide f(A).
\end{eqnarray*}
%
One the right-hand side, an event $e$ of~$P$ is hidden if $f(e) \in f(A)$; but
then $e \in A$, by Lemma~\ref{lem:T-invariant-inclusion}.  The rest of the
proof is then as for external choice.

A guarded process $g ~\M{\&}~ P$ is equivalent to $\M{if}~ g ~ \M{then}~
P ~ \M{else STOP}$.  The result follows from the result for if-statements.

%%%%%%%%%%%%%%%%%%%%%%%%%%%%%%%%%%%%%%%%%%%%%%%%%%%%%%%

\paragraph{Renaming}

The semantics of a renaming is a set of pairs of events $(e,e')$ indicating
that~$e$ is renamed to~$e'$.  
%% We start by considering renamings where each individual renaming \CSPM{e <-
%% e'} applies to complete events; we then extend to channels.
We start by showing the renaming itself satisfies the desired result.  Later,
we use this to show that the application of a renaming to a process also
satisfies the desired result.  
%
\begin{lemma}
\label{lem:eval-renaming}
Let $\rename{ren}$ be a renaming.  Then 
\begin{eqnarray*}
f(\eval \rho~\rename{ren}) & = & \eval (f\after\rho)~ \rename{ren}.
\end{eqnarray*}
\end{lemma}
%
\begin{proof}
Let $\rename{ren} = \rename{ e_1 \lArrow e_1', \ldots, e_k \lArrow e_k' \|
stmts}$.  Suppose, for the moment, that each individual renaming $e_i \lArrow
e_i'$ is complete, i.e.~it renames an event as opposed to a channel.  Below,
``$id_1$'' represents the identity function over events based on~$T_1$, and
``$id_2$'' represents the identity function over events based on~$T_2$.
%
\begin{calc}
& f(\eval \rho~
   \rename{ e_1 \leftarrow e_1', \ldots, e_k \leftarrow e_k' \|  stmts}) \\
= & f(id_1 \oplus \set{ (\eval \rho'~e_i, \eval \rho'~e_i') \|
     \rho' \in \evalStmtsSet \rho~stmts, i \in \set{1,\ldots,k} }) \\
= & id_2 \oplus \set{ (f(\eval \rho'~e_i), f(\eval \rho'~e_i')) \|
     \rho' \in \evalStmtsSet \rho~stmts, i \in \set{1,\ldots,k} } \\
= & \com{inductive hypothesis} \\
  & \begin{align}
    id_2 \oplus\null \\ 
    \quad \set{ (\eval (f\after\rho')~e_i, \eval (f\after\rho')~e_i') \|
     \rho' \in \evalStmtsSet \rho~stmts, i \in \set{1,\ldots,k} }
     \end{align} \\
= & \com{letting $\rho'' = f \after \rho'$} \\ 
  & id_2 \oplus \set{ (\eval \rho''~e_i, \eval \rho''~e_i') \|
     \rho'' \in f(\evalStmtsSet \rho~stmts), i \in \set{1,\ldots,k} } \\
= & \com{inductive hypothesis, part~\ref{item:evalStmtsSet}} \\
  & id_2 \oplus \set{ (\eval \rho''~e_i, \eval \rho''~e_i') \|
     \rho'' \in \evalStmtsSet (f\after\rho)~stmts, i \in \set{1,\ldots,k} } \\
= & \eval (f\after\rho)
       \rename{ e_1 \leftarrow e_1', \ldots, e_k \leftarrow e_k' \|  stmts}.
\end{calc}

Finally, if an individual renaming $e_i \lArrow e_i'$ is not complete, we can
replace it by a complete renaming of the form $e_i.x_1.\ldots.x_k \lArrow
e_i'.x_1.\ldots.x_k$, with additional generators for $x_1,\ldots,x_k$, and then
apply the above result.
\end{proof}

%%%%%

Recall that each renaming satisfies condition~\ref{item:di-renaming} of
Definition~\ref{defn:data-independent}.  We will need the following lemma,
which shows that the value of a renaming is consistent with~$f$.
%
\begin{lemma}
\label{lem:renaming-invariant}
Consider a renaming $\rename{r} = \rename{ e_1 \lArrow e_1', \ldots, e_k
  \lArrow e_k' \| stmts}$, and let $R = \eval \rho~ \rename{r}$.  Then
%
\begin{eqnarray*}
%\label{eqn:renaming-invariant}
(v,v') \in R \land f(v) = f(w) & \implies &
  \exists w' \spot (w,w') \in R \land f(v') = f(w').
\end{eqnarray*}
\end{lemma}

%%%%%

\begin{proof}
Without loss of generality, we can assume that each individual renaming
$e_i \lArrow e_i'$ is complete, i.e.~it renames an event as opposed to a
channel (otherwise we can use an argument similar to that at the end of the
proof of Lemma~\ref{lem:eval-renaming}).

Suppose $(v,v') \in R$ and $f(v) = f(w)$.  Suppose $(v,v')$ is produced by the
individual renaming $e_j \lArrow e_j'$.  

Consider the fields $v_1,\ldots,v_k$ in~$v$ of types that depend upon~$t$.  By
condition~\ref{item:di-renaming} of Definition~\ref{defn:data-independent},
these correspond to distinct identifiers $x_1,\ldots,x_k$ in~$e_j$, each of
which is bound by a generator $x_i \lArrow s_i$ in $stmts$, where $s_i$ is
invariant.  Let $S_i = \eval \rho_i~s_i$, where $\rho_i$ is the
environment~$\rho$ augmented corresponding to values taken by earlier
generators in the production of~$(v,v')$; so $v_i \in S_i$.  By
Lemma~\ref{lem:invariant}, $S_i$~is invariant.

Let the corresponding fields in~$w$ be $w_1,\ldots,w_k$, so $f(v_i) = f(w_i)$
for each~$i$.  We show that the identifiers $x_1,\ldots,x_k$ could also be
bound to $w_1,\ldots,w_k$ by the generators, with all other generators being
bound as in the production of~$(v,v')$.  
Consider a generator $x_i \,\M{<-}\, s_i$.  Let $\rho_i'$ be the
environment~$\rho$ augmented corresponding to values taken by earlier
generators.  But, by assumption, $s_i$ uses none of the earlier bound
variables~$x_1,\ldots,x_{i-1}$, and so $\eval \rho_i'~s_i = \eval \rho_i~s_i =
S_i$.   And $w_i \in S_i$, since $S_i$ is $T$-invariant, so this generator can
indeed produce~$w_i$.  Further, the identifiers $x_1,\ldots,x_k$ do not
appear in predicates in $stmts$, so each such predicate will still be true
with this binding.  Then  the individual renaming $e_j \lArrow e_j'$
produces $(w,w')$ for some $w'$ such that $f(v') = f(w')$.
%% Consider the fields
%% $v_1,\ldots,v_k$ in~$v$ of type~$t$.  By condition~\ref{item:di-renaming} of
%% Definition~\ref{defn:data-independent}, these correspond to distinct
%% identifiers $x_1,\ldots,x_k$ in~$e_j$, each of which is bound by a generator
%% $x_i \lArrow t$ in $stmts$.  Let the corresponding fields in~$w$ be
%% $w_1,\ldots,w_k$, so $f(v_i) = f(w_i)$ for each~$i$.
%% %
%% Consider the case where each~$x_i$ is bound to $w_i$, and every other
%% generator is bound as in the production of $(v,v')$.  These identifiers do not
%% appear in predicates in $stmts$, so each such predicate will still be true
%% with this binding.  Then the individual renaming $e_j \lArrow e_j'$
%% produces $(w,w')$ for some $w'$ such that $f(v') = f(w')$.
\end{proof}

%% \framebox{Generalise}, like for set comprehensions.

%%%%%%%%%%%%%%%%%%%%%%%%%%%%%%%%%%%%%%%%%%%%%%%%%%%%%%%%%%%%

%% \framebox{** Partial events?}  Extend to total. 

Consider a syntactic renaming $\rename{ren}$, which corresponds to a
semantic renaming $R$ (i.e.~a set of pairs).  When applied to a LTS~$P$, each
state $(Q,\rho)$ of~$P$ is replaced by $(Q\M{[[}ren\M{]]}, \rho)$, and the
transitions renamed according to~$R$.  We denote this renamed LTS
$P\rename{R}^{ren}$.
%
\begin{lemma}
\label{lem:renaming-f}
Let $P$ be an LTS, $\rename{ren}$ a syntactic renaming, and $R$ the
corresponding renaming relation.  Then
%
\begin{eqnarray*}
f(P\rename{R}^{ren}) & = & (f(P))\rename{f(R)}^{ren}.
\end{eqnarray*}
%
\end{lemma}

%%%%%

\begin{proof}
For each state $(Q,\rho)$ of~$P$, each side of the equation will have a
state $(Q\rename{ren}, f\after\rho)$.

%% On each side, each state $Q$ of $P$ is replaced by $f(Q)$ (here each state $Q$
%% will be a pair $(QQ\M{[[}ren\M{]]}, \rho)$, where $QQ$ is a syntactic
%% process, \M{[[}ren\M{]]} is a syntactic renaming, and $\rho$ is an
%% environment). 

Suppose that $P$ contains a transition $(Q,\rho) \trans{e} (Q',\rho')$.  Then
$P\rename{R}^{ren}$ contains a transition $(Q\rename{ren},\rho) \trans{e'}
(Q'\rename{ren},\rho')$ for every $(e,e') \in R$; and so the left-hand side
contains a transition $(Q\rename{ren}, f\after\rho) \trans{f(e')}
(Q'\rename{ren}, {f\after\rho'})$.  And $f(P)$ contains a transition $(Q,
f\after\rho) \trans{f(e)} (Q', f\after\rho')$; so the right-hand side contains
a transition $(Q\rename{ren}, f\after\rho) \trans{e''} (Q'\rename{ren},
{f\after\rho'})$ for every $e''$ such that $(f(e), e'') \in f(R)$.  We show
that these transitions correspond.
%
\begin{itemize}
\item Consider such a transition of the left-hand side labelled with~$f(e')$.
Then $(e,e') \in R$ implies $(f(e),f(e')) \in f(R)$, so the right-hand side
  has a corresponding transition.

\item Consider such a transition of the right-hand side labelled with~$e''$.
  If $(f(e), e'') \in f(R)$ then there is some $(e_1,e_2) \in R$ such that
  $f(e) = f(e_1)$ and $e'' = f(e_2)$.  But then
  Lemma~\ref{lem:renaming-invariant} implies $(e,e_3) \in R$ for some $e_3$
  such that $f(e_3) = f(e_2) = e''$.  Hence the left-hand side has a
  corresponding transition labelled with $f(e_3) = e''$.
\end{itemize}
\end{proof}

%%%%%

Let $\rename{ren} = \rename{ e_1 \leftarrow e_1', \ldots, e_k \leftarrow e_k' \|
  stmts}$ be a renaming.  We show the desired result is satisfied for the
application of~$ren$ to a process~$P$.
%
\begin{calc}
& f(\eval \rho ~ (P~\rename{ren})) \\
= & f( (\eval \rho~P) \rename{ \eval \rho~ren }^{ren} ) \\
= & \com{Lemma \ref{lem:renaming-f}} \\
  &  (f(\eval \rho~P)) \rename{ f(\eval \rho~ren) }^{ren} \\
= & \com{inductive hypothesis; Lemma \ref{lem:eval-renaming}} \\
  &  (\eval (f\after\rho)~P)) \rename{ \eval (f\after\rho)~ren }^{ren} \\
= & \eval (f\after\rho) (P~\rename{ren}).
\end{calc}


%%%%%%%%%%%%%%%%%%%%%%%%%%%%%%%%%%%%%%%%%%%%%%%%%%%%%%%

\paragraph{Binary Parallel operators.}

We now consider the binary parallel operators.  We start with the generalised
parallel operator $\_ \parallel[A] \_$.  Recall that we assume that the
expression defining the synchronisation set is invariant, and hence the value
of that set is $T$-invariant.  

We write $\_ \parallel[A] \_$ for a semantic operator that forms the parallel
composition of LTSs, synchronising on events from~$A$.  Recall that we assume
that the identifiers bound by the two components of the parallel composition
are disjoint; this means that we can restrict a ``global'' environment to the
identifiers of each component, or form the union of the environments of the
two components.  For example, if $(P, \rho_P) \trans{e} (P', \rho_P')$ and
$(Q, \rho_Q) \trans{e} (Q',
\rho_Q')$ with $e \in A$, and $\rho_P$ and~$\rho_Q$ agree on common
identifiers, then $(P \parallel[A] Q, \rho_P \union \rho_Q) \trans{e} (P'
\parallel[A] Q', \rho_P' \union \rho_Q')$: the resulting environments
$\rho_P'$ and $\rho_Q'$ agree on common identifiers, so their union makes
sense.

We show that, assuming $A$ is $T$-invariant:
%
\begin{eqnarray}
\label{eqn:parallel-dist}
f((P,\rho_P) \parallel[A] (Q,\rho_Q)) & = & 
  f(P,\rho_P) \parallel[f(A)] f(Q,\rho_Q).
\end{eqnarray}
%
The tricky case is as follows.  Consider a transition of the right-hand side
corresponding to a synchronisation on event~$e \in f(A)$.  This must
correspond to transitions 
\[
(P,\rho_P) \trans{e_P} (P',\rho_P') \quad\mbox{and}\quad
(Q,\rho_Q) \trans{e_Q} (Q',\rho_Q') \quad
\mbox{with}\quad f(e_P) = f(e_Q) = e.
\]
The subsequent state is 
\[
f(P',\rho_P') \parallel[f(A)] f(Q',\rho_Q') = 
  (P',f(\rho_P')) \parallel[f(A)] (Q',f(\rho_Q')).
\]
If $e_P \ne e_Q$, then consider the first field in which they differ; say $e_p
= c.\vec{u}.v_P.\vec{w_P}$ and $e_Q = c.\vec{u}.v_Q.\vec{w_Q}$ with $v_P \ne
v_Q$ but $f(v_P) = f(v_Q)$.
%
Then by item~\ref{item:sync} of Definition~\ref{defn:data-independent}, one of
the processes must be willing to accept an arbitrary value in this position;
without loss of generality, suppose this is~$P$, so $P$ can also perform $e_P'
= c.\vec{u}.v_Q.\vec{w_P}$.  If the transition binds $x$ to this field in
$P$'s environment, then this corresponds to the transition
$(P,\rho_P) \trans{e_P'} (P,\rho_P'')$ where $\rho_P''
= \rho_P' \oplus \set{x \mapsto v_Q}$.  Note that $f(e_P') = e$ and
$f(\rho_P'') = f(\rho_P')$.
%
Continuing in this way, we obtain transitions
\[
\begin{array}{c}
(P,\rho_P) \trans{e'} (P',\hat\rho_P) \quad\mbox{and}\quad
(Q,\rho_Q) \trans{e'} (Q',\hat\rho_Q) \\
\mbox{with}\quad f(e')  = e, \quad 
f(\hat\rho_P) = f(\rho_P'), \quad f(\hat\rho_Q) = f(\rho_Q').
\end{array}
\]
Also, note that $f(e') = e \in f(A)$, so $e' \in A$, by
Lemma~\ref{lem:T-invariant-inclusion}, since $A$ is $T$-invariant.
%
Hence, in the left-hand side of (\ref{eqn:parallel-dist}), the two
components can synchronise on~$e'$, giving a transition on $f(e') = e$, and
producing state 
\[
f( (P',\hat\rho_P) \parallel[A] (Q',\hat\rho_Q) ) .
\]
This matches the original transition of the right-hand side.

Other cases ---that every synchronisation of the left-hand side is matched by
a synchronisation of the right-hand side, and that non-synchronisation
transitions match--- can be proved in a similar but more straightforward way.

The proof then proceeds as for other binary operators.  

The cases of the alphabetised parallel ($\_ \parallel[A][B] \_$) and
interleaving ($\interleave$) operators are similar; recall that for the
former, it is assumed that the expressions that define the alphabets are
invariant, and hence the values of those alphabets are $T$-invariant. 

Link parallel \framebox{?}


%%%%%%%%%%%%%%%%%%%%%%%%%%%%%%%%%%%%%%%%%%%%%%%%%%%%%%%%%%%%%%%%%

\paragraph{Replicated operators.}

.......

Below we write $\Extchoice$ for a semantic operator: ``$\Extchoice S$''
represents an external choice over the set of processes in~$S$, each of which
is an LTS\@.  We show that
%
\begin{eqnarray*}
f(\Extchoice S) & = & \Extchoice \set{f(P) \| P \in S}.
\end{eqnarray*}
%
For every $P \in S$, and for every transition $P \trans{e} Q$, the initial
state of each of the above processes has a transition labelled with $f(e)$ to
$f(Q)$.

We then have the following (for syntactic expressions~$S$ and~$P$):
\begin{calc}
& f(\eval \rho~ (\M{**[]} stmts \spot P)) \\
= & f(\Extchoice \set{\eval \rho'~ P \| \rho' \in \evalStmtsSet \rho~stmts}) \\
= & \com{above result} \\
  & \Extchoice \set{f(\eval \rho' ~P) \| \rho' \in \evalStmtsSet \rho~stmts} \\
= & \com{inductive hypothesis for~$P$} \\
  & \Extchoice \set{\eval (f(\rho'))~P \| 
      \rho' \in \evalStmtsSet \rho~stmts} \\
= & \com{letting $\rho'' = f(\rho')$} \\
  & \Extchoice \set{\eval \rho''~P \| 
      \rho'' \in f(\evalStmtsSet \rho~stmts)} \\
= & \com{inductive hypothesis, part \ref{item:evalStmtsSet}} \\
  & \Extchoice \set{\eval \rho''~P \| 
      \rho'' \in \evalStmtsSet (f(\rho))~stmts} \\
= & \eval (f(\rho'))~ (\M{**[]} stmts \spot P).
\end{calc}



The case for a replicated nondeterministic choice is similar.  

Other replicated operators \framebox{**}.  Indexing set independent of $t$. 


Consider now a replicated alphabetised parallel composition $\M{\|\|} ~ stmts
~ \M{@ [} A \M{]}~ P$.  We write $\Parallel$ for a semantic parallel
composition operator, which takes a multiset of (process, alphabet) pairs as an
argument.   We have that
\begin{eqnarray}
\label{eqn:indexed-parallel}
f(\Parallel S) & = & \Parallel \set{(f(P), f(A)) \| (P,a) \in S}
\end{eqnarray}
which follows from the corresponding equation $f(P \parallel[A][B] Q) =
f(P) \parallel[f(A)][f(B)] f(Q)$ for binary parallel composition, and the fact
that the indexed parallel composition can be rewritten using binary parallel
composition (it is important here that we are using a \emph{multiset}, so the
two sides of~(\ref{eqn:indexed-parallel}) have the same number of processes.


Recall that the indexing statements~$stmt$ are independent of~$t$.  Hence
$\evalStmts~\rho~stmts$ and $\evalStmts~(f(\rho))~stmts =
f(\evalStmts \rho~stmts)$ have the same cardinality (recall that each is a
multiset of environments).  \framebox{???????}

\begin{calc}
& f(\eval \rho (\M{\|\|} ~ stmts~ \M{@ [} A \M{]}~ P)) \\
= & f( \Parallel \set{ (\eval \rho' ~ P, \eval \rho'~ A) \| 
       \rho' \in \evalStmts \rho~stmts} ) \\
= & \com{equation (\ref{eqn:indexed-parallel})} \\
& \Parallel \set{ (f(\eval \rho'~P), f(\eval \rho'~A)) \| 
     \rho' \in \evalStmts \rho~stmts}  \\
= & \com{inductive hypothesis} \\
& \Parallel \set{ (\eval (f(\rho'))~P, \eval (f(\rho'))~A) \| 
    \rho' \in \evalStmts \rho~stmts}  \\
= & \com{letting $\rho'' = f(\rho')$} \\
& \Parallel \set{ (\eval \rho''~P, \eval \rho''~A) \| 
    \rho'' \in f(\evalStmts \rho~stmts)}  \\
= & \com{inductive hypothesis, part~\ref{item:evalStmts}} \\
& \Parallel \set{ (\eval \rho''~P, \eval \rho''~A) \| 
    \rho'' \in \evalStmts (f(\rho))~stmts}  \\
= & \eval (f(\rho))~(\M{\|\|} ~ stmts~ \M{@ [} A \M{]}~ P)
\end{calc}

\framebox{Other} replicated parallel operators, replicated sequential
composition.  

