\subsubsection{Prefixing}  
\label{sec:prefixing}

Consider the prefixing expression $e_c ~ f_1 \ldots f_n \then P$, where $e_c$
is an expression that should evaluate to a possibly incomplete event (i.e.~a
channel name with zero or more fields supplied), $n \ge 0$, and each~$f_i$ is
a field of one of the following forms:
  %
  \begin{itemize}
  \item $? pat$, where $pat$ is a pattern: this offers an external choice
    between all values matching $pat$ and consistent with the channel type. 

  \item $?pat : E$, where $pat$ is a pattern, and $E$ should evaluate to a
    set: this is like the previous case, but restricts values to elements
    of~$E$. 

  \item $!e$, where $e$ is an expression: this matches only the value of~$e$. 

  \item $\$ pat$: this performs an internal choice between all values matching
    $pat$ and consistent with the channel type.
 
  \item $\$ pat : E$: this is like the previous case, but restricts values to
    elements of~$E$.
  \end{itemize}
  %
  The semantics of pattern matching by fields depends upon the field's
  location: a variable pattern (e.g.~$?x$) matches a \emph{single} complete
  value, \emph{except} in the final field where it matches the whole of the
  rest of the event.  For example, the prefix construct $c?x?y$ matches the
  event $c.1.2.3$, binding $x$ to~$1$, and $y$ to~$2.3$.

  If there are one or more \$-fields, then the process has initial
  $\tau$-transitions to resolve the internal choices; each \$-field is
  replaced by a field $!x$, where $x$ is a fresh variable bound in the
  environment to the value chosen; also the environment is updated
  corresponding to any other variables bound in the field.  The definition of
  \CSPm\ requires that every \$-field precedes every !- or ?-field.

  Recall that we assume that distinct variables have distinct names.  In
  particular, this means that the initial expression~$e_c$ will use no
  variables bound by a \$-field; this is required, since we will
  evaluate~$e_c$ \emph{after} the \$-fields. 

  We perform a case analysis on whether or not the prefixing construct
  contains any \$-fields.

  %%%%%%%%%%

\paragraph{Case of no \$-fields.} 

Consider first a prefixing construct of the form ${e_c~{\it fields} \then P}$
that contains no $\$$-fields.  We define the semantics of prefixing using a
function
\[
  \evalField :: 
    \begin{align}
    Env \fun Bool \fun Value \fun Field \fun  \power (Value \cross Env).
    \end{align}
\]
If $c.v$ is an incomplete event on channel~$c$, and $field$ is a field, then
for a correct usage, $\evalField \rho~last~(c.v)~field$ gives a set of
$(w,\rho')$ pairs; for each such pair, $c.v.w$ is an extension of~$c.v$
compatible with~$c$, corresponding to adding this field, and $\rho'$ is the
updated environment caused by binding any variables in patterns in~$field$.
Compatibility will mean that if $\rho(c) = \channel S$ then $v.w$ is a prefix
of some element of~$S$; we denote this $v.w \le \rho(c)$.  However, an error
will occur if a $!e$ or $?pat:E$~field gives a value not compatible with the
channel.  The argument~$last$ of $\evalField$ indicates whether this is the
last field in the prefixing construct, which affects the semantics of pattern
matching, as explained above.

%% Recall that we assumed that the script is error-free.  This implies 

We prove the following
\begin{equation}\label{eqn:evalField}
f (\evalField \rho~last~(c.v)~field)  = 
    \evalField (f(\rho))~last~(c.f(v))~field .
\end{equation}
%
We perform a case analysis on~$field$.
%
\begin{itemize}
\item Case $!e$. 
  \begin{calc}
  &  f (\evalField \rho~last~(c.v)~(!e)) \\
  = & f(\Let w = \eval \rho~e \In 
        \If v.w \le \rho(c) \Then \set{(w, \rho)} \Else error) \\
  = & \com{Lemma~\ref{lem:invariant-prefix}} \\
  & \begin{align}
    \Let w = \eval \rho~e \In \\
    \If f(v).f(w) \le f(\rho(c)) \Then \set{(f(w), f(\rho))} \Else f(error)
    \end{align} \\
  = & \com{letting $w' = f(w)$} \\
  & \begin{align}
    \Let w' = f(\eval \rho~e) \In \\
    \If f(v).w' \le (f(\rho))(c) \Then \set{w', f(\rho))} \Else error
    \end{align} \\
  = & \com{inductive hypothesis applied to~$e$} \\
  & \begin{align}
    \Let w' = \eval (f(\rho))~e \In \\
    \If f(v).w' \le (f(\rho))(c) \Then \set{w', f(\rho))} \Else error
    \end{align} \\
  = & \evalField (f(\rho))~last~(c.f(v))~(!e).
  \end{calc}

\item Case $?pat$.  We write $arity(pat)$ for the number of fields matched by
  $pat$ in the case that this is not the final field.  Similarly, we write
  $arity(w)$ for the number of complete values within~$w$.  Below, $W$ is the
  set of candidate values that $pat$ could be matched against: it contains all
  values $w$, with the appropriate arity, that can be appended to~$v$ to give
  a prefix of an element of~$\rho(c)$.
%
  \begin{calc}
  & f (\evalField \rho~last~(c.v)~(?pat)) \\
  = & f(
    \begin{align}
    \Let W = 
      \begin{align}
      \If last \Then \set{w \mid v.w \in \rho(c)} \\
      \Else \set{w \mid v.w \prefix \rho(c), arity(pat) = arity(w)} 
      \end{align} \\
    \In 
    \set{(w, \bind \rho~pat~w) \mid w \in W, \matches \rho~pat~w})
    \end{align} \\
  = & 
    \begin{align}
    \Let W = 
      \begin{align}
      \If last \Then \set{w \mid v.w \in \rho(c)} \\
      \Else \set{w \mid v.w \prefix \rho(c), arity(pat) = arity(w)} 
      \end{align} \\
    \In 
    \set{(f(w), f(\bind \rho~pat~w)) \mid  w \in W, \matches \rho~pat~w}
    \end{align} \\
  = & \com{parts \ref{item:matches} and \ref{item:bind}} \\
        %% letting $w' = f(w)$,\, $W' = f(W)$; Corollary \ref{cor:channel-types}} \\
  & \begin{align}
    \Let W = 
      \begin{align}
      \If last \Then \set{w \mid v.w \in \rho(c)} \\
      \Else \set{w \mid v.w \prefix \rho(c), arity(pat) = arity(w)} 
      \end{align} \\
    \In 
    \set{(f(w), \bind (f(\rho))~pat~(f(w))) \mid \\
    \quad\qquad   w \in W, \matches (f(\rho))~pat~(f(w))}
    \end{align} 
  \\
  = & \com{letting $w' = f(w)$,\, $W' = f(W)$} \\
  & \begin{align}
    \Let W' = 
      \begin{align}
      \If last \Then \set{f(w) \mid v.w \in \rho(c)} \\
      \Else \set{f(w) \mid 
         \begin{align}
         v.w \prefix \rho(c), 
         arity(pat) = arity(w)} 
         \end{align}
      \end{align} \\
    \In 
    \set{(w', \bind (f(\rho))~pat~w') \mid 
       w' \in W', \matches (f(\rho))~pat~w'}
    \end{align} 
  \\
  = & \com{Lemma~\ref{lem:invariant-prefix}} \\
  & \begin{align}
    \Let W' = 
      \begin{align}
      \If last \Then \set{f(w) \mid f(v).f(w) \in (f(\rho))(c)} \\
      \Else \set{f(w) \mid 
         \begin{align}
         f(v).f(w) \prefix (f(\rho))(c), \\
         arity(pat) = arity(w)} 
         \end{align}
      \end{align} \\
    \In 
    \set{(w', \bind (f(\rho))~pat~w') \mid 
      w' \in W', \matches (f(\rho))~pat~w'}
    \end{align} 
  \\
  = & \com{letting $w' = f(w)$; $arity(w) = arity(f(w))$} \\
  & \begin{align}
    \Let W' = 
      \begin{align}
      \If last \Then \set{w' \mid f(v).w' \in (f(\rho))(c)} \\
      \Else \set{w' \mid 
         \begin{align}
         f(v).w' \prefix (f(\rho))(c), 
         arity(pat) = arity(w')} 
         \end{align}
      \end{align} \\
    \In 
    \set{(w', \bind (f(\rho))~pat~w') \mid 
       w' \in W', \matches (f(\rho))~pat~w'}
    \end{align} 
  \\
  = & \evalField (f(\rho))~last~(c.f(v))~(?pat) .
  \end{calc}

%% *** Need that $\rho(c)$ is symmetric: if $v \in \rho(c)$ and $f(v) = f(v')$
%% then $v' \in \rho(c)$.  

\item Case $?pat:E$.  This is very similar to the previous case.
  Note that $f(\eval \rho~ E) = \eval(f(\rho))~ E$, by the inductive
  hypothesis.  Recall that we do not assume $E$ is invariant.

  If, for some $w \in\eval \rho~ E$,\, $v.w$ is not a prefix of $\rho(c)$,
  then the left-hand side evaluates to $error$.  But in this case,
  $f(w) \in \eval (f(\rho))~E$, and $f(v).f(w)$ is not a prefix of
  $(f(\rho))(c)$ by Corollary~\ref{cor:channel-types}, so the right-hand
  side also evaluates to $error$.  
% 
  Conversely, if for some $w' \in \eval (f(\rho))~ E$,\, $f(v).w'$ is
  not a prefix of $(f(\rho))(c)$, then the right-hand side evaluates to
  $error$.  But then $w' = f(w)$ for some $w \in \eval \rho~E$, and $v.w$ is
  not a prefix of $\rho(c)$, so the left-hand side also evaluates to~$error$. 

  %% An error arises if any element of $\set{v.w \| w \in \eval \rho~ E}$ is not
  %% a prefix of a member of $\rho(c)$; in which case, an element of $\set{v.w'
  %%   \| w' \in \eval (f(\rho))~ E}$ is not a prefix of a member of
  %% $(f(\rho))(c)$, and vice versa, by Corollary~\ref{cor:channel-types}; both
  %% sides then evaluate to $error$.

  Otherwise, in the main set comprehensions, $w$ ranges over $W \inter \eval
  \rho~ E$.  Then $w'$ ranges over $W' \inter f(\eval \rho~ E) = W' \inter
  \eval(f(\rho))~ E$.
\end{itemize} %% End of case analysis on field

%%%%%%%%%% evalFields

We now define a function that accumulate over the fields to get the
resulting events and environments (but handling errors appropriately).
\[
\begin{align}
\evalFields :: 
  \begin{align}
  Env \fun  Value \fun Field^* \fun \\
  \qquad (\power (Value \cross Env) \union \set{error})
  \end{align} \\
\evalFields \rho~(c.v)~\seq{} = \set{(c.v,\rho)} \\
% \evalFields \rho~(c.v)~\seq{f}  =  \evalField \rho~true~(c.v)~f \\
\evalFields \rho~(c.v)~(\seq{field} \cat fs)  = \\
\qquad
  \begin{align}
  \Let S = \evalField \rho~(fs=\seq{})~(c.v)~field \In \\
  \If S = \error \Then \error \\
  \Else \Union \set{ \evalFields \rho'~(c.w)~fs \mid  (c.w, \rho') \in S }.
  \end{align}
\end{align}
\]
(The term $fs=\seq{}$ tests whether $f$ is the last field.)
We can then show 
\begin{eqnarray*}%\label{eqn:evalFields}
f(\evalFields \rho~(c.v)~fs) & = & 
  \evalFields (f(\rho))~(c.f(v))~fs ,
\end{eqnarray*}
%
by an induction on $fs$ (the proof is identical to the corresponding proof
in~\cite{symmetry-reduction}).

%%%%%%%%%%

We now consider the process $e_c~ fs \then P$.  Let $c.v = \eval \rho~e_c$.
Then $\eval (f(\rho))~e_c = c.f(v)$, by the inductive hypothesis.  Let
%
\begin{eqnarray*}
L & = & \eval \rho~ (e_c~ fs \then P), \\
S & = & \evalFields \rho~(c.v)~fs, \\
L' & = &  \eval (f(\rho))~ (e_c~fs \then P), \\
S' & = & \evalFields (f(\rho))~(c.f(v))~fs.
\end{eqnarray*}%
%
The above result shows $f(S) = S'$.  We need to show $f(L) = L'$.

If $S = error$ then $S' = error$, and vice versa; but then each of $L$, $f(L)$
and $L'$ equal error, so $f(L) = L'$, as required. 

Otherwise, $L$ is an LTS as follows.
%
\begin{itemize}
\item The initial state is $(e_c~ fs \then P, \rho)$.

\item For each $(a, \rho') \in S$, there is a transition labelled with $a$
  from the initial state.  %  to $(P, \rho')$.

\item For each $(a, \rho') \in S$, the sub-LTS following the transition of
  the previous item equals $\eval \rho'~ P$.
\end{itemize}
%
Then $f(L)$ is an  LTS as follows.
%
\begin{itemize}
\item The initial state is $(e_c~fs \then P, f(\rho))$.
  This is the same as the initial state of~$L'$.

\item From the initial state, for each $(a, \rho') \in S$, there is a
transition labelled with $f(a)$.  Letting $a' = f(a)$ and $\rho'' = f(\rho')$,
  this is equivalent to saying that for every $(a',\rho'') \in f(S) = S'$,
  there is a transition labelled with~$a'$ from the initial state.  This is
  the same as the initial transitions of~$L'$.

\item For each $(a, \rho') \in S$, the sub-LTS after the transition of the
  previous item equals $f(\eval \rho'~ P)$.  But this equals $\eval
  (f(\rho'))~P$, by the inductive hypothesis.  Letting $a' = f(a)$ and
  $\rho'' = f(\rho')$, this is equivalent to saying that for every
  $(a',\rho'') \in f(S) = S'$, the sub-LTS after this transition equals $\eval
  \rho''~P$.  This is the same as for~$L'$.
\end{itemize} 
%
Hence $f(L) = L'$, as required. 

%%%%%%%%%%

\paragraph{Case of at least one \$-field.}

Now suppose that the prefixing construct contains at least one $\$$-field.
We define the semantics using a function 
\[
\evalDollarField ::
  Env \fun Bool \fun Value \fun Field \fun  \power (Value \cross Env).
\]
If $c.v$ is an incomplete event on channel~$c$, and $field$ is a $\$$-field,
then, for a correct usage, $\evalDollarField \rho~last~(c.v)~field$ gives a
set of $(w,\rho')$ pairs; each~$w$ is a value that could be communicated in
the place of~$field$, compatible with~$c.v$, and $\rho'$ is the updated
environment caused by binding any variables in patterns in~$field$.
Compatibility again means that if $\rho(c) = \channel S$ then $v.w$ is a
prefix of some element of~$S$, denoted $v.w \le \rho(c)$.  The argument~$last$
of $\evalField$ again indicates whether this is the final field in the
prefixing construct.

We can prove the following, for $field$ a \$-field:
\[ % begin{equation}\label{eqn:evalDollarField}
  f (\evalDollarField \rho~last~(c.v)~field)  = 
    \evalDollarField (f(\rho))~last~(c.f(v))~field .
\] % end{equation}
The proof is almost identical to the cases for ?-fields, and so is omitted. 

We now define a function that, given an environment~$\rho$, an incomplete
event~$c.v$, and a list of fields~$fs$, gives a set of pairs $(fs', \rho')$
such that $fs'$ is the result of substituting each \$-field of $fs$ with a
field $!x$ where $x$ is a fresh variable, and $\rho'$ is the environment
resulting from binding~$x$ to a value~$w$ that could instantiate this field,
and also binding any variables in the \$-fields with the corresponding values.
The definition makes use of the fact that every \$-field must precede every !-
or ?-field.
\[
\begin{align}
\evalDollarFields :: 
  Env \fun  Value \fun Field^* \fun  \power (Field^* \cross Env) ,
\\
\evalDollarFields \rho~(c.v)~\seq{} = \set{(\seq{},\rho)} ,
\\
\evalDollarFields \rho~(c.v)~(\seq{field} \cat fs)  = \\
\quad
  \begin{align}
  \If \mbox{$field$ is a \$-field} \\
  \Then 
    \begin{align}
    \set{ (\seq{!x} \cat fs', \rho'') \mid 
    (w,\rho') \in \evalDollarField \rho~(fs=\seq{})~(c.v)~field, \\
    \qquad\quad (fs', \rho'') \in
       \evalDollarFields (\rho' \union \set{x \mapsto w})~ (c.v.w)~fs } \\
    \mbox{where $x$ is  fresh}
    \end{align} \\
  \Else \set{ (\seq{field} \cat fs, \rho) } .
  \end{align}
\end{align}
\]

We can then  show
\begin{eqnarray*} % \label{eqn:evalDollarFields}
f (\evalDollarFields \rho~(c.v)~fs) & = & 
  \evalDollarFields (f(\rho))~(c.f(v))~fs ,
\end{eqnarray*}
%
by an induction on $fs$ (again, identical to the corresponding proof
in~\cite{symmetry-reduction}).

We now consider the process $e_c~ fs \then P$; recall we are assuming there is
at least one \$-field.  Let $c.v = \eval \rho~e_c$.  So $\eval
(f(\rho))~e_c = f (\eval \rho~e_c) = c.f(v)$, using the inductive
hypothesis.  So let
\begin{eqnarray*}
L & = & \eval \rho ~ (e_c~fs \then P), \\
S & = & \evalDollarFields \rho ~(c.v)~ fs, \\
L' & = & \eval (f(\rho))~ (e_c~fs \then P), \\
S' & = & \evalDollarFields (f(\rho)) ~(c.f(v)) ~fs .
\end{eqnarray*}%
%
The above result shows $f(S) = S'$.  We need to show $f(L) = L'$.

If $S = \set{}$ then $S' = \set{}$, and vice versa.   But then each of $L$,
$f(L)$ and~$L'$ equals $error$, so $f(L) = L'$, as required.  The case where
$S$ and~$S'$ equal $error$ is similar.

Otherwise, $L$ is an augmented LTS as follows.
\begin{itemize}
\item The initial state is $(e_c~fs \then P, \rho)$.

\item From the initial state, for each $(fs',\rho') \in S$, there is a
  $\tau$-transition to $(e_c~fs' \then P, \rho')$.

\item For each $(fs',\rho') \in S$, the sub-LTS rooted at
  $(e_c~fs' \then P, \rho')$ equals $\eval~\rho'~(e_c~fs' \then P)$.
\end{itemize}

%
Then $f(L)$ is an augmented LTS as follows.
\begin{itemize}
\item The initial state is
  $(e_c~fs \then P, f(\rho))$.  This is the
  same as the initial state of~$L'$.

\item From this initial state, for each $(fs',\rho') \in S$, there is a
  $\tau$-transition to
  \(
  (e_c~fs' \then P, f(\rho')). 
  \)
  %since the only constants from~$\T$ must appear in~$fs'$ and~$v$.  
  Letting $\rho'' = f(\rho')$, this is equivalent to
  saying that for each $(fs',\rho'') \in f(S) = S'$, there is a
  $\tau$-transition to $(e_c~fs' \then P,\linebreak[1] \rho'')$.
  This is the same as the initial transitions of~$L'$.

\item For each $(fs',\rho') \in S$, the sub-LTS rooted at
  $(e_c~fs' \then  P,\linebreak[2]
    {f(\rho')})$ is 
  equal to 
  \( f(\eval~\rho'~(e_c~fs' \then P)) . \)
  But this equals $\eval~{(f(\rho'))} ~ (e_c~fs' \then P)$
  using the result for the case that there are no \$-fields.
  % (equation~(\ref{eqn:prefix-no-dollar})).  
  Letting 
  $\rho'' = f(\rho')$, this is equivalent to saying that for each
  $(fs',\rho'') \in f(S) = S'$, the sub-LTS rooted at
  $(e_c~fs' \then P, \linebreak[2] \rho'')$ is equal to
  $\eval~\rho''~(e_c~fs' \then P)$.  This is the same as for~$L'$.
\end{itemize}
%
Hence $f(L) = L'$, as required.



