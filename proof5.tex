\subsection{Statements}

We start with sequence-generating statements.  
We show that 
\begin{eqnarray*}
f(\evalStmt \rho ~stmt) & = & \evalStmt (f(\rho))~stmt,
\end{eqnarray*}
%
via a case analysis.
%
\begin{itemize}
\item Case $pat \lArrow e$.
%
\begin{calc}
& f(\evalStmt \rho ~(pat \lArrow e)) \\
= & f(\seq{ \bind \rho ~pat~v \| v \leftarrow \eval \rho~e,\, 
   \matches \rho~pat~v}) \\
= & \seq{ f(\bind \rho ~pat~v) \| v \leftarrow \eval \rho~e,\, 
   \matches \rho~pat~v} \\
= & \com{clauses~\ref{item:matches} and~\ref{item:bind}} \\
& \seq{ \bind (f(\rho)) ~pat~(f(v)) \| v \leftarrow \eval \rho~e,\,
    \matches (f(\rho))~pat~(f(v)) } \\
= & \com{letting $w = f(v)$} \\
& \seq{ \bind (f(\rho)) ~pat~w \| w \leftarrow f(\eval \rho~e),\,
    \matches (f(\rho))~pat~w} \\
= & \com{clause \ref{item:eval}} \\
& \seq{ \bind (f(\rho)) ~pat~w \| w \leftarrow \eval (f(\rho))~e,\,
    \matches (f(\rho))~pat~w } \\
= & \evalStmt (f(\rho)) ~(pat \lArrow e).
\end{calc}

\item Case a predicate $pred$.
%
\begin{calc}
& f(\evalStmt \rho ~pred) \\
= & f(\If \eval \rho~pred \Then \seq{\rho} \Else \seq{}) \\
= & \com{$\eval \rho~pred = f(\eval \rho~pred)$, since it is independent
  of~\CSPM{T}} \\
 & \If f(\eval \rho~pred) \Then f(\seq{\rho}) \Else f(\seq{}) \\
= & \com{clause \ref{item:eval}} \\
& \If \eval (f(\rho)~pred \Then \seq{f(\rho)} \Else \seq{} \\
= & \evalStmt (f(\rho)) ~ pred.
\end{calc}
\end{itemize}

%%%%%%%%%%

We now consider a list $stmts$ of statements.  We prove
\begin{eqnarray*}
f (\evalStmts \rho ~ stmts) & = & \evalStmts (f(\rho))~stmts,
\end{eqnarray*}
by induction on the length of $stmts$.  The base case is trivial.  For the
inductive case:
\begin{calc}
& f (\evalStmts \rho ~ (\seq{stmt} \cat stmts)) \\
= & f( \seq{ \rho_2 \| \rho_1 \leftarrow \evalStmt \rho~stmt, \, 
         \rho_2 \leftarrow \evalStmts \rho_1~stmts }) \\
= & \seq{ f(\rho_2) \| \rho_1 \leftarrow \evalStmt \rho~stmt, \, 
         \rho_2 \leftarrow \evalStmts \rho_1~stmts } \\
=  & \com{letting $\rho_2' = f(\rho_2)$} \\
& \seq{ \rho_2' \| \rho_1 \leftarrow \evalStmt \rho~stmt, \, 
         \rho_2' \leftarrow f(\evalStmts \rho_1~stmts) } \\
= & \com{inductive hypothesis} \\
& \seq{ \rho_2' \| \rho_1 \leftarrow \evalStmt \rho~stmt, \, 
         \rho_2' \leftarrow \evalStmts (f(\rho_1)) ~stmts } \\
= & \com{letting $\rho_1' = f(\rho_1)$} \\
& \seq{ \rho_2' \| \rho_1' \leftarrow f(\evalStmt \rho~stmt), \, 
         \rho_2' \leftarrow \evalStmts \rho_1' ~stmts } \\
= & \com{earlier result} \\
& \seq{ \rho_2' \| \rho_1' \leftarrow \evalStmt (f(\rho))~stmt, \, 
         \rho_2' \leftarrow \evalStmts \rho_1' ~stmts } \\
= & \evalStmts (f(\rho)) (\seq{stmt} \cat stmts).
\end{calc}


%%%%%%%%%%%%%%%%%%%%%%%%%%%%%%%%%%%%%%%%%%%%%%%%%%%%%%%


The proofs for set-generating statements are essentially identical. 


