%% \subsubsection{Processes}
%% \label{sec:processes}

%We now consider process expressions.

\subsubsection{Basic processes}  
\label{sec:basic-procs}

The cases for $STOP$, $SKIP$, and $DIV$ are all trivial.

The process-creating functions |RUN| and |CHAOS| satisfy the conditions of
Lemma~\ref{lem:built-in-functions}, and so can be dealt with as in
Section~\ref{sssec:function-decl}. 

%%%%%%%%%%%%%%%%%%%%%%%%%%%%%%%%%%%%%%%%%%%%%%%%%%%%%%%

\paragraph{Prefixing.}  Consider $e_c ~ f_1 \ldots f_n \then P$, where $e_c$
is an expression that should evaluate to a possibly incomplete event (i.e.~a
channel name with zero or more fields supplied), $n \ge 0$, and each~$f_i$ is
a field of one of the following forms:
  %
  \begin{itemize}
  \item $? pat$, where $pat$ is a pattern; this offers an external choice
    between all values matching $pat$ and consistent with the channel type. 

  \item $?pat : E$, where $pat$ is a pattern, and $E$ should evaluate to a
    set; this is like the previous case, but restricts values to elements
    of~$E$. 

  \item $!e$, where $e$ is an expression; this matches only the value of~$e$. 

  \item $\$ pat$; this performs an internal choice between all values matching
    $pat$ and consistent with the channel type.
 
  \item $\$ pat : E$; this is like the previous case, but restricts values to
    elements of~$E$.
  \end{itemize}
  %
  The semantics of pattern matching by fields depends upon the field's
  location: a variable pattern (e.g.~$?x$) matches a \emph{single} complete
  value, \emph{except} in the final field where it matches the whole of the
  rest of the event.  For example, the prefix construct $c?x?y$ matches the
  event $c.1.2.3$, binding $x$ to~$1$, and $y$ to~$2.3$.

  If there are one or more \$-fields, then the process has initial
  $\tau$-transitions to resolve the internal choices; each such field is
  replaced by a field $!x$, where $x$ is a fresh variable bound in the
  environment to the value chosen; also the environment is updated
  corresponding to any other variables bound in the field.  The definition of
  \CSPm\ requires that every \$-field precedes every !- or ?-field.

  Recall that we assume that distinct variables have distinct names.  In
  particular, this means that the initial expression~$e_c$ will use no
  variables bound by a \$-field; this is required, since we will
  evaluate~$e_c$ \emph{after} the \$-fields. 

  We perform a case analysis on whether or not the prefixing construct
  contains any \$-fields.

  %%%%%%%%%%

\subparagraph{Case of no \$-fields.} 

Consider first a prefixing construct of the form ${e_c~{\it fields} \then P}$
that contains no $\$$-fields.  We define the semantics of prefixing using a
function
\[
  \evalField :: 
    \begin{align}
    Env \fun Bool \fun Value \fun Field \fun  \power (Value \cross Env).
    \end{align}
\]
If $c.v$ is an incomplete event on channel~$c$, and $field$ is a field, then
for a correct usage, $\evalField \rho~last~(c.v)~field$ gives a set of
$(w,\rho')$ pairs; for each such pair, $c.v.w$ is an extension of~$c.v$
compatible with~$c$, corresponding to adding this field, and $\rho'$ is the
updated environment caused by binding any variables in patterns in~$field$.
Compatibility will mean that if $\rho(c) = \channel S$ then $v.w$ is a prefix
of some element of~$S$; we denote this $v.w \le \rho(c)$.  However, an error
will occur if a $!e$ or $?pat:E$~field gives a value not compatible with the
channel.  The argument~$last$ of $\evalField$ indicates whether this is the
last field in the prefixing construct, which affects the semantics of pattern
matching, as explained above.

%% Recall that we assumed that the script is error-free.  This implies 

We prove the following
\begin{equation}\label{eqn:evalField}
f (\evalField \rho~last~(c.v)~field)  = 
    \evalField (f(\rho))~last~(c.f(v))~field .
\end{equation}
%
We perform a case analysis on~$field$.
%
\begin{itemize}
\item Case $!e$. 
  \begin{calc}
  &  f (\evalField \rho~last~(c.v)~(!e)) \\
  = & f(\Let w = \eval \rho~e \In 
        \If v.w \le \rho(c) \Then \set{(w, \rho)} \Else error) \\
  = & \com{Corollary \ref{cor:channel-types}} \\
  & \begin{align}
    \Let w = \eval \rho~e \In \\
    \If f(v).f(w) \le f(\rho(c)) \Then \set{(f(w), f(\rho))} \Else f(error)
    \end{align} \\
  = & \com{letting $w' = f(w)$} \\
  & \begin{align}
    \Let w' = f(\eval \rho~e) \In \\
    \If f(v).w' \le (f(\rho))(c) \Then \set{w', f(\rho))} \Else error
    \end{align} \\
  = & \com{inductive hypothesis applied to~$e$} \\
  & \begin{align}
    \Let w' = \eval (f(\rho))~e \In \\
    \If f(v).w' \le (f(\rho))(c) \Then \set{w', f(\rho))} \Else error
    \end{align} \\
  = & \evalField (f(\rho))~last~(c.f(v))~(!e).
  \end{calc}

\item Case $?pat$.  We write $arity(pat)$ for the number of fields matched by
  $pat$ in the case that this is not the final field.  Similarly, we write
  $arity(w)$ for the number of complete values within~$w$.  Below, $W$ is the
  set of candidate values that $pat$ could be matched against: it contains all
  values $w$, with the appropriate arity, that can be appended to~$v$ to give
  a prefix of an element of~$\rho(c)$.
%
  \begin{calc}
  & f (\evalField \rho~last~(c.v)~(?pat)) \\
  = & f(
    \begin{align}
    \Let W = 
      \begin{align}
      \If last \Then \set{w \mid v.w \in \rho(c)} \\
      \Else \set{w \mid v.w.u \in \rho(c), arity(pat) = arity(w)} 
      \end{align} \\
    \In 
    \set{(w, \bind \rho~pat~w) \mid w \in W, \matches \rho~pat~w})
    \end{align} \\
  = & 
    \begin{align}
    \Let W = 
      \begin{align}
      \If last \Then \set{w \mid v.w \in \rho(c)} \\
      \Else \set{w \mid v.w.u \in \rho(c), arity(pat) = arity(w)} 
      \end{align} \\
    \In 
    \set{(f(w), f(\bind \rho~pat~w)) \mid  w \in W, \matches \rho~pat~w}
    \end{align} \\
  = & \com{parts \ref{item:matches} and \ref{item:bind}} \\
        %% letting $w' = f(w)$,\, $W' = f(W)$; Corollary \ref{cor:channel-types}} \\
  & \begin{align}
    \Let W = 
      \begin{align}
      \If last \Then \set{w \mid v.w \in \rho(c)} \\
      \Else \set{w \mid v.w.u \in \rho(c), arity(pat) = arity(w)} 
      \end{align} \\
    \In 
    \set{(f(w), \bind (f(\rho))~pat~(f(w))) \mid \\
    \quad\qquad   w \in W, \matches (f(\rho))~pat~(f(w))}
    \end{align} 
  \\
  = & \com{letting $w' = f(w)$,\, $W' = f(W)$} \\
  & \begin{align}
    \Let W' = 
      \begin{align}
      \If last \Then \set{f(w) \mid v.w \in \rho(c)} \\
      \Else \set{f(w) \mid 
         \begin{align}
         v.w.u \in \rho(c), 
         arity(pat) = arity(w)} 
         \end{align}
      \end{align} \\
    \In 
    \set{(w', \bind (f(\rho))~pat~w') \mid 
       w' \in W', \matches (f(\rho))~pat~w'}
    \end{align} 
  \\
  = & \com{Corollary \ref{cor:channel-types}} \\
  & \begin{align}
    \Let W' = 
      \begin{align}
      \If last \Then \set{f(w) \mid f(v).f(w) \in (f(\rho))(c)} \\
      \Else \set{f(w) \mid 
         \begin{align}
         f(v).f(w).f(u) \in (f(\rho))(c), \\
         arity(pat) = arity(w)} 
         \end{align}
      \end{align} \\
    \In 
    \set{(w', \bind (f(\rho))~pat~w') \mid 
      w' \in W', \matches (f(\rho))~pat~w'}
    \end{align} 
  \\
  = & \com{letting $w' = f(w)$,\, $u' = f(u)$; % \\
          $arity(w) = arity(f(w))$} \\
  & \begin{align}
    \Let W' = 
      \begin{align}
      \If last \Then \set{w' \mid f(v).w' \in (f(\rho))(c)} \\
      \Else \set{w' \mid 
         \begin{align}
         f(v).w'.u' \in (f(\rho))(c), \\
         arity(pat) = arity(w')} 
         \end{align}
      \end{align} \\
    \In 
    \set{(w', \bind (f(\rho))~pat~w') \mid 
       w' \in W', \matches (f(\rho))~pat~w'}
    \end{align} 
  \\
  = & \evalField (f(\rho))~last~(c.f(v))~(?pat) .
  \end{calc}

%% *** Need that $\rho(c)$ is symmetric: if $v \in \rho(c)$ and $f(v) = f(v')$
%% then $v' \in \rho(c)$.  

\item Case $?pat:E$.  This is very similar to the previous case.
  Note that $f(\eval \rho~ E) = \eval(f(\rho))~ E$, by the inductive
  hypothesis.  Recall that we do not assume $E$ is invariant.

  If, for some $w \in\eval \rho~ E$,\, $v.w$ is not a prefix of $\rho(c)$,
  then the left-hand side evaluates to $error$.  But in this case,
  $f(w) \in \eval (f(\rho))~E$, and $f(v).f(w)$ is not a prefix of
  $(f(\rho))(c)$ by Corollary~\ref{cor:channel-types}, so the right-hand
  side also evaluates to $error$.  
% 
  Conversely, if for some $w' \in \eval (f(\rho))~ E$,\, $f(v).w'$ is
  not a prefix of $(f(\rho))(c)$, then the right-hand side evaluates to
  $error$.  But then $w' = f(w)$ for some $w \in \eval \rho~E$, and $v.w$ is
  not a prefix of $\rho(c)$, so the left-hand side also evaluates to~$error$. 

  %% An error arises if any element of $\set{v.w \| w \in \eval \rho~ E}$ is not
  %% a prefix of a member of $\rho(c)$; in which case, an element of $\set{v.w'
  %%   \| w' \in \eval (f(\rho))~ E}$ is not a prefix of a member of
  %% $(f(\rho))(c)$, and vice versa, by Corollary~\ref{cor:channel-types}; both
  %% sides then evaluate to $error$.

  Otherwise, in the main set comprehensions, $w$ ranges over $W \inter \eval
  \rho~ E$.  Then $w'$ ranges over $W' \inter f(\eval \rho~ E) = W' \inter
  \eval(f(\rho))~ E$.
\end{itemize} %% End of case analysis on field

%%%%%%%%%% evalFields

We now define a function that accumulate over the fields to get the
resulting events and environments (but handling errors appropriately).
\[
\begin{align}
\evalFields :: 
  \begin{align}
  Env \fun  Value \fun Field^* \fun \\
  \qquad (\power (Value \cross Env) \union \set{error})
  \end{align} \\
\evalFields \rho~(c.v)~\seq{} = \set{(c.v,\rho)} \\
% \evalFields \rho~(c.v)~\seq{f}  =  \evalField \rho~true~(c.v)~f \\
\evalFields \rho~(c.v)~(\seq{field} \cat fs)  = \\
\qquad
  \begin{align}
  \Let S = \evalField \rho~(fs=\seq{})~(c.v)~field \In \\
  \If S = \error \Then \error \\
  \Else \Union \set{ \evalFields \rho'~(c.w)~fs \mid  (c.w, \rho') \in S }.
  \end{align}
\end{align}
\]
(The term $fs=\seq{}$ tests whether $f$ is the last field.)
We can then show 
\begin{eqnarray*}%\label{eqn:evalFields}
f(\evalFields \rho~(c.v)~fs) & = & 
  \evalFields (f(\rho))~(c.f(v))~fs ,
\end{eqnarray*}
%
by an induction on $fs$ (identical to the corresponding proof
in~\cite{symmetry-reduction}). 

%%%%%%%%%%

We now consider the process $e_c~ fs \then P$.  
Let $c.v = \eval \rho~e_c$.  Then
$\eval (f(\rho))~e_c = c.f(v)$, by the inductive hypothesis. 
Let
%
\begin{eqnarray*}
L & = & \eval \rho~ (e_c~ fs \then P), \\
S & = & \evalFields \rho~(c.v)~fs, \\
L' & = &  \eval (f(\rho))~ (e_c~fs \then P), \\
S' & = & \evalFields (f(\rho))~(c.f(v))~fs.
\end{eqnarray*}%
%
The above result shows $f(S) = S'$.  We need to show $f(L) = L'$.

If $S = error$ then $L = error$ and $f(L) = error$.  But then $S' = error$,
so $L' = error$, as required. 

Otherwise, $L$ is an LTS as follows.
%
\begin{itemize}
\item The initial state is $(e_c~ fs \then P, \rho)$.

\item For each $(a, \rho') \in S$, there is a transition labelled with $a$
  from the initial state.  %  to $(P, \rho')$.

\item For each $(a, \rho') \in S$, the sub-LTS following the transition of
  the previous item equals $\eval \rho'~ P$.
\end{itemize}
%
Then $f(L)$ is an  LTS as follows.
%
\begin{itemize}
\item The initial state is $(e_c~fs \then P, f(\rho))$.
  This is the same as the initial state of~$L'$.

\item For each $(a, \rho') \in S$, there is a transition labelled with $f(a)$
  from the initial state.  Letting $a' = f(a)$ and $\rho'' = f(\rho')$,
  this is equivalent to saying that for every $(a',\rho'') \in f(S) = S'$,
  there is a transition labelled with~$a'$ from the initial state.  This is
  the same as the initial transitions of~$L'$.

\item For each $(a, \rho') \in S$, the sub-LTS after the transition of the
  previous item equals $f(\eval \rho'~ P)$.  But this equals $\eval
  (f(\rho'))~P$, by the inductive hypothesis.  Letting $a' = f(a)$ and
  $\rho'' = f(\rho')$, this is equivalent to saying that for every
  $(a',\rho'') \in f(S) = S'$, the sub-LTS after this transition equals $\eval
  \rho''~P$.  This is the same as for~$L'$.
\end{itemize} 
%
Hence $f(L) = L'$, as required. 

%%%%%%%%%%

\subparagraph{Case of at least one \$-field.}

Now suppose that the prefixing construct contains at least one $\$$-field.
We define the semantics using a function 
\[
\evalDollarField ::
  Env \fun Bool \fun Value \fun Field \fun  \power (Value \cross Env).
\]
If $c.v$ is an incomplete event on channel~$c$, and $field$ is a $\$$-field,
then, for a correct usage, $\evalDollarField \rho~last~(c.v)~field$ gives a
set of $(w,\rho')$ pairs; each~$w$ is a value that could be communicated in
the place of~$field$, compatible with~$c.v$, and $\rho'$ is the updated
environment caused by binding any variables in patterns in~$field$.
Compatibility again means that if $\rho(c) = \channel S$ then $v.w$ is a
prefix of some element of~$S$, denoted $v.w \le \rho(c)$.  The argument~$last$
of $\evalField$ again indicates whether this is the final field in the
prefixing construct.

We can prove the following, for $field$ a \$-field:
\[ % begin{equation}\label{eqn:evalDollarField}
  f (\evalDollarField \rho~last~(c.v)~field)  = 
    \evalDollarField (f(\rho))~last~(c.f(v))~field .
\] % end{equation}
The proof is almost identical to the cases for ?-fields in the proof of
equation~(\ref{eqn:evalField}), and so is omitted. 

We now define a function that, given an environment~$\rho$, an incomplete
event~$c.v$, and a list of fields~$fs$, gives a set of pairs $(fs', \rho')$
such that $fs'$ is the result of substituting each \$-field of $fs$ with a
field $!x$ where $x$ is a fresh variable, and $\rho'$ is the environment
resulting from binding~$x$ to a value~$w$ that could instantiate this field,
and also binding any variables in the \$-fields with the corresponding values.
The definition makes use of the fact that every \$-field must precede every !-
or ?-field.
\[
\begin{align}
\evalDollarFields :: 
  Env \fun  Value \fun Field^* \fun  \power (Field^* \cross Env) ,
\\
\evalDollarFields \rho~(c.v)~\seq{} = \set{(\seq{},\rho)} ,
\\
\evalDollarFields \rho~(c.v)~(\seq{field} \cat fs)  = \\
\quad
  \begin{align}
  \If \mbox{$field$ is a \$-field} \\
  \Then 
    \begin{align}
    \set{ (\seq{!x} \cat fs', \rho'') \mid 
    (w,\rho') \in \evalDollarField \rho~(fs=\seq{})~(c.v)~field, \\
    \qquad\quad (fs', \rho'') \in
       \evalDollarFields (\rho' \union \set{x \mapsto w})~ (c.v.w)~fs } \\
    \mbox{where $x$ is  fresh}
    \end{align} \\
  \Else \set{ (\seq{field} \cat fs, \rho) } .
  \end{align}
\end{align}
\]

We can then  show
\begin{eqnarray*} % \label{eqn:evalDollarFields}
f (\evalDollarFields \rho~(c.v)~fs) & = & 
  \evalDollarFields (f(\rho))~(c.f(v))~fs ,
\end{eqnarray*}
%
by an induction on $fs$ (again, identical to the corresponding proof
in~\cite{symmetry-reduction}).

We now consider the process $e_c~ fs \then P$; recall we are assuming there is
at least one \$-field.  Let $c.v = \eval \rho~e_c$.  So $\eval
(f(\rho))~e_c = f (\eval \rho~e_c) = c.f(v)$, using the inductive
hypothesis.  So let
\begin{eqnarray*}
L & = & \eval \rho ~ (e_c~fs \then P), \\
S & = & \evalDollarFields \rho ~(c.v)~ fs, \\
L' & = & \eval (f(\rho))~ (e_c~fs \then P), \\
S' & = & \evalDollarFields (f(\rho)) ~(c.f(v)) ~fs .
\end{eqnarray*}%
%
The above result shows
$f(S) =  S'$.
We need to show $f(L) = L'$.

If $S = \set{}$ then $L = error$ and $f(L) = error$.  But then $S' =
\set{}$, so $L' = error$, as required. 

Otherwise, $L$ is an augmented LTS as follows.
\begin{itemize}
\item The initial state is $(e_c~fs \then P, \rho)$.

\item From the initial state, for each $(fs',\rho') \in S$, there is a
  $\tau$-transition to $(e_c~fs' \then P, \rho')$.

\item For each $(fs',\rho') \in S$, the sub-LTS rooted at
  $(e_c~fs' \then P, \rho')$ equals $\eval~\rho'~(e_c~fs' \then P)$.
\end{itemize}

%
Then $f(L)$ is an augmented LTS as follows.
\begin{itemize}
\item The initial state is
  $(e_c~fs \then P, f(\rho))$.  This is the
  same as the initial state of~$L'$.

\item From this initial state, for each $(fs',\rho') \in S$, there is a
  $\tau$-transition to
  \(
  (e_c~fs' \then P, f(\rho')). 
  \)
  %since the only constants from~$\T$ must appear in~$fs'$ and~$v$.  
  Letting $\rho'' = f(\rho')$, this is equivalent to
  saying that for each $(fs',\rho'') \in f(S) = S'$, there is a
  $\tau$-transition to $(e_c~fs' \then P,\linebreak[1] \rho'')$.
  This is the same as the initial transitions of~$L'$.

\item For each $(fs',\rho') \in S$, the sub-LTS rooted at
  $(e_c~fs' \then  P,\linebreak[2]
    {f(\rho')})$ is 
  equal to 
  \( f(\eval~\rho'~(e_c~fs' \then P)) . \)
  But this equals $\eval~{(f(\rho'))} ~ (e_c~fs' \then P)$
  using the result for the case that there are no \$-fields.
  % (equation~(\ref{eqn:prefix-no-dollar})).  
  Letting 
  $\rho'' = f(\rho')$, this is equivalent to saying that for each
  $(fs',\rho'') \in f(S) = S'$, the sub-LTS rooted at
  $(e_c~fs' \then P, \linebreak[2] \rho'')$ is equal to
  $\eval~\rho''~(e_c~fs' \then P)$.  This is the same as for~$L'$.
\end{itemize}
%
Hence $f(L) = L'$, as required.





%%%%%%%%%%%%%%%%%%%%%%%%%%%%%%%%%%%%%%%%%%%%%%%%%%%%%%%

\subsubsection{Non-replicated sequential operators.}
\label{sec:sequential-procs}

We now consider non-replicated sequential operators.  

\paragraph{Binary sequential operators.}

We start with external choice.
%
Below we write $\extchoice$ for a semantic operator that takes two ALTSs and
forms an external choice over them.  
%% : ``${P \extchoice Q}$''
%% represents an external choice between~$P$ and~$Q$, each of which is an LTS\@.
So $\eval \rho~(P\; \M{[]} Q) = (\eval \rho~P) \extchoice (\eval \rho~Q)$,
where the first ``$\M{[]}$'' is the syntactic operator, and the second
$\extchoice$ is the semantic operator.  

By induction, we can assume that $\eval \rho~P \bisim_f \eval (f(\rho))~ P$,
and similarly for~$Q$.  We show that $\eval \rho~(P \;\M{[]} Q) \bisim_f \eval
(f(\rho))~ (P \;\M{[]} Q)$, following Definition~\ref{def:f-bisim}.  It
is enough to consider transitions up to the resolution of the choice.
%
\begin{enumerate}
\item Consider a transition $(P\; \M{[]} Q, \rho) \trans{a}_{T_1}
(P', \rho')$ corresponding to a transition $(P, \rho) \trans{a}_{T_1}
(P',\rho')$ with $a \ne \tau$.  Then by induction, $(P,
f(\rho)) \trans{f(a)}_{T_2} (P', f(\rho'))$, and so $(P\; \M{[]} Q,
f(\rho)) \trans{f(a)}_{T_2} (P', f(\rho'))$.

The cases for transitions of~$Q$ and $\tau$-transitions are very similar. 

\item Consider a transition $(P\; \M{[]} Q, f(\rho)) \trans{b}_{T_2}
(P', \rho')$ corresponding to a transition $(P, f(\rho)) \trans{b}_{T_2}
(P', \rho')$ with $b \ne \tau$.  But then by induction, there is some $a$
and~$\rho''$ such that $(P, \rho) \trans{a}_{T_1} (P', \rho'')$ with $f(a) =
b$ (so $a \ne \tau$) and $f(\rho'') = \rho'$.  But then $(P\; \M{[]}
Q, \rho) \trans{a}_{T_1} (P', \rho'')$.

Again, the cases for transitions of~$Q$ and $\tau$-transitions are very
similar.
\end{enumerate}

%% Hence $f(\eval \rho ~ (P \; \M{[]} Q)) = \eval (f(\rho)) (P \; \M{[]} Q)$ by
%% Lemma~\ref{lem:ALTS-respects-f-implication}. 

%% We show that
%% %
%% \begin{eqnarray*}
%% f(P \extchoice Q) & = & f(P) \extchoice f(Q).
%% \end{eqnarray*}
%% %
%% For every transition $P \trans{e} P'$, the initial state of each of the above
%% processes has a transition labelled $f(e)$ to~$f(P')$; and similarly for
%% transitions of~$Q$.

%% We then have the following (for syntactic expressions~$P$ and~$Q$; in the
%% first and last lines, ``\CSPM{[]}'' is syntax; in other lines, it is the
%% semantic operator):
%% %
%% \begin{calc}
%% & f(\eval \rho ~ (P \; \M{[]} Q) \\
%% = & f((\eval \rho~P) \extchoice (\eval \rho~Q)) \\
%% = & \com{above result} \\
%% & f(\eval \rho~P) \extchoice f(\eval \rho~Q) \\
%% = & \com{inductive hypothesis} \\
%% & (\eval (f(\rho))~P) \extchoice (\eval (f(\rho))~Q) \\
%% = & \eval (f(\rho)) (P \; \M{[]} Q).
%% \end{calc}

%%%%%

The proofs for the binary operators nondeterministic choice (\CSPM{_ \|~\|
_}), sliding choice (\CSPM{_ [> _}), interrupt (\CSPM{_ /\\ _}), and
sequential composition (\CSPM{_ ; _}) are very similar.

%%%%%


For the exception operator (\CSPM{_ [\| E \|> _}), recall that we assume that
|E| is invariant (item~\ref{item:di-invariant} of
Definition~\ref{defn:data-independent}).  Hence $\eval \rho~E$ is
$T_1$-invariant, by Lemma~\ref{lem:invariant}, so $a \in \eval \rho~E \iff
f(a) \in f(\eval \rho~E) = \eval (f(\rho))~E$ by
Lemma~\ref{lem:T-invariant-inclusion}.

We assume, by induction, that $\eval \rho~P \bisim_f \eval (f(\rho))~P$, and
similarly for~$Q$.  We show that $\eval \rho~(P \throw{E} Q) \bisim_f \eval
(f(\rho)) (P \throw{E} Q)$.  It is enough to consider transitions up to the
resolution of the exception operator; we write $P'$ for the corresponding
state of~$P$.
%
\begin{enumerate}
\item Consider a transition $(P' \throw{E} Q, \rho) \trans{a}_{T_1}
(P'' \throw{E} Q, \rho')$ corresponding to a transition
$(P', \rho) \trans{a}_{T_1} (P'', \rho')$, with $a \nin \eval \rho~E$.  Then
by induction, $(P', f(\rho)) \trans{f(a)}_{T_2} (P'', f(\rho'))$.  Also,
$f(a) \nin f(\eval \rho~E)$, by $T_1$-invariance.  Hence $(P' \throw{E} Q,
f(\rho)) \trans{f(a)}_{T_2} (P'' \throw{E} Q, f(\rho'))$.

The case of $a \in \eval \rho~E$ is very similar: then $f(a) \in
f(\eval \rho~E)$, and both transitions of $P' \throw{E} Q$ produce~$Q$.

\item Consider a transition $(P' \throw{E} Q, f(\rho)) \trans{b}_{T_2}
(P'' \throw{E} Q, \rho')$ corresponding to a transition $(P',
f(\rho)) \trans{b}_{T_2} (P'', \rho')$, with $b \nin \eval (f(\rho))~E$.  Then
by induction, there is some~$a$ and $\rho''$ such that
$(P', \rho) \trans{a}_{T_1} (P'', \rho'')$ with $f(a) = b$ and $f(\rho'')
= \rho'$.  But then $a \nin \eval \rho~E$, by $T_1$-invariance.  Hence
$(P' \throw{E} Q, \rho) \trans{a}_{T_1} (P'' \throw{E} Q, \rho'')$.

The case of $b \in \eval (f(\rho))~E$ is again very similar.
\end{enumerate}


%% %  
%% We write $\_ \throw{A} \_$ for the corresponding semantic operator.  We show
%% that if $A$ is $T_1$-invariant then
%% \begin{eqnarray*}
%% f(P \throw{A} Q) & = & f(P) \throw{f(A)} f(Q).
%% \end{eqnarray*}
%% %
%% On the left-hand side, every transition of~$P$ on an event from~$A$ is replace
%% by a transition to the initial state of~$Q$, and then all events are renamed
%% by~$f$.  On the right-hand side, the events are renamed first, and then every
%% transition from $f(P)$ on an event from~$f(e) \in f(A)$ is replaced by a
%% transition to the initial state of~$f(Q)$.  But if $f(e) \in f(A)$ then $e \in
%% A$, by Lemma~\ref{lem:T-invariant-inclusion}, so there is a corresponding
%% transition on the left-hand side.  The rest of the proof is then as for
%% external choice.


%%%%%%%%%%%%%%%%%%%%%%%%%%%%%%%%%%%%%%%%%%%%%%%%%%%%%%%

\paragraph{Renaming}

The semantics of a renaming is a set of pairs of events $(e,e')$ indicating
that~$e$ is renamed to~$e'$.  
%% We start by considering renamings where each individual renaming \CSPM{e <-
%% e'} applies to complete events; we then extend to channels.
We start by showing the renaming itself satisfies the desired result.  Later,
we use this to show that the application of a renaming to a process also
satisfies the desired result.  
%
\begin{lemma}
\label{lem:eval-renaming}
Let $\rename{ren}$ be a renaming.  Then 
\begin{eqnarray*}
f(\eval \rho~\rename{ren}) & = & \eval (f(\rho))~ \rename{ren}.
\end{eqnarray*}
\end{lemma}
%
\begin{proof}
Let $\rename{ren} = \rename{ e_1 \lArrow e_1', \ldots, e_k \lArrow e_k' \|
stmts}$.  Suppose, for the moment, that each individual renaming $e_i \lArrow
e_i'$ is complete, i.e.~it renames an event as opposed to a channel.  Below,
``$id_1$'' represents the identity function over events based on~$T_1$, and
``$id_2$'' represents the identity function over events based on~$T_2$.
%
\begin{calc}
& f(\eval \rho~
   \rename{ e_1 \leftarrow e_1', \ldots, e_k \leftarrow e_k' \|  stmts}) \\
= & f(id_1 \oplus \set{ (\eval \rho'~e_i, \eval \rho'~e_i') \|
     \rho' \in \evalStmtsSet \rho~stmts, i \in \set{1,\ldots,k} }) \\
= & id_2 \oplus \set{ (f(\eval \rho'~e_i), f(\eval \rho'~e_i')) \|
     \rho' \in \evalStmtsSet \rho~stmts, i \in \set{1,\ldots,k} } \\
= & \com{inductive hypothesis} \\
  & \begin{align}
    id_2 \oplus\null \\ 
    \quad \set{ (\eval (f(\rho'))~e_i, \eval (f(\rho'))~e_i') \|
     \rho' \in \evalStmtsSet \rho~stmts, i \in \set{1,\ldots,k} }
     \end{align} \\
= & \com{letting $\rho'' = f(\rho')$} \\ 
  & id_2 \oplus \set{ (\eval \rho''~e_i, \eval \rho''~e_i') \|
     \rho'' \in f(\evalStmtsSet \rho~stmts), i \in \set{1,\ldots,k} } \\
= & \com{inductive hypothesis, part~\ref{item:evalStmtsSet}} \\
  & id_2 \oplus \set{ (\eval \rho''~e_i, \eval \rho''~e_i') \|
     \rho'' \in \evalStmtsSet (f(\rho))~stmts, i \in \set{1,\ldots,k} } \\
= & \eval (f(\rho))
       \rename{ e_1 \leftarrow e_1', \ldots, e_k \leftarrow e_k' \|  stmts}.
\end{calc}

Finally, if an individual renaming $e_i \lArrow e_i'$ is not complete, we can
replace it by a complete renaming of the form $e_i.x_1.\ldots.x_k \lArrow
e_i'.x_1.\ldots.x_k$, with additional generators for $x_1,\ldots,x_k$, and then
apply the above result.
\end{proof}

%%%%%

Recall that each renaming satisfies condition~\ref{item:di-renaming} of
Definition~\ref{defn:data-independent}.  We will need the following lemma,
which shows that the value of a renaming is consistent with~$f$.
%
\begin{lemma}
\label{lem:renaming-invariant}
Consider a renaming $\rename{r} = \rename{ e_1 \lArrow e_1', \ldots, e_k
  \lArrow e_k' \| stmts}$, and let $R = \eval \rho~ \rename{r}$.  Then
%
\begin{eqnarray*}
%\label{eqn:renaming-invariant}
(v,v') \in R \land f(v) = f(w) & \implies &
  \exists w' \spot (w,w') \in R \land f(v') = f(w').
\end{eqnarray*}
\end{lemma}

%%%%%

\begin{proof}
Without loss of generality, we can assume that each individual renaming
$e_i \lArrow e_i'$ is complete, i.e.~it renames an event as opposed to a
channel (otherwise we can use an argument similar to that at the end of the
proof of Lemma~\ref{lem:eval-renaming}).

Suppose $(v,v') \in R$ and $f(v) = f(w)$.  Suppose $(v,v')$ is produced by the
individual renaming $e_j \lArrow e_j'$.  

Consider the fields $v_1,\ldots,v_k$ in~$v$ of types that depend upon~$t$.  By
condition~\ref{item:di-renaming} of Definition~\ref{defn:data-independent},
these correspond to distinct identifiers $x_1,\ldots,x_k$ in~$e_j$, each of
which is bound by a generator $x_i \lArrow s_i$ in $stmts$, where $s_i$ is
invariant.  Let $S_i = \eval \rho_i~s_i$, where $\rho_i$ is the
environment~$\rho$ augmented corresponding to values taken by earlier
generators in the production of~$(v,v')$; so $v_i \in S_i$.  By
Lemma~\ref{lem:invariant}, $S_i$~is invariant.

Let the corresponding fields in~$w$ be $w_1,\ldots,w_k$, so $f(v_i) = f(w_i)$
for each~$i$.  We show that the identifiers $x_1,\ldots,x_k$ could also be
bound to $w_1,\ldots,w_k$ by the generators, with all other generators being
bound as in the production of~$(v,v')$.  
Consider a generator $x_i \,\M{<-}\, s_i$.  Let $\rho_i'$ be the
environment~$\rho$ augmented corresponding to values taken by earlier
generators.  But, by assumption, $s_i$ uses none of the earlier bound
variables~$x_1,\ldots,x_{i-1}$, and so $\eval \rho_i'~s_i = \eval \rho_i~s_i =
S_i$.   And $w_i \in S_i$, since $S_i$ is $T$-invariant, so this generator can
indeed produce~$w_i$.  Further, the identifiers $x_1,\ldots,x_k$ do not
appear in predicates in $stmts$, so each such predicate will still be true
with this binding.  Then  the individual renaming $e_j \lArrow e_j'$
produces $(w,w')$ for some $w'$ such that $f(v') = f(w')$.
%% Consider the fields
%% $v_1,\ldots,v_k$ in~$v$ of type~$t$.  By condition~\ref{item:di-renaming} of
%% Definition~\ref{defn:data-independent}, these correspond to distinct
%% identifiers $x_1,\ldots,x_k$ in~$e_j$, each of which is bound by a generator
%% $x_i \lArrow t$ in $stmts$.  Let the corresponding fields in~$w$ be
%% $w_1,\ldots,w_k$, so $f(v_i) = f(w_i)$ for each~$i$.
%% %
%% Consider the case where each~$x_i$ is bound to $w_i$, and every other
%% generator is bound as in the production of $(v,v')$.  These identifiers do not
%% appear in predicates in $stmts$, so each such predicate will still be true
%% with this binding.  Then the individual renaming $e_j \lArrow e_j'$
%% produces $(w,w')$ for some $w'$ such that $f(v') = f(w')$.
\end{proof}

%% \framebox{Generalise}, like for set comprehensions.

%%%%%%%%%%%%%%%%%%%%%%%%%%%%%%%%%%%%%%%%%%%%%%%%%%%%%%%%%%%%

%% \framebox{** Partial events?}  Extend to total. 

Consider a syntactic renaming $\rename{ren}$, which corresponds to a
semantic renaming $R$ (i.e.~a set of pairs).  When applied to a LTS~$P$, each
state $(Q,\rho)$ of~$P$ is replaced by $(Q\M{[[}ren\M{]]}, \rho)$, and the
transitions renamed according to~$R$.  We denote this renamed LTS
$P\rename{R}^{ren}$.
%
\begin{lemma}
\label{lem:renaming-f}
Let $P$ be an LTS, $\rename{ren}$ a syntactic renaming, and $R$ the
corresponding renaming relation.  Then
%
\begin{eqnarray*}
f(P\rename{R}^{ren}) & = & (f(P))\rename{f(R)}^{ren}.
\end{eqnarray*}
%
\end{lemma}

%%%%%

\begin{proof}
For each state $(Q,\rho)$ of~$P$, each side of the equation will have a
state $(Q\rename{ren}, f(\rho))$.

%% On each side, each state $Q$ of $P$ is replaced by $f(Q)$ (here each state $Q$
%% will be a pair $(QQ\M{[[}ren\M{]]}, \rho)$, where $QQ$ is a syntactic
%% process, \M{[[}ren\M{]]} is a syntactic renaming, and $\rho$ is an
%% environment). 

Suppose that $P$ contains a transition $(Q,\rho) \trans{e} (Q',\rho')$.  Then
$P\rename{R}^{ren}$ contains a transition $(Q\rename{ren},\rho) \trans{e'}
(Q'\rename{ren},\rho')$ for every $(e,e') \in R$; and so the left-hand side
contains a transition $(Q\rename{ren}, f(\rho)) \trans{f(e')}
(Q'\rename{ren}, {f(\rho')})$.  And $f(P)$ contains a transition $(Q,
f(\rho)) \trans{f(e)} (Q', f(\rho'))$; so the right-hand side contains
a transition $(Q\rename{ren}, f(\rho)) \trans{e''} (Q'\rename{ren},
{f(\rho')})$ for every $e''$ such that $(f(e), e'') \in f(R)$.  We show
that these transitions correspond.
%
\begin{itemize}
\item Consider such a transition of the left-hand side labelled with~$f(e')$.
Then $(e,e') \in R$ implies $(f(e),f(e')) \in f(R)$, so the right-hand side
  has a corresponding transition.

\item Consider such a transition of the right-hand side labelled with~$e''$.
  If $(f(e), e'') \in f(R)$ then there is some $(e_1,e_2) \in R$ such that
  $f(e) = f(e_1)$ and $e'' = f(e_2)$.  But then
  Lemma~\ref{lem:renaming-invariant} implies $(e,e_3) \in R$ for some $e_3$
  such that $f(e_3) = f(e_2) = e''$.  Hence the left-hand side has a
  corresponding transition labelled with $f(e_3) = e''$.
\end{itemize}
\end{proof}

%%%%%

Let $\rename{ren} = \rename{ e_1 \leftarrow e_1', \ldots, e_k \leftarrow e_k' \|
  stmts}$ be a renaming.  We show the desired result is satisfied for the
application of~$ren$ to a process~$P$.
%
\begin{calc}
& f(\eval \rho ~ (P~\rename{ren})) \\
= & f( (\eval \rho~P) \rename{ \eval \rho~ren }^{ren} ) \\
= & \com{Lemma \ref{lem:renaming-f}} \\
  &  (f(\eval \rho~P)) \rename{ f(\eval \rho~ren) }^{ren} \\
= & \com{inductive hypothesis; Lemma \ref{lem:eval-renaming}} \\
  &  (\eval (f(\rho))~P)) \rename{ \eval (f(\rho))~ren }^{ren} \\
= & \eval (f(\rho)) (P~\rename{ren}).
\end{calc}


%%%%%%%%%%%%%%%%%%%%%%%%%%%%%%%%%%%%%%%%%%%%%%%%%%%%%%%

\paragraph{Other sequential operators.} 

For hiding, note that if $A$ is $T_1$-invariant, then
\begin{eqnarray*}
f(P \hide A) & = & f(P) \hide f(A).
\end{eqnarray*}
%
On the right-hand side, an event $e$ of~$P$ is hidden if $f(e) \in f(A)$; but
then $e \in A$, by Lemma~\ref{lem:T-invariant-inclusion}, so the same event is
hidden on the left-hand side.  The rest of the proof is then as for external
choice.

The case for projection is almost identical to that for hiding. 

The result holds for structural operators (local declarations and |if|
statements) as in Section~\ref{sec:structural}.

A guarded process $g ~\M{\&}~ P$ is equivalent to $\M{if}~ g ~ \M{then}~
P ~ \M{else STOP}$.  The result follows from the result for |if| statements.

%%%%%%%%%%%%%%%%%%%%%%%%%%%%%%%%%%%%%%%%%%%%%%%%%%%%%%%

\subsubsection{Binary Parallel operators.}
\label{sec:parallel}

We now consider the binary parallel operators.  We start with the generalised
parallel operator $\_ \synPar{e_A}  \_$\,.  

We write $\_ \parallel{A}^{e_A} \_$ for a semantic operator that forms the
parallel composition of LTSs, synchronising on events from~$A$, corresponding
to the syntactic expression~$e_A$.  Recall that we assume that the identifiers
bound by the two components of the parallel composition are disjoint; this
means that we can restrict a ``global'' environment to the identifiers of each
component, or form the union of the environments of the two components.  For
example, if in LTS~$L_P$ there is a transition $(P, \rho_P) \trans{e}
(P', \rho_P')$, and in LTS~$L_Q$ there is a transition $(Q, \rho_Q) \trans{e}
(Q', \rho_Q')$, with $e \in A$, and $\rho_P$ and~$\rho_Q$ agree on common
identifiers, then in LTS $L_P \parallel{A}^{e_A} L_Q$ there is a transition
$(P \synPar{e_A} Q, \rho_P \union \rho_Q) \trans{e} (P'
\synPar{e_A} Q', \rho_P' \union \rho_Q')$: the resulting environments
$\rho_P'$ and $\rho_Q'$ agree on common identifiers, so their union makes
sense.

Let $A = \eval \rho~e_A$.  So $\eval (f(\rho))~e_A = f(A)$, by induction.
Then
\begin{eqnarray*}
\eval \rho~ (P_1 \synPar{e_A} P_2) & = & 
  (\eval \rho~ P_1) \parallel{A}^{e_A} (\eval \rho~ P_2) , \\
\eval (f(\rho)) ~ (P_1 \synPar{e_A} P_2) & = & 
  (\eval (f(\rho))~ P_1) \parallel{f(A)}^{e_A} (\eval (f(\rho))~ P_2).
\end{eqnarray*}
%
Recall that we assume that $e_A$ is invariant, and hence $A$ $T_1$-invariant.

We can prove that $\eval \rho~ (P_1 \synPar{e_A} P_2) \bisim_f \eval (f(\rho))
~ (P_1 \synPar{e_A} P_2)$.  The tricky case is as follows.
%
Consider a transition, in the ALTS for $\eval (f(\rho)) ~ (P_1 \synPar{e_A}
P_2)$ corresponding to a synchronisation:
%
\begin{eqnarray*}
(Q_1 \synPar{e_A} Q_2, f(\rho_1 \union \rho_2)) & \trans{b}_{T_2} &
  (Q_1' \synPar{e_A} Q_2', \rho_1' \union \rho_2')
\end{eqnarray*}
%
with $b \in f(A)$, corresponding to transitions in the ALTSs for $\eval
f(\rho)~ P_1$ and $\eval f(\rho)~ P_2$:
%
\begin{eqnarray*}
(Q_1, f(\rho_1)) & \trans{b}_{T_2} & (Q_1', \rho_1') , \\
(Q_2, f(\rho_2)) & \trans{b}_{T_2} & (Q_2', \rho_2').
\end{eqnarray*}
%
By the inductive hypothesis, there are transitions in the ALTSs
for $\eval \rho~ P_1$ and $\eval \rho~ P_2$:
\begin{eqnarray*}
(Q_1, \rho_1) & \trans{a_1}_{T_1} & (Q_1', \rho_1'') , 
   \quad \mbox{with $f(a_1) = b$ and $f(\rho_1'') = \rho_1'$}, \\
(Q_2, \rho_2) & \trans{a_2}_{T_1} & (Q_2', \rho_2''),
   \quad \mbox{with $f(a_2) = b$ and $f(\rho_2'') = \rho_2'$}.
\end{eqnarray*}
%
%% with $f(a_1) = f(a_2) = b$,\, $f(\rho_1'') = \rho_1'$ and $f(\rho_2'')
%% = \rho_2'$. 
%
If $a_1 \ne a_2$, then consider the first field in which they differ; say $a_1
= c.\vec{u}.v_1.\vec{w_1}$ and $a_2 = c.\vec{u}.v_2.\vec{w_2}$ with $v_1 \ne
v_2$ but $f(v_1) = f(v_2)$.
%
Then by item~\ref{item:sync} of Definition~\ref{defn:data-independent}, one of
the processes must be willing to accept an arbitrary value in this position;
without loss of generality, suppose this is~$Q_1$, so $Q_1$ can also perform
$a_1' = c.\vec{u}.v_2.\vec{w_1}$.  If the transition binds $x$ to this field
in $Q_1$'s environment, then this corresponds to the transition
$(Q_1,\rho_1) \trans{a_1'} (Q_1',\rho_1''')$ where $\rho_1'''
= \rho_1'' \oplus \set{x \mapsto v_2}$.  Note that $f(a_1') = b$ and
$f(\rho_1''') = f(\rho_1'') = \rho_1'$.
%
Continuing in this way, we obtain transitions
\[
\begin{array}{c}
(Q_1,\rho_1) \trans{a'}_{T_1} (Q_1',\hat\rho_1) \quad\mbox{and}\quad
(Q_2,\rho_2) \trans{a'}_{T_1} (Q_2',\hat\rho_2) \\
\mbox{with}\quad f(a')  = b, \quad 
f(\hat\rho_1) = \rho_1', \quad f(\hat\rho_2) = \rho_2'.
\end{array}
\]
Also, $f(a') = b \in f(A)$, and $A$ is $T_1$-invariant, so $a' \in A$.  
Hence in the ALTS for $\eval \rho~ (P_1 \synPar{e_A} P_2)$, we have a
transition synchronising on~$a'$: 
%
\begin{eqnarray*}
(Q_1 \synPar{e_A} Q_2, \rho_1 \union \rho_2) & \trans{a'}_{T_1} &
  (Q_1' \synPar{e_A} Q_2', \hat\rho_1 \union \hat\rho_2),
\end{eqnarray*}
and $f(\hat\rho_1 \union \hat\rho_2) =  \rho_1' \union \rho_2'$, as required.

Other cases ---that every synchronisation of $\eval \rho~ (P_1 \synPar{e_A}
P_2)$ is matched by a synchronisation of $\eval (f(\rho)) ~ (P_1 \synPar{e_A}
P_2)$, and that non-synchronisation transitions match--- can be proved in a
similar but more straightforward way.

%%%%%%%%%%%%%%%%%%%%%%%%%%%%%%%%%%%%%%%%%%%%%%%%%%%%%%%

%% \subsection{OLD}

%% We show that, assuming $A$ is $T_1$-invariant:
%% %
%% \begin{eqnarray}
%% \label{eqn:parallel-dist}
%% f((P,\rho_P) \parallel{A} (Q,\rho_Q)) & = & 
%%   f(P,\rho_P) \parallel{f(A)} f(Q,\rho_Q).
%% \end{eqnarray}
%% %
%% The tricky case is as follows.  Consider a transition of the right-hand side
%% corresponding to a synchronisation on event~$e \in f(A)$.  This must
%% correspond to transitions 
%% \[
%% (P,\rho_P) \trans{e_P} (P',\rho_P') \quad\mbox{and}\quad
%% (Q,\rho_Q) \trans{e_Q} (Q',\rho_Q') \quad
%% \mbox{with}\quad f(e_P) = f(e_Q) = e.
%% \]
%% The subsequent state is 
%% \[
%% f(P',\rho_P') \parallel{f(A)} f(Q',\rho_Q') = 
%%   (P',f(\rho_P')) \parallel{f(A)} (Q',f(\rho_Q')).
%% \]
%% If $e_P \ne e_Q$, then consider the first field in which they differ; say $e_p
%% = c.\vec{u}.v_P.\vec{w_P}$ and $e_Q = c.\vec{u}.v_Q.\vec{w_Q}$ with $v_P \ne
%% v_Q$ but $f(v_P) = f(v_Q)$.
%% %
%% Then by item~\ref{item:sync} of Definition~\ref{defn:data-independent}, one of
%% the processes must be willing to accept an arbitrary value in this position;
%% without loss of generality, suppose this is~$P$, so $P$ can also perform $e_P'
%% = c.\vec{u}.v_Q.\vec{w_P}$.  If the transition binds $x$ to this field in
%% $P$'s environment, then this corresponds to the transition
%% $(P,\rho_P) \trans{e_P'} (P,\rho_P'')$ where $\rho_P''
%% = \rho_P' \oplus \set{x \mapsto v_Q}$.  Note that $f(e_P') = e$ and
%% $f(\rho_P'') = f(\rho_P')$.
%% %
%% Continuing in this way, we obtain transitions
%% \[
%% \begin{array}{c}
%% (P,\rho_P) \trans{e'} (P',\hat\rho_P) \quad\mbox{and}\quad
%% (Q,\rho_Q) \trans{e'} (Q',\hat\rho_Q) \\
%% \mbox{with}\quad f(e')  = e, \quad 
%% f(\hat\rho_P) = f(\rho_P'), \quad f(\hat\rho_Q) = f(\rho_Q').
%% \end{array}
%% \]
%% Also, note that $f(e') = e \in f(A)$, so $e' \in A$, by
%% Lemma~\ref{lem:T-invariant-inclusion}, since $A$ is $T_1$-invariant.
%% %
%% Hence, in the left-hand side of (\ref{eqn:parallel-dist}), the two
%% components can synchronise on~$e'$, giving a transition on $f(e') = e$, and
%% producing state 
%% \[
%% f( (P',\hat\rho_P) \parallel{A} (Q',\hat\rho_Q) ) .
%% \]
%% This matches the original transition of the right-hand side.

%% Other cases ---that every synchronisation of the left-hand side is matched by
%% a synchronisation of the right-hand side, and that non-synchronisation
%% transitions match--- can be proved in a similar but more straightforward way.

%% The proof then proceeds as for other binary operators.  

%%%%%%%%%%%%%%%%%%%%%%%%%%%%%%%%%%%%%%%%%%%%%%%%%%%%%%%

The cases of the alphabetised parallel 
($\M{[} A\, {\|\|}\, B \M{]} $),
interleaving (\CSPM{\|\|\|}), synchronising external choice
(\CSPM{[+}\,$A$\,\CSPM{+]}) and synchronising interrupt
(\CSPM{/+}\,A\,\CSPM{+\\}) operators are similar; recall that it is assumed
that the expressions that define the alphabets are invariant, and hence the
values of those alphabets are $T_1$-invariant.

The case of  link parallel follows from earlier cases, since any instance can
be rewritten using a combination of renaming, generalised parallel, and
hiding: the conditions placed on use of link parallel
(clause~\ref{item:di-renaming} of Definition~\ref{defn:data-independent}) are
enough to imply the requirements for use of those other operators.



%%%%%%%%%%%%%%%%%%%%%%%%%%%%%%%%%%%%%%%%%%%%%%%%%%%%%%%%%%%%%%%%%

\subsubsection{Replicated operators}  
\label{sec:replicated}

We now consider replicated operators.  We start with external choice.  Below
we write $\Extchoice$ for a semantic operator that takes a set of ALTSs and
builds their external choice.


%% that takes a set $T$ of
%% (process, environment) pairs, and builds the  ALTS corresponding to the
%% external choice over processes $\eval \rho Q$ for $(Q,\rho) \in T$.

%% Consider $P = \M{**[]} stmts \spot Q$.  Let $S = \evalStmtsSet \rho~stmts$.
%% Then $\evalStmtsSet (f(\rho))~stmts = f(S)$ by clause \ref{item:evalStmts} of
%% the inductive hypothesis.
%% Hence we have
%% %
%% \begin{eqnarray*}
%% \eval \rho~ P & = &  \Extchoice \set{(Q,\rho') \| \rho' \in S}, \\
%% \eval (f(\rho))~ P & = &   \Extchoice \set{(Q,\rho'') \| \rho'' \in f(S)}. 
%% \end{eqnarray*}

%% We show that $\Extchoice T \bisim_f \Extchoice (f(T))$, which implies
%% $\eval \rho~ P \bisim_f \eval (f(\rho))~ P$.
%% %
%% \begin{enumerate}
%% \item
%% Each initial visible transition of $\Extchoice T$ is of the form
%% $(P,\rho) \trans{a}_{T_1} (Q',\rho'')$ such that $(Q,\rho') \trans{a}_{T_1}
%% (Q',\rho'')$ for some $(Q,\rho') \in T$.  By induction, that is matched by a
%% transition $(Q,f(\rho')) \trans{f(a)}_{T_2} (Q',f(\rho''))$.  But
%% $f(\rho') \in f(S)$, so this produces a corresponding transition from
%% $(P,f(\rho))$.

%% \end{enumerate}

%%%%%%%%%%%%%%%%%%%%%%%%%%%%%%%%%%%%%%%%%%%%%%%%%%%%%%%


Consider $P = \M{**[]} stmts \spot Q$.  Let $S = \evalStmtsSet \rho~stmts$.
Then $\evalStmtsSet (f(\rho))~stmts = f(S)$ by clause \ref{item:evalStmts} of
the inductive hypothesis.  Hence we have
%
\begin{eqnarray*}
\eval \rho~ P & = & 
  \Extchoice \set{\eval \rho' ~ Q \| \rho' \in S}, \\
\eval (f(\rho))~ P & = & 
  \Extchoice \set{\eval \rho'' ~ Q \| \rho'' \in f(S)} 
   =  \Extchoice \set{\eval (f(\rho'))~ Q \| \rho' \in S} .
\end{eqnarray*}
%
We show $\eval \rho~ P \bisim_f \eval (f(\rho))~ P$.  By
induction, $\eval \rho' ~ Q \bisim_f \eval (f(\rho'))~ Q$, for each~$\rho'$,
so it is enough to consider transitions up to the resolution of the external
choice. 
%
\begin{enumerate}
\item
Each initial visible transition of $\eval \rho~ P$ is of the form
$(P,\rho) \trans{a}_{T_1} (Q',\rho'')$ such that $(Q,\rho') \trans{a}_{T_1}
(Q',\rho'')$ for some $\rho' \in S$.  By induction, that is matched by a
transition $(Q,f(\rho')) \trans{f(a)}_{T_2} (Q',f(\rho''))$.  But
$f(\rho') \in f(S)$, so this produces a corresponding transition from
$(P,f(\rho))$.

The case for an initial $\tau$-transition of $\eval \rho~ P$ is similar,
except each subsequent state is an external choice where the component
$(Q,\rho')$ is replaced by $(Q',\rho'')$.  The case for transitions from those
subsequent states is similar.

\item
Each initial visible transition of $\eval (f(\rho))~ P$ is of the form
$(P,f(\rho)) \trans{b}_{T_2} (Q',\rho'')$ such that
$(Q,f(\rho')) \trans{b}_{T_2} (Q',\rho'')$ for some $\rho' \in S$.  By
induction, that is matched by a transition $(Q,\rho') \trans{a}_{T_1}
(Q',\rho''')$ such that $f(a) = b$ and $f(\rho''') = \rho''$, which produces a
corresponding transition from~$(P,\rho)$.

Again, the case for an initial $\tau$-transition is similar. 
\end{enumerate}

%%%%%%%%%%%%%%%%%%%%%%%%%%%%%%%%%%%%%%%%%%%%%%%%%%%%%%%

%% \subsection*{Ignore}

%% We show
%% %
%% \begin{eqnarray*}
%% f(\Extchoice S) & = & \Extchoice \set{f(P) \| P \in S}.
%% \end{eqnarray*}
%% %
%% For every $P \in S$, and for every transition $P \trans{e} Q$, the initial
%% state of each of the above processes has a transition labelled with $f(e)$ to
%% $f(Q)$.

%% We then have the following (the external choices in the first and last line
%% are syntax; the other instances are semantic):
%% \begin{calc}
%% & f(\eval \rho~ (\M{**[]} stmts \spot P)) \\
%% = & f(\Extchoice \set{\eval \rho'~ P \| \rho' \in \evalStmtsSet \rho~stmts}) \\
%% = & \com{above result} \\
%%   & \Extchoice \set{f(\eval \rho' ~P) \| \rho' \in \evalStmtsSet \rho~stmts} \\
%% = & \com{inductive hypothesis for~$P$} \\
%%   & \Extchoice \set{\eval (f(\rho'))~P \| 
%%       \rho' \in \evalStmtsSet \rho~stmts} \\
%% = & \com{letting $\rho'' = f(\rho')$} \\
%%   & \Extchoice \set{\eval \rho''~P \| 
%%       \rho'' \in f(\evalStmtsSet \rho~stmts)} \\
%% = & \com{inductive hypothesis, clause \ref{item:evalStmts}} \\
%%   & \Extchoice \set{\eval \rho''~P \| 
%%       \rho'' \in \evalStmtsSet (f(\rho))~stmts} \\
%% = & \eval (f(\rho))~ (\M{**[]} stmts \spot P).
%% \end{calc}

The case for a replicated nondeterministic choice is similar.  

%%%%%%%%%%%%%%%%%%%%%%%%%%%%%%%%%%%%%%%%%%%%%%%%%%%%%%%


Consider now a replicated interleaving $\M{**\|\|\|} ~ stmts ~ \M{@} ~ P$.
% Recall that the indexing statements~$stmts$ are independent of~|T|.
We write $\Interleave$ for a semantic interleaving operator, which takes a
multiset\footnote{We write a multiset as ``$\mset{\ldots}$''.} of processes as
an argument, and interleaves them.  Then
\begin{eqnarray*}
\eval \rho ~(\M{**\|\|\|} ~ stmts ~ \M{@} ~ P) & = & 
  \Interleave \mset{ \eval \rho'~P \| \rho' \in \evalStmtsSet \rho~stmts }, \\
\eval (f(\rho))~(\M{**\|\|\|} ~ stmts ~ \M{@} ~ P) & = & 
  \Interleave \mset{ \eval \rho''~P \| 
     \rho'' \in \evalStmtsSet (f(\rho))~stmts } \\
& = & \Interleave \mset{ \eval \rho''~P \| 
    \rho'' \in f(\evalStmtsSet \rho~stmts) }.
\end{eqnarray*}
%
using clause~\ref{item:evalStmts} of the inductive hypothesis.

Recall that the indexing statements~$stmts$ are independent of~|T|.  Suppose
$\evalStmtsSet~\rho~stmts$ includes an environment $\rho \oplus \rho'$ where
$\rho'$ captures the bindings in~$stmts$, and so is independent of~|T|; 
then $f(\rho \oplus \rho') = f(\rho) \oplus \rho'$.  
This means that the sets $\evalStmtsSet~\rho~stmts$ and
$f(\evalStmtsSet \rho~stmts)$ have the same cardinality: the application
of~$f$ does not unify two different bindings corresponding to $stmts$.
Hence we also have
%
\begin{eqnarray*}
\eval (f(\rho))~(\M{**\|\|\|} ~ stmts ~ \M{@} ~ P) & = & 
   \Interleave \mset{ \eval (f(\rho'))~P \| \rho' \in \evalStmts \rho~stmts }.
\end{eqnarray*}


Now, $\eval \rho'~P \bisim_f \eval (f(\rho'))~P$ for each~$\rho'$, by
induction.  
Hence we have $\eval \rho ~(\M{**\|\|\|} ~ stmts ~ \M{@} ~
P) \bisim_f \eval (f(\rho))~(\M{**\|\|\|} ~ stmts ~ \M{@} ~ P)$, using the
fact that $f$-bisimulation is compositional with respect to binary
interleaving, and the fact that a replicated interleaving can be rewritten
using binary interleaving.


%%%%%%%%%%%%%%%%%%%%%%%%%%%%%%%%%%%%%%%%%%%%%%%%%%%%%%%

%% multiset\footnote{We write a multiset as ``$\mset{\ldots}$''.} of processes as
%% an argument.  We have that
%% %
%% \begin{eqnarray}
%% \label{eqn:replicated-interleave}
%% f(\Interleave S) & = & \Interleave \mset{f(P) \| P \in S}.
%% \end{eqnarray}
%% %
%% This follows from the corresponding equation $f(P \interleave Q) =
%% f(P) \interleave f(Q)$ for binary interleaving, and the fact that the replicated
%% interleaving can be rewritten using binary interleaving.  It is important here
%% that we are using a \emph{multiset}, so the two sides
%% of~(\ref{eqn:replicated-interleave}) have the same number of processes, even if
%% $f$ unifies two elements of~$S$: this matches the semantics of~\CSPm, which
%% produces a set of environments from $stmts$, and then from each such
%% environment produces a process (possibly with repetitions).

%% Recall that the indexing statements~$stmts$ are independent of~|T|.  Suppose
%% $\evalStmtsSet~\rho~stmts$ includes an environment $\rho \oplus \rho'$ where
%% $\rho'$ captures the bindings in~$stmts$, and so is independent of~$f$; 
%% then $f(\rho \oplus \rho') = f(\rho) \oplus \rho'$.  
%% This means that the sets $\evalStmtsSet~\rho~stmts$ and
%% $f(\evalStmtsSet \rho~stmts)$ have the same cardinality: the application
%% of~$f$ does not unify two different bindings corresponding to $stmts$.

%% We calculate as follows.
%% %
%% \begin{calc}
%% & f(\eval \rho (\M{**\|\|\|} ~ stmts~ \M{@} ~ P)) \\
%% = & f( \Interleave \mset{ \eval \rho' ~ P \|
%%           \rho' \in \evalStmtsSet \rho~stmts} ) \\
%% = & \com{equation (\ref{eqn:replicated-interleave})} \\
%% & \Interleave \mset{ f(\eval \rho'~P) \| 
%%      \rho' \in \evalStmtsSet \rho~stmts}  \\
%% = & \com{inductive hypothesis} \\
%% & \Interleave \mset{ \eval (f(\rho'))~P \| 
%%     \rho' \in \evalStmtsSet \rho~stmts}  \\
%% = & \com{letting $\rho'' = f(\rho')$; above observation about cardinality} \\
%% & \Interleave \mset{ \eval \rho''~P \| 
%%     \rho'' \in f(\evalStmtsSet \rho~stmts)}  \\
%% = & \com{inductive hypothesis, clause~\ref{item:evalStmts}} \\
%% & \Interleave \mset{ \eval \rho''~P \| 
%%     \rho'' \in \evalStmtsSet (f(\rho))~stmts}  \\
%% = & \eval (f(\rho))~(\M{**\|\|\|} ~ stmts~ \M{@}~ P).
%% \end{calc}

%%%%%%%%%%

The cases for other replicated operators are very similar.  Recall that we
assume that each of the alphabets on parallel operators is defined by an
invariant expression, and so evaluates to a $T_1$-invariant set.  In each
case, we can show that $f$-bisimulation is compositional with respect to the
replicated operator because it is compositional with respect to the
corresponding binary operator.  For replicated sequential composition and link
parallel, the statements are evaluated using $\evalStmts$ to produce a
sequence, rather than a set.

% Alphabetised parallel, generalised parallel, link parallel, sequential
% composition. 

%%%%%%%%%%
