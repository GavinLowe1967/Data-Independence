\subsection{$\matches$}

We show that for every pattern $pat$ and value~$v$
\begin{eqnarray*}
  \matches \rho~ pat~v & \iff & \matches (f(\rho)) ~ pat~(f(v)).
\end{eqnarray*}

We proceed by a case analysis over $pat$.   
%
\begin{itemize}
  \item Case $x$, a variable.  This matches all values, so both sides are
    $true$. 

  \item A pattern of the form $pat = pat_1 @@ pat_2$.  Then (using the
    inductive hypothesis):
    \begin{calc}
    & \matches \rho~pat~v \\
    \iff & \matches \rho~pat_1~v \land \matches \rho~pat_2~v \\
    \iff & \matches (f(\rho))~pat_1~(f(v)) 
         \land \matches (f(\rho))~pat_2~(f(v)) \\
    \iff & \matches (f(\rho))\rho~pat~(f(v)).
  \end{calc}

  \item A pattern of the form $pat = A$ where $A$ is a constructor of a
    datatype (i.e., $\rho(A)$ and $(f(\rho))(A)$ are $\dtcons$ values).
    Necessarily $A$ is not an element of~$t$ (since the pattern contains no
    constant of type~$t$).  Then $\matches pat~v$ iff $v = A$.  But this holds
    iff $\matches pat~(f(v))$, since $f(A) = A$.
  
  \item A pattern of the form $pat = A.pat'$ where $A$ is a constructor of a
    datatype, and $pat'$ is a non-empty pattern.  Again $A$ is not an element
    of~$t$.  Then $\matches \rho~pat~v$ iff $v$ is of the form $A.v'$ and
    $\matches \rho~pat'~v'$.  By the inductive hypothesis, the latter clause
    holds iff $\matches \linebreak[1] {(f(\rho))}~pat'~(f(v'))$.  But then
    $\matches (f(\rho))~pat~f(v)$, since $f(v) = A.f(v')$.

  \item A pattern of the form $pat = c$ or $pat = c.pat'$ where $c$ is a
    channel name.  These cases are very similar to the two previous cases.

  \item A pattern of the form $pat = x.pat_1$ where $x$ is a variable, and
    $pat_1$ is a pattern.  Consider, first, a dotted value $v = v_0.v_1$,
    where $v_0$ is the shortest prefix of~$v$ that is a completed value, and
    $v_1$ is non-empty (so $x$ will be matched by~$v_0$).  Note that
    $f(v_0)$ is also a completed value, by Lemma~\ref{lem:complete}.  Note
    also that $x$ cannot appear in~$pat_1$.  Then
    \begin{calc}
    & \matches \rho~pat~v \\
    \iff & \matches \rho~pat_1~v_1 \\
    \iff & \com{inductive hypothesis} \\
         & \matches (f(\rho))~pat_1~(f(v_1)) \\
    \iff & \com{$f(v_0)$ is a completed value} \\
         & \matches (f(\rho))~(x.pat_1)~(f(v_0).f(v_1)) \\
    \iff & \matches (f(\rho))~pat~(f(v)).
    \end{calc}%
    If $v$ is not of the above form, then neither $\matches \rho~pat~ v$ nor
    $\matches \linebreak[1] (f(\rho))~pat~(f(v))$ holds.

  \item A pattern of the form $pat = \_.pat_1$ where $pat_1$ is a pattern.
    This case is very similar to the previous.

  \item A pattern of the form $pat = (pat_1 @@ pat_2) . pat_3$.  Suppose
    $pat_1 @@ pat_2$ matches dotted tuples of minimal size $n>0$.  (Note that
    the sub-patterns may end with variables or wildcards that allow them to
    match arbitrarily long dotted tuples; here we are defining $n$ to be the
    minimal length dotted tuple that can be matched; FDR calculates $n$ in a
    straightforward way.)  If $v$ is not a dotted tuple of size
    at least $n+1$ then $pat$ matches neither $v$ nor~$f(v)$.  Let
    $v = v_1.v_2$ where $v_1$ has size~$n$.  Then
    \begin{calc}
    & \matches \rho~pat~v \\
    \iff & \matches \rho~pat_1~v_1 \land \matches \rho~pat_2~v_1 \land
         \matches \rho~pat_3~v_2 \\
    \iff & 
      \begin{align}
      \matches (f(\rho))~pat_1~(f(v_1)) 
      \land \matches (f(\rho))~pat_2~(f(v_1)) \land \\
      \matches (f(\rho))~pat_3~(f(v_2))
      \end{align} \\
    \iff & \matches (f(\rho))~pat~(f(v)),
    \end{calc}%
    noting that $f(v) = f(v_1).f(v_2)$ and $f(v_1)$ has size~$n$.

  \item A pattern of the form $pat = pat_0.pat_1$, other than the previous
    five cases; i.e.~$pat_0$ is a basic value, tuple, set
    or sequence pattern, so $pat_0$ matches just the first term of a dotted
    tuple.  Consider a dotted value $v = v_0.v_1$, where $v_0$ is not a dotted
    value.  Then
    \begin{calc}
    & \matches  \rho~pat~v \\
    \iff & \matches \rho~pat_0~v_0 \land \matches \rho~pat_1~v_1 \\
    \iff & \com{inductive hypothesis} \\
    \iff & \matches (f(\rho))~pat_0~(f(v_0)) \land 
           (f(\rho))~\matches pat_1~(f(v_1)) \\
    \iff & \matches (f(\rho))~(pat_0.pat_1)~(f(v_0).f(v_1)) \\
    \iff & \matches (f(\rho))~pat~(f(v)).
    \end{calc}%
    If $v$ is not a dotted value, then neither $\matches \rho~pat~ v$ nor
    $\matches {(f(\rho))}~ pat~(f(v))$ holds.

  \item All other cases are similar.  % Constants are like a previous
    % case.  Tuples, literal list pattern and variants
    % (e.g.~\verb!<x,y>^z^<a,b,c>!), singleton or empty set, are all similar to
    % the previous cases.  Wildcard is like a variable.
\end{itemize}
