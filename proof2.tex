
\subsection{$\bind$}

We now prove clause~\ref{item:bind} of Proposition~\ref{prop:expressions-1}:
if $pat$ is a pattern, and $v$ a value, then
\begin{eqnarray*}
f \after (\bind_0 \rho~pat~v) &  = & \bind_0 (f\after\rho)~pat~(f(v)).
\end{eqnarray*}
It follows immediately that 
\begin{eqnarray*}
f \after (\bind \rho~pat~v) & = & \bind (f\after\rho)~pat~(f(v)).
\end{eqnarray*}

We proceed by a case analysis on~$pat$.  Most cases are as
in~\cite{symmetry-TR}, but we include them for completeness.  In all cases,
\begin{calc}
& \mbox{$\bind_0 \rho~pat~v$ is defined} \\
\iff & \matches \rho~pat~v \\
\iff & \com{clause~\ref{item:matches}} \\
 &  \matches (f\after\rho)~pat~(f(v)) \\
\iff & \mbox{$\bind_0 (f\after\rho)~pat~(f(v))$ is defined.}
\end{calc}
%
We assume these hold in the following.
%
\begin{itemize}
\item Case $x$, a variable.  
  \begin{calc}
    & f \after \bind_0 \rho~x~v \\
    = & f \after \set{x \mapsto v} \\
    = & \set{x \mapsto f(v)} \\
    = & \bind_0 (f\after\rho)~x~(f(v)).
  \end{calc}

\item A pattern of the form $pat = A$ where $A$ is a constructor of a
  datatype.  Again, $A$ is not a member of~$t$, by assumption.  Necessarily,
  $v = A$, and so $f(v) = A$.  Then
  \[
  f \after \bind_0 \rho~pat~v = \set{}
  = \bind_0 (f \after \rho)~pat~(f(v)).
  \]  

\item A pattern of the form $pat = A.pat_1$ where $A$ is a constructor of a
  datatype and $pat_1$ is nonempty.  Again, $A$ is not a member of~$t$.
  Necessarily, $v$ is of the form $A.v_1$ such that
  $\matches \rho~pat_1~v_1$, and so $f(v) = A.f(v_1)$.  Then 
  \begin{calc}
    & f \after \bind_0 \rho~pat~v \\
    = & f \after \bind_0 \rho~pat_1~v_1 \\
    = & \bind_0 (f \after \rho)~pat_1~(f(v_1)) \\
    = & \bind_0 (f \after \rho)~pat~(f(v)).
  \end{calc}

\item A pattern of the form $pat = pat_0.pat_1$, where $pat_0$ is not a dotted
  term, a variable or wildcard, a datatype constructor or channel name, nor a
  double pattern; hence $pat_0$ matches just the first term of a dotted tuple.
  Necessarily, the variables bound by $pat_0$ and~$pat_1$ are disjoint.
  Consider a dotted value $v = v_0.v_1$ where $v_0$ is not a dotted term; so
  $\matches \rho~pat_0~v_0$ and $\matches \rho~pat_1~v_1$, and $f(v_0)$ is not
  a dotted term.  Then
  \begin{calc}
    & f \after \bind_0 \rho~pat~v \\
    = & f \after (\bind_0 \rho~pat_0~v_0 \union \bind_0 \rho~pat_1~v_1) \\
    = & (f \after \bind_0 \rho~pat_0~v_0) \union 
        (f \after \bind_0 \rho~pat_1~v_1) \\
    = & \bind_0 (f \after \rho)~pat_0~(f(v_0)) \union 
        \bind_0 (f \after \rho)~pat_1~(f(v_1)) \\
    = & \bind_0 (f \after \rho)~(pat_0.pat_1)~(f(v_0).f(v_1)) \\
    = & \bind_0 (f \after \rho)~pat~v.
  \end{calc}

\item A pattern of the form $pat = x.pat_1$ where $x$ is a variable, and
  $pat_1$ is a pattern.  This is very similar to the previous case, except $x$
  matches the shortest prefix~$v_1$ of~$v$ that is a completed value; and $x$
  also matches $f(v_1)$, which is the shortest prefix of~$f(v)$ that is a
  completed value.

\item A pattern of the form $(pat_0 @@ pat_1).pat_2$, where $pat_0 @@ pat_1$
  matches dotted tuples of minimal size $n>0$.  This is very similar to the
  previous case, taking $v_0$ to be the first $n$ terms of~$v$.
\end{itemize}
Other cases are similar.
