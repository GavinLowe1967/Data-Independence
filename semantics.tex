\section{Semantics}
\label{sec:semantics}

Something akin to Appendix~A from the symmetry reduction TR.  This maybe lives
in the main body. 

The semantics is defined using the following functions:
%
\begin{itemize}
\item $\eval : Env \fun Expr \pfun Value$ such that $\eval \rho~e$ gives the
  value of expression~$e$ in environment~$\rho$.

\item $\matches : Env \fun Pat \fun Value \fun Bool$ such that
  $\matches \rho~pat ~ v$ tests whether pattern~$pat$ can be matches by
  value~$v$.

\item $\bind_0, \bind : Env \fun Pat \fun Value \pfun Env$ such that
  $\bind_0 \rho~pat~v$ gives the updates to the environment by binding
  pattern~$pat$ to value~$v$, and is defined when $\matches \rho~ pat~v$; and 
  $\bind \rho~pat~v$ gives the result of those updates, i.e.
  \begin{eqnarray*}
  \bind \rho~pat~v & = & \rho \oplus \bind_0 \rho~pat~v.
  \end{eqnarray*}
% Do we need $\rho$ for $bind_0$?  Yes, we do: we need to know whether an
% identifier represents a datatype constructor or a channel. 

\item $\bindDecls : Env \fun Decl^* \pfun Env$ such that
  $\bindDecls \rho~decls$, updates $\rho$ corresponding to the list of
  declarations $decls$, assuming that all names bound by the declarations are
  distinct, and that each pattern matches the corresponding expression.

\item $\evalStmts : Env \fun Stmt^* \fun Env^*$ (where $Stmt$ contains
  generators or predicates in set or sequence comprehensions) such that
  $\evalStmts \rho~stmts$ gives the sequence of environments resulting from
  evaluating the statements~$stmts$.  (In some circumstances, this sequence of
  environments is converted into a multiset; we elide this conversion
  below.)

\item $\evalStmtsSet : Env \fun Stmt^* \fun \power(Env)$ such that
  $\evalStmtsSet \rho~stmts$ gives the set of environments resulting from
  evaluating the statements~$stmts$.
\end{itemize}

%%%%%%%%%%%%%%%%%%%%%%%%%%%%%%%%%%%%%%%%%%%%%%%%%%%%%%%%%%%%

\section{Channel types}

We'll need the following lemma concerning $T$-invariant sets.
%
\begin{lemma}
\label{lem:T-invariant-inclusion}
Suppose $S$ is $T$-invariant.  Then 
%
\begin{eqnarray*}
v \in S & \iff & f(v) \in f(S).
\end{eqnarray*}
\end{lemma}

%%%%%

\begin{proof}
The left-to-right implication is immediate.  For the right-to-left
implication, suppose $f(v) \in f(S)$.  Then there is some $v' \in S$ such that
$f(v) = f(v')$.  Note that $\unitf \circ f = \unitf$: each maps elements
of~$T$ to~$\unit$, and is lifted in the normal way to other values.  Hence
\[
\unitf(v) = \unitf(f(v)) = \unitf(f(v')) = \unitf(v').
\]
Hence $v \in S$, since $S$ is $T$-invariant.
\end{proof}

%%%%%

The following corollary follows immediately, using
Lemma~\ref{lem:T-invariant-inclusion}.
%
\begin{corollary}
\label{cor:channel-types}
Consider a script instantiating~$t$ with $T$.  Consider a channel
identifier~$c$ and environment~$\rho$, and suppose $\rho(c) = \channel S$.
Then for each value~$v$,
\begin{eqnarray*}
v \in S & \iff & f(v) \in f(S).
\end{eqnarray*}
\end{corollary}

%% \begin{proof}
%% By Lemma~\ref{lem:T-invariant}, $S$ is $T$-invariant.  The left-to-right
%% implication is immediate.  For the right-to-left implication, suppose $f(v)
%% \in f(S)$.  Then there is some $v' \in S$ such that $f(v) = f(v')$.  Note that
%% $\unitf \circ f = \unitf$: each maps elements of~$T$ to~$\unit$, and is lifted
%% in the normal way to other values.  Hence 
%% \[
%% \unitf(v) = \unitf(f(v)) = \unitf(f(v')) = \unitf(v').
%% \]
%% Hence $v \in S$, since $S$ is $T$-invariant.
%% \end{proof}


%% The following definition captures a property of all sets of dotted values that
%% are defined at the top level.  It states that after two prefixes that map
%% under~$f$ to the same value, the extensions also map under~$f$ to the same
%% value. 
%% %
%% \begin{definition}
%% A set $S$ \emph{respects} $f$ if whenever $f(v) = f(v')$, then
%% \begin{eqnarray*}
%% \set{f(w) \| v.w \in S} & = & \set{f(w') \| v'.w' \in S}.
%% \end{eqnarray*}
%% \end{definition}

%% \begin{note}
%% The set $S = \set{x.x \| x \in T_1}$ respects~$f$: suppose $f(v) = f(v')$;
%% then
%% \[
%% \set{f(w) \| v.w \in S} = \set{f(v)} = \set{f(v')} =
%%    \set{f(w') \| v'.w' \in S}.
%% \]

%% However, the set $S' = \set{ x.y.(x=y) \| x \in T_1, y \in T_1}$ does not
%% respect~$f$ for non-injective~$f$.  Suppose $f(A) = f(B) = C$, then taking $v
%% = A.A$ and $v' = A.B$, we have $f(v) = f(v')$, but
%% \[
%% \set{f(w) \| v.w \in S} = \set{true} \ne \set{false} = 
%%   \set{f(w') \| v'.w' \in S}.
%% \]
%% The set~$S'$ is one that can be produced under the assumptions of the symmetry
%% reduction paper~\cite{symmetry}.  Hence we cannot use the result there to
%% deduce that all top-level sets in the current setting respect~$f$. 
%% \end{note}

%% \begin{definition}
%% We say that environment~$\rho$ respects $f$ if:
%% %% s Let $\rho_1$ be the top-level environment formed by processing the
%% %% top-level declarations starting with environment $\set{t \mapsto T_1}$.
%% %% Then
%% \begin{itemize}
%% \item For every set-valued identifier $s$,\, $\rho(s)$ respects~$f$;

%% \item If $\rho(c) = \channel S$ then $S$ respects~$f$;

%% \item If $\rho(d) = \dtcons S$ then $S$ respects~$f$.
%% \end{itemize}

%% We say that $\rho$ respects $f$ for channels and datatype constructors if the
%% latter two conditions hold.
%% \end{definition}

%% We will show that the environment formed from the top-level declarations
%% for~$T_1$ respects~$f$.  Hence every subsequent environment respects~$f$ for
%% channels and datatype constructors (since channels and datatypes are declared
%% only at the top level, so subsequent environments inherit the corresponding
%% bindings).

%% \framebox{Check.}  Do we need datatype constructors? 

%% \begin{lemma}
%% Suppose $S$ respects~$f$.  Then 
%% \begin{eqnarray*}
%% f(v).w' \in f(S) & \implies & \exists w \spot f(w) = w' \land v.w \in S.
%% \end{eqnarray*}
%% \end{lemma}
%% %
%% \begin{proof}
%% Assume the premise of the implication.  Then there is some $v'.w'' \in S$ such
%% that $f(v') = f(v)$ and $f(w'') = w'$.  But then because~$S$ respects~$f$,
%% there is some~$w$ such that $v.w \in S$ and $f(w) = f(w'') = w'$.   
%% \end{proof}

%% \begin{lemma}
%% \label{lem:respects}
%% Suppose $S$ respects~$f$.  Then for every~$v$,
%% \begin{eqnarray*}
%% \set{f(w) \| v.w \in S} & = & \set{f(w) \| f(v).f(w) \in f(S)}.
%% \end{eqnarray*}
%% \end{lemma}

%% \begin{proof}
%% %\textbf{($\subseteq$)}  
%% Suppose $w'$ is an element of the left-hand side.
%% Then $w' = f(w)$ for some~$w$ such that $v.w \in S$.  But then $f(v).f(w) \in
%% f(S)$, so $w'$ is an element of the right-hand side.

%% %\textbf{($\supseteq$)}  
%% Conversely, suppose $w'$ is an element of the right-hand side.  Then $w' =
%% f(w)$ for some~$w$ such that $f(v).f(w) = f(v).w' \in f(S)$.  Then there is
%% some $v'.w'' \in S$ such that $f(v') = f(v)$ and $f(w'') = w'$.  But $S$
%% respects~$f$, so there is some~$w$ such that $v.w \in S$ and $f(w) = f(w'') =
%% w'$.  So $w'$ is an element of the left-hand side.
%% \end{proof}



%% The following definition captures that the set of values associated with a
%% channel or a datatype constructor is symmetric in values from~$T$.
%% %
%% \begin{definition}
%% Let $T$ be an instantiation for~$t$, and let $\rho : Var \pfun Value$ be an
%% environment.  We say that $\rho$ is $T$-symmetric if for every bijection~$\pi: T
%% \fun T$:
%% \begin{itemize}
%% \item If $\rho(c) = \channel S$ then $\pi(S) = S$;

%% \item If $\rho(A) = \dtcons S$ then $\pi(S) = S$.
%% \end{itemize}
%% \end{definition}
%% %
%% Channel and datatype declarations may appear only as top-level declarations.
%% In~\cite{symmetry-reduction}, we showed (under assumptions weaker than data
%% independence) that the environment~$\rho_1$ formed from these declarations is
%% $T$-symmetric, for each instantiation~$T$ of~$t$.  Hence each subsequent
%% environment is also $T$-symmetric.

%% Result: if $\rho$ is $T$-symmetric then $v \in \rho(c) \iff f(v) \in
%% (f\after\rho)(c)$.  \framebox{No!}  Suppose $\rho(c) = \channel \set{w.w | w
%%   \in T}$,\, $v = v_1.v_2$ with $v_1 \ne v_2$, $f(v_1) = f(v_2)$.  Then RHS
%% true but LHS false.


%% \begin{conjecture}
%% Let $\rho_1$ be the environment formed from the top-level declarations.  Let
%% $c$ be a channel.  If $v \in \rho(c)$ and $f(v) = f(v')$ then $v' \in
%% \rho(c)$.

%% This seems to depend on the channel declaration being rectangular.
%% \end{conjecture}
